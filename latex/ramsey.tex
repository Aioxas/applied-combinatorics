% ramsey.tex
% Updated January 11, 2012

\chapter{Applying Probability to Combinatorics}\label{ch:probmeth}

Bob likes to think of himself as a wild and crazy guy, totally 
unpredictable.  Most guys do.  But Alice says that Bob
can't change his basic nature, which is excruciatingly
boring.  Carlos remarks that perhaps we shouldn't be
so hard on Bob, because under certain circumstances, we
can all be forced to be dull and repetitive. 

\subsection{The Pigeon Hole Principle Revisited}\label{s:ramsey:php-revisit}

Recall that when $n$ is a positive integer, 
we let $[n]=\{1,2,\dots,n\}$.  In this chapter, when $X$ is
a set and $k$ is a non-negative integer with $k\le |X|$, we
borrow from our in-line notation for binomial coefficients
and let $C(X,k)$ denote the family of all $k$-element subsets
of $X$.  So $|C([n],k)|=C(n,k)$ whenever $0\le k\le n$.

Recall that the pigeon hole principle asserts that if
$n+1$ pigeons are placed in $n$ holes, then there must
be some hole into which two or more pigeons have been
placed.  More formally, if $n$ and $k$ are positive integers,
$t>n(k-1)$ and $f:[t]\longrightarrow[n]$ is any function,
then there is a $k$-element subset $H\subseteq [t]$ and
an element $j\in[n]$ so that $f(i)=j$ for every $i\in H$.

We now embark on a study of an elegant extension of this
basic result, one that continues to fascinate and challenge.

\subsection{A First Taste of Ramsey Theory}

Returning to the discussion at the start of this section,
you might say that an induced subgraph $H$ of a graph $G$ is
``boring'' if it is either a complete subgraph or
an independent set.  In either case, exactly every pair
of vertices in $H$ behaves in exactly the same boring way.
So is boredom inevitable?  The answer is yes---at least
in a relative sense.  As a starter, let's show that
any graph on six (or more) vertices has a boring subgraph
of size three.

\begin{lemma}\label{lem:r33}
Let $G$ be any graph with six of more vertices.
Then either $G$ contains a complete subgraph of size~$3$ or
an independent set of size~$3$.
\end{lemma}
\begin{proof}
Let $x$ be any vertex in $G$.  Then 
split the remaining vertices into two sets
$S_1$ and $S_2$ with $S_1$ being the neighbors of
$x$ and $S_2$ the non-neighbors.  Since $G$ has
at least six vertices, we know that either $|S_1|\ge 3$
or $|S_2|\ge 3$.  Suppose first that $|S_1|\ge 3$ and
let $y_1$, $y_2$ and $y_3$ be distinct vertices from
$S_1$.  If $y_iy_j$ is an edge in $G$ for some distinct
pair $i, j\in\{1,23\}$, then $\{x,y_i,y_j\}$ is a complete
subgraph of size $3$ in $G$.  On the other hand, if
there are no edges among the vertices in $\{y_1,y_2,y_3\}$,
then we have an independent set of size~$3$.

The argument when $|S_2|\ge3$ is dual.
\end{proof} 
We note that the bound of six in the preceding lemma is sharp,
as a cycle on five vertices does not contain either a complete
set of size~$3$ nor an independent set of size~$3$.

Next, here is the general statement for a result, which is
usually called the graph version of Ramsey's theorem.

\begin{theorem}\label{thm:graphramsey}
If $m$ and $n$ are positive integers, then there exists
a least positive integer $R(m,n)$ so that if $G$ is a graph
and $G$ has at least $R(m,n)$ vertices, then either $G$ contains
a complete subgraph on $m$ vertices, or $G$ contains an
independent set of size $n$.
\end{theorem}
\begin{proof}
We show that $R(m,n)$ exists and is at most $\binom{m+n-2}{m-1}$.
This claim is trivial when either $m\le 2$ or $n\le2$, so we
may assume that $m,n\ge3$.  From this point, we proceed by induction
on $t=m+n$ assuming that the result holds when $t\le 5$.

Now let $x$ be any vertex in $G$.  Then there are at least
$\binom{m+n-2}{m-1}-1$ other vertices, which we partition
as $S_1\cup S_2$, where $S_1$ are those vertices adjacent to $x$
in $G$ and $S_2$ are those vertices which are not adjacent to
$s$.

We recall that the binomial coefficients satisfy
\[
\binom{m+n-2}{m-1}=\binom{m+n-3}{m-2}+\binom{m+n-3}{m-1} =
\binom{m+n-3}{m-2}+\binom{m+n-3}{n-2}
\]
So either $|S_1|\ge \binom{m+n-3}{m-2}$ or $|S_1|\ge\binom{m+n-3}{m-1}$.
If the first option holds, and $S_1$ does not have an independent set
of size $n$, then it contains a complete subgraph of size $m-1$.  It follows
that we may add $x$ to this set to obtain a complete subgraph of size $m$
in $G$.

Similarly, if the second option holds, and $S_2$ does not contain a
complete subgraph of size $m$, then $S_2$ contains an independent
set of size $n-1$, and we may add $x$ to this set to obtain an
independent set of size $n$ in $G$.
\end{proof} 

\section{Small Ramsey Numbers}\label{s:ramsey:small}

Actually determing the Ramsey numbers $R(m,n)$ referenced in \autoref{thm:graphramsey}
seems to be a notoriously difficult problem, and only a handful of these
values are known precisely.  In particular, $R(3,3)=6$ and $R(4,4)=18$, while
$43\le R(5,5)\le 49$.  The distinguished Hungarian mathematician
Paul Erd\"os said on many occasions that it might be possible
to determine $R(5,5)$ exactly, if all the world's mathematical
talent were to be focused on the problem.  But he also said that
finding the exact value of $R(6,6)$ might be beyond our collective
abilities.

In the following table, we provide information about
the ramsey numbers $R(m,n)$ when $m$ and $n$ are at least $3$ and at
most $9$.   When a cell contains a single number, that
is the precise answer.  When there are two numbers, they
represent upper and lower bounds.

\begin{center}
\begin{tabular}{||cc||c|c|c|c|c|c|c||}
\hline\hline
&n&3&4&5&6&7&8&9\\
m&&&&&&&&\\
\hline\hline
3&&6&9&14&18&23&36&39\\
4&&&18&25&35, 41&49, 61&56, 84&69, 115\\
5&&&&43, 49&58, 87&80, 143&95, 216&121, 316\\
6&&&&&102, 165&111, 298&127, 495&153, 780\\
7&&&&&&205, 540&216, 1031&216, 1713\\
8&&&&&&&282, 1870&282, 3583\\
9&&&&&&&&565, 6588\\
\hline\hline
\end{tabular}
\end{center}

For additional data, do a web search and look for
Stanley Radziszowski, who maintains the most current information
on his web site.

\section{Estimating Ramsey Numbers}\label{s:ramsey:estimatesize}

We will find it convenient to utilize the following approximation
due to Stirling.  You can find a proof in almost any
advanced calculus book.

\[
n!\equiv \sqrt{2\pi n} \bigl( \frac{n}{e}\bigr)^n\bigl(1+
  \frac{1}{12n}+\frac{1}{288n^2}-\frac{139}{51840n^3} +O(\frac{1}{n^4})\bigr).
\]
Of course, we will normally be satisfied with the
first term:
\[
n!\equiv \sqrt{2\pi n} \bigl( \frac{n}{e}\bigr)^n
\]
Using Stirling's approximation, we have the following
upper bound:
\[
R(n,n) \le \binom{2n-2}{n-1} \equiv \frac{2^{2n}}{4\sqrt{\pi n}} 
\]


\section{Applying Probability to Ramsey Theory}\label{s:ramsey:probability}

The following theorem, due to P. Erd\"os, is a true classic,
and is presented here in a manner that is faithful to how it
was first published.  As we shall see later, it was subsequently
recast---but that's getting the cart ahead of the horse.

\begin{theorem}\label{thm:rnn}
\[
R(n,n) \ge \frac{n}{e\sqrt2} 2^{\frac{1}{2}n}
\]
\end{theorem}

\begin{proof}
Let $t$ be an integer with $t>n$ and consider the set $\cgF$ of
all labeled graphs with vertex set $\{1,2,\dots,t\}$.  Clearly,
there are $2^{C(t,2)}$ graphs in this family.  Let $\cgF_1$ denote the
subfamily consisting of those graphs which contain a complete subgraph of
size~$n$.  It is easy to see that
\[
|\cgF_1|\le \binom{t}{n}2^{n(t-n)}2^{C(t-n,2)}.
\]
Similarly, let $\cgF_2$ denote the subfamily consisting of those graphs which 
contain an independent set of size~$n$.  It follows that
\[
|\cgF_2|\le \binom{t}{n}2^{n(t-n)}2^{C(t-n,2)}.
\]
We want to take the integer $t$ as large as we can while still guaranteeing that
$|\cgF_1|+|\cgF_2|\le |\cgF|$.  This will imply that there is a graph
$G$ in $\cgF$ which does not contain a complete subgraph of size $n$ or
an independent set of size~$n$.   So consider the following inequality:
\begin{equation}\label{eqn:rnn}
2\binom{t}{n}2^{n(t-n)}2^{C(t-n,2)}<2^{C(t,2)}.
\end{equation}

Now we ask how large can $t$ be without violating inequality~\ref{eqn:rnn}?
To answer this, we use the trivial inequality $\binom{t}{n}\le t^n/n!$ and the
use the Stirling approximation for $n!$.  After some algebra and taking the
$n^{\text{th}}$ root of both sides, we see 
that we need only guarantee that 
\[
t\le \frac{n}{e\sqrt{n}}2^{\frac{1}{2}n} 
\]
\end{proof}

Now let's take a second look at the proof of \autoref{thm:rnn}.
We consider a probability space $(S,P)$ where the
outcomes are graphs with vertex set $\{1,2,\dots,t\}$.  For each
$i$ and $j$ with $1\le i < j\le t$, edge $ij$ is present
in the graph with probability $1/2$.  Furthermore, the
events for distinct pairs are independent.

Let $X_1$ denote the random variable which counts the number
of $n$-element subsets of $\{1,2,\dots,t\}$ for which all
$\binom{n}{2}$ pairs are edges in the graph.  Similarly,
$X_2$ is the random variable which counts the number
of $n$-element independent subsets of $\{1,2,\dots,t\}$.
Then set $X=X_1+X_2$.

By linearity of expectation, $E(X)=E(X_1)+E(X_2)$ while

\[
E(X_1)=E(X_2) = \binom{t}{n} \frac{1}{2^{C(n,2)}}.
\]

If $E(X)<1$, then there must exist a graph with vertex
set $\{1,2,\dots,t\}$ without a $K_n$ or an $I_n$.
And the question of how large $t$ can be while maintaining
$E(X)<1$ leads to exactly the same calculation we had before.

After more than fifty years and the efforts of many very bright
researchers, only marginal improvements have been made on the
bounds on $R(n,n)$ from \autoref{thm:graphramsey} and 
\autoref{thm:rnn}.  In particular, no one can settle
whether there is some constant $c<2$ and an integer $n_0$ so that $R(n,n)<
2^{cn}$ when $n>n_0$.  Similarly, no one has been able to answer
whether there is some constant $d>1/2$ and an integer $n_1$ so
that $R(n,n)>2^{dn}$ when $n>n_1$.  We would certainly give
you an $A$ for this course if you managed to do either.

Carlos said that he had been trying to prove a good lower bound
on $R(n,n)$ using only constructive methods, i.e., no random
techniques allowed.  But he was having problems.  Anything
he tried seemed only to show that $R(n,n)\ge n^c$ where $c$
is a constant.  That seems so weak compared to the exponential
bound which the probabilistic method gives easily.  Usually
Alice was not very sympathetic to the complaints of others and
certainly not from Carlos, who seemed always to be out front.
But this time, Alice said to Carlos and in a manner that all could
hear ``Maybe you shouldn't be so hard on yourself.  I read an
article on the web that nobody has been able to show that
there is a constant $c>1$ and an integer $n_0$ so that $R(n,n)>c^n$
when $n>n_0$, provided that only constructive methods are allowed.
And maybe, just maybe, saying that you are unable to do
something that lots of other famous people seem also unable
to do is not so bad.''  Bob saw a new side of Alice and this
too wasn't all bad.

\section{Ramsey's Theorem}\label{s:ramsey:theorem}

By this time, you are probably not surprised to see that there
is a very general form of Ramsey's theorem.  We have a bounded
number of bins or colors and we are placing the subsets of a
fixed size into these categories.  The conclusion is that there
is a large set which is treated uniformly.

Here's the formal statement.

\begin{theorem}\label{thm:genramsey}
Let $r$ and $s$ be positive integers and
let $\mathbf{h}=(h_1,h_2,\dots,h_r)$ be a string
of integers with $h_i\ge s$ for each $i=1,2,\dots,s$. 
Then there exists a least positive integer $R(s:h_1,h_2,\dots,h_r)$
so that if $n\ge n_0$ and $\phi:C([n],s]\longrightarrow [r]$
is any function, then there exists an integer $\alpha\in[r]$
and a subset $H_\alpha\subseteq [n]$ with $|H_\alpha|=h_\alpha$
so that $\phi(S)=\alpha$ for every $S\in C(H_\alpha, s)$.
\end{theorem}
We don't include the proof of this general statement here, but
the more ambitious students may attempt it on their own.  Note
that the case $s=1$ is just the Pigeon Hole principle, while the
case $s=r=2$ is just the graph version of Ramsey's theorem,
as established in \autoref{thm:graphramsey}.  An argument
using double induction is required for the proof in the general
case.  The first induction is on $r$ and the second is on $s$.

\section{The Probabilistic Method}\label{s:ramsey:girth}

At the outset of this chapter, we presented Erd\H{o}s' original proof
for the lower bound for the Ramsey number $R(n,n)$ using counting.
Later, we recast the proof in a probabilistic setting.  History has
shown that this second perspective is the right one. To illustrate 
the power of this approach, we present a classic 
theorem, which is also due to Erd\H{o}s, showing that there
are graphs with large girth and large chromatic number.
 
The \textit{girth} $g$ of a graph $G$ is the smallest integer for
which $G$ contains a cycle on $g$ vertices.  The girth of a forest is
taken to be infinite, while the girth of a graph is three if and only
if it has a triangle.  You can check the families of triangle-free,
large chromatic number, graphs constructed in \autoref{ch:graphs} and
see that each has girth four.

\begin{theorem}[Erd\H{o}s]\label{thm:girth}
For every pair $g,t$ of integers with $g\ge3$, there exists
a graph $G$ with $\chi(G)>t$ and the girth of $G$ greater than $g$.
\end{theorem}

\begin{proof}
Before proceeding with the details of the argument, let's pause
to get the general idea behind the proof.  We choose integers
$n$ and $s$ with $n>s$, and it will eventually be clear how
large they need to be in terms of $g$ and $t$.  We will
then consider a random graph on vertex set $\{1,2,\dots,n\}$,
and just as before, for each $i$ and $j$ with $1\le i<j\le n$,
the probability that the pair $ij$ is an edge is $p$, but now
$p$ will depend on $n$.  Of course, the probability that any
given pair is an edge is completely independent of all other
pairs.

Our first goal is to choose the values of $n$, $s$ and $p$ so that
with high probability, a random graph does not have an independent
set of size $s$.  You might think as a second goal, we would try
to get a random graph without small cycles.  But this goal is too
restrictive.  Instead, we just try to get a graph in which there
are relatively few small cycles.  In fact, we want the number
of small cycles to be less than $n/2$.  Then we will remove one
vertex from each small cycles, resulting in a graph with at
least $n/2$ vertices, having no small cycles and no independent
set of size~$s$.  The chromatic number of this graph is at least
$n/2s$, so we will want to have the inequality $n>2st$.

Now for some details.  Let $X_1$ be the random variable that counts
the number of $s$-element independent sets.  Then
\[
E(X_1)=\binom{n}{s}(1-p)^{C(s,2)}
\]
Now we want $E(X_1)<1/4$.  Since $C(n,s)\le n^s=e^{s\ln n}$ and
$(1-p)^{C(s,2)}\le e^{-ps^2/2}$, it suffices to set
$s=2\ln n/p$.  By Markov's law, the probability that $X_1$ exceeds
$1/2\ge 2E(X_1)$ is less than $1/2$.

Now let $X_2$ count the number of cycles in $G$ of size at most $g$.
Then
\[
E(X_2)\le \sum_{i=3}^g n(n-1)(n-2)\dots(n-i+1) p^i\le g(pn)^g.
\]
Now, we want $E(X_2)\le n/4$, and an easy calculation shows that
$g(np)^g\le n/4$ when $p=n^{1/g-1}/10$.  Again by Markov's Law,
the probability that $X_2$ exceeds $n/2\ge 2E(X_2)$ is less than $1/2$.

We conclude that there is a graph $G$ for which $X_1=0$ and $X_2\le n/2$.
Remove a vertex from each of the small cycles in $G$ and let $H$ be
the graph that remains.  Clearly, $H$ has at least $n/2$ vertices,
no cycle of size at most~$g$ and no independent set of size $s$. 
Finally, the inequality $n>2st$ requires $n^{1/g}/(40\ln n)>t$.
\end{proof}

\subsection{Gaining Intuition with the Probabilistic Method}

Experienced researchers are able to simplify the calculations in
an argument of this type, as they know what can safely be discarded
and what can not.   Here's a quick tour of the essential steps.
We want $E(X_1)$ to be small, so we set $n^se^{-ps^2}=1$ and
get $s=\ln n/p$.  We want the number of small cycles to be about
$n$ so we set $(gp)^g=n$ and get $p=n^{1/g-1}$.  Finally, we want
$n=st$ which requires $n^{1/g}=t$.  The rest is just paying attention
to details.

\section{Discussion}\label{s:ramsey:discussion}

Zori started the conversation with ``Who in their right
mind would trust their lives to an algorithm that used random
methods?''  Xing quickly responded ``Everyone.  At least everyone
should.  We routinely deal with probabilistic concepts, like
getting run over by a bus when crossing the street or having
a piano fall on our head.  The general public is much more
comfortable with notions of probability, even though they
may never know the formal definition of a probability space.
I for one am completely comfortable taking an airline flight
if I can be assured that the probability of a disaster is
less than $10^{-20}$.''

Dave wasn't biting on this topic.  Instead he offered ``You have
to be struck by the statements that it appears difficult to
construct objects which you can prove exist in abundance. I wonder
why this is so.''  Alice said ``We all find your brain to be
a totally random thing, sometimes making sense but often not.''
There was laughter or at least some snickering.  But after a bit,
Carlos said ``There's something fundamental here.  Maybe one
could prove that there are easily stated theorems which
only have long proofs.'' Bob blurted ``That doens't make any
sense.''  Zori saw an opportunity where a client would,
at considerable expense, commission her to solve a problem
(at least better than the competition) that was readily understood 
but somehow difficult in the end.   She knew about the class
$NP$ but maybe there were even bigger challenges (and bigger
paychecks) out there.

\section{Exercises}\label{s:ramsey:exercises}

\begin{enumerate}
\item  Consider a random graph with vertex set $\{1,2,\dot,n\}$.
If the edge probability is $p=1/2$, then let $X$ denote the number
of complete subgraphs is size $t=2\log n$ and let $Y$ denote the
number of independent sets of size $t=2\log n$.
\begin{enumerate}
\item  Show that $E(X+Y)<1$, when $n$ is sufficiently large.
\item  Use the result from part~a to show that $\omega(G)$ is less
than $2\log n$, while the chromatic number of $G$ is at least $n/(2\log n)$
(both statements holding with high probability).  As a result, the basic
inequality $\chi(G)\ge\omega(G)$ is far from being tight for a random
graph.
\end{enumerate}
\item  We form a random tournament as follows.  Start with a complete graph 
with vertex set $\{1,2,\dots,n\}$.  For each distinct
pair $i$, $j$ with $1\le i<j\le n$, flip a fair coin.  If the result is heads,
orient the edge from $i$ to $j$, which we denote by $(x,y)$.  If the toss is
tails, then the edge is oriented from $j$ to $i$, denoted $(y,x)$.
Show that when $n$ is large, with high probability, the following 
statement is true: For every set $S$ of size $\log n/10$, there is a vertex 
$x$ so that $(x,y)$ in $T$ for every $y\in S$.
\item  Let $T$ be a random tournament on $n$ vertices.  Show that with
high probability, the following statement is true: For every pair $x$, $y$
of distinct vertices, either (1)~$(x,y)$ in $T$, or (2)~there is a vertex
$z$ for which both $(x,z)$ and $(z,y)$ are in $T$.
\item Many statements for random graphs exhibit a threshold behavior.
Show that a random graph with edge probability $p=10\log n/n$ almost
certainly has no isolated vertices, while a random graph with edge probability 
$p=\log n/(10 n)$ almost certainly has at least one isolated vertices.
\item In the sense of the preceding problem, determine the threshold probability
for a graph to be connected.
\end{enumerate}

%%% Local Variables: 
%%% mode: latex
%%% TeX-master: "book"
%%% End: 
%%% Local Variables: 
%%% mode: latex
%%% TeX-master: "book"
%%% End: 
