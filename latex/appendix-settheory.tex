% appendix-settheory.tex
% Updated January 4, 2012

\chapter{Set Theory for Combinatorics}\label{app:settheory}

This appendix treats background material essential to the
study of combinatorial mathematics.  Students will find that
most---and perhaps all---of this material has been covered somewhere
in their prior course work, and we expect that very few instructors
will include this appendix in the syllabus.  Nevertheless, students
may find it convenient to consult this appendix from time to
time when studying functions and relations.

\section{Introduction}

Set theory is concerned with \textit{elements}, certain collections
of elements called \textit{sets} and a concept of \textit{membership}.
For each element $x$ and each set $X$, \textit{exactly} one of the 
following two statements holds:

\begin{enumerate}
\item $x$ is a member of $X$. 
\item $x$ is \textit{not} a member of $X$. 
\end{enumerate}

It is important to note that membership cannot be ambiguous.

When $x$ is an element and $X$ is a set, we write $x\in X$ when
$x$ is a member of $X$.  Also, the statement $x$ belongs to $X$ means
exactly the same thing as $x$ is a member of $X$.  Similarly, when
$x$ is not a member of $X$, we write $x\notin X$ and say $x$ does
not belong to $X$.

Certain sets will be defined explicitly by listing the elements.
For example, let $X=\{a,b,d,g,m\}$.  Then $b\in X$ and $h\notin X$.
The order of elements in such a listing is irrelevant, so we could
also write $X=\{g,d,b,m,a\}$.  In other situations, sets will be 
defined by giving a rule for membership.
As examples, let $\posints$ denote the set of positive integers.  
Then let $X=\{n\in\posints:5\le n\le 9\}$.   Note that $6,8\in X$ while
$4,10,238\notin X$.

Given an element $x$ and a set $X$, it may at times be tedious and
perhaps very difficult to determine which of the statements
$x\in X$ and $x\notin X$ holds.  But if we are discussing sets,
it must be the case that \textit{exactly} one is true.

\begin{example}
Let $X$ be the set consisting of the following $12$ positive
integers:

\begin{align*}
&13232112332\\
&13332112332\\
&13231112132\\
&13331112132\\
&13232112112\\
&13231112212\\
&13331112212\\
&13232112331\\
&13231112131\\
&13331112131\\
&13331112132\\
&13332112111\\
&13231112131
\end{align*}
Note that one number is listed twice.  Which one is it?
Also, does $13232112132$ belong to $X$?  Note that the
apparent difficulty of answering these questions stems from
(1) the size of the set $X$; and (2) the size of the integers 
that belong to $X$.  Can you think of circumtances in which
it is difficult to answer whether $x$ is a member of $X$
even when it is known that $X$ contains exactly one element?
\end{example}

\begin{example}
Let $P$ denote the set of primes.  Then $35\notin P$ since
$35= 5\times 7$.  Also, $19\in P$.  Now consider the number
\[
n = 77788467064627123923601532364763319082817131766346039653933
\]
Does $n$ belong to $P$?  Alice says yes while Bob says no.
How could Alice justify her affirmative answer?  How could Bob
justify his negative stance?  In this specific case, I know that
Alice is right.  Can you explain why?
\end{example}

\section{Intersections and Unions}

When $X$ and $Y$ are sets, the \textit{intersection}
of $X$ and $Y$, denoted $X\cap Y$, is defined by
\[
X\cap Y = \{x: x\in X, x\in Y\}
\]
Note that this notation uses the convention followed by
many programming languages. Namely, the ``comma'' in the
definition means that \textit{both} requirements for
membership be satisfied.  For example, if $X=\{b,c,e,g,m\}$
and $Y=\{a,c,d,h,m,n,p\}$, then $X\cap Y=\{c,m\}$.

\subsection{The Meaning of $2$-Letter Words}

In the not too distant past, there was considerable discussion
in the popular press on the meaning of the $2$-letter word \textit{is}.
For mathematicians and computer scientists, it would have
been far more significant to have a discussion of the $2$-letter
word \textit{or}.  The problem is that the English language uses
\textit{or} in two fundamentally different ways. Consider the
following sentences:

\begin{enumerate}
\item  A nearby restaurant has a dinner special featuring two choices for
dessert: flan de casa \textbf{or} tirami-su.
\item  A state university accepts all students who have graduated
from in-state high schools and have SAT scores above $1000$ \textbf{or} 
have grade point averages above $3.0$.
\item A local newpaper offers 
customers the option of paying their for their newspaper bills on a monthly
\textbf{or} semi-annual basis.
\end{enumerate}
In the first and third statement, it is clear that there are
two options but that \textit{only} one of them is allowed.  However,
in the second statement, the interpretation is that admission will
be granted to students who satisfy \textit{at least one} of the
two requirements.  These interpretations are called 
respectively the \textit{exclusive}
and \textit{inclusive} versions of \textbf{or}.  In this class, we
will assume that whenever the word ``or'' is used, 
the inclusive interpretation is intended---unless otherwise
stated.

For example, when $X$ and $Y$ are sets, the \textit{union}
of $X$ and $Y$, denoted $X\cup Y$, is defined by
\[
X\cup Y = \{x: x\in x \text{ or } x\in Y\}.
\]
membership be satisfied.  For example, if $X=\{b,c,e,g,m\}$
and $Y=\{a,c,d,h,m,n,p\}$, then $X\cup Y=\{a,b,c,d,e,g,h,m,n,p\}$.

Note that $\cap$ and $\cup$ are \textit{commutative} and
\textit{associative} binary operations, as is the case with
addition and multiplication for the set $\posints$ of positive
integers, i.e., if $X$, $Y$ and $Z$ are sets, then
\[
X\cap Y = Y\cap X \quad\text{ and }\quad X\cup Y = Y\cup X. 
\]
Also,
\[
X\cap(Y\cap Z)= (X\cap Y)\cap Z\quad\text{ and }\quad
X\cup(Y\cup Z)= (X\cup Y)\cup Z.
\]
Also, note that each of $\cap$ and $\cup$ distributes
over the other, i.e., 
\[
X\cap(Y\cup Z)= (X\cap Y)\cup (X\cap Z)\quad\text{ and }\quad
X\cup(Y\cap Z)= (X\cup Y)\cap (X\cup Z)
\]
On the other hand, in $\posints$, multiplication distributes
over addition but not vice-versa.

\subsection{The Empty Set: Much To Do About Nothing}

The \textit{empty set}, denoted $\emptyset$ is the set for
which $x\notin \emptyset$ for every element $x$.  Note that
$X\cap \emptyset =\emptyset$ and $X\cup \emptyset=X$, for every
set $X$. 

The empty set is unique in the sense that if $x\notin X$ for every
element $x$, then $X=\emptyset$.

\subsection{The First So Many Positive Integers}

In our course, we will use the symbols $\posints$, $\ints$,
$\rats$ and $\reals$ to denote respectively the
set of positive integers, the set of all integers (positive, negative
and zero), the set of rational numbers (fractions) and the set of real
numbers (rationals and irrationals).  On occasion, we will discuss
the set $\nonnegints$ of \textit{non-negative integers}.
When $n$ is a positive integer, we will use the 
abbreviation $[n]$ for the set $\{1,2,\dots,n\}$ of the
first $n$ positive integers.  For example, $[5]=\{1,2,3,4,5\}$.
For reasons that may not be clear at the moment but hopefully
will be transparent later in the semester, we use the
notation $\bfn$ to denote the $n$-element set $\{0,1,2,\dots,n-1\}$.
Of course, $\bfn$ is just the set of the first $n$ non-negative
integers.  For example, $\mathbf{5}=\{0,1,2,3,4\}$.

\subsection{Subsets, Proper Subsets and Equal Sets} 

When $X$ and $Y$ are sets, we say $X$ is a \textit{subset} of $Y$ and
write $X\subseteq Y$ when $x\in Y$ for every $x\in X$.  When $X$ is a
subset of $Y$ and there exists at least one element $y\in Y$ with
$y\notin X$, we say $X$ is a proper subset of $Y$ and write
$X\subsetneq Y$.  For example, the $P$ of primes is a proper subset of
the set $\posints$ of positive integers.

Surprisingly often, we will encounter a situation where
sets $X$ and $Y$ have different rules for membership
yet both are in fact the same set.  For example, let
$X=\{0,2\}$ and $Y=\{z\in\ints: z+z=z\times z\}$. Then $X=Y$.
For this reason, it is useful to have a test when sets are
equal.  If $X$ and $Y$ are sets, then

\[
X = Y \quad\text{ if and only if }\quad X\subseteq Y \text{ and } Y\subseteq X.
\]  

\section{Cartesian Products} 

When $X$ and $Y$ are sets, the \textit{cartesian product}
of $X$ and $Y$, denoted $X\times Y$, is defined by
\[
X\times Y=\{(x,y): x\in X \text{ and } y\in Y\}
\]
For example, if $X=\{a,b\}$ and $Y=[3]$, then
$X\times Y=\{(a,1),(b,1),(a,2),(b,2),(a,3),(b,3)\}$.
Elements of $X\times Y$ are called \textit{ordered pairs}.
When $p=(x,y)$ is an ordered pair, the element $x$ is referred
to as the \textit{first coordinate} of $p$ while $y$ is the
\textit{second coordinate} of $p$.
Note that if either $X$ or $Y$ is the empty set, then
$X\times Y=\emptyset$.

\begin{example}
Let $X=\{\emptyset,(1,0),\{\emptyset\}\}$ and $Y=\{(\emptyset,0)\}$.
Is $((1,0),\emptyset)$ a member of $X\times Y$?
\end{example}

Cartesian products can be defined for more than two factors.
When $n\ge 2$ is a positive integer and $X_1,X_2,\dots,X_n$ are
non-empty sets, their \textit{cartesian product} is defined
by
\[
X_1\times X_2\times\dots\times X_n=\{(x_1,x_2,\dots,x_n): x_i\in X_i
\text{ for } i = 1,2,\dots,n\}
\]


\section{Binary Relations and Functions}

A subset $R\subseteq X\times Y$ is called a \textit{binary
relation} on $X\times Y$, and a binary relation $R$ on $X\times Y$ 
is called a \textit{function from} $X$ \textit{to} $Y$ when the
following condition is satisfied:

\medskip
\noindent
$C$:\quad For every $x\in X$, there is \textit{exactly one}
element $y\in Y$ for which $(x,y)\in R$.

\medskip
Many authors prefer to write Condition~\textbf{C} in two parts:

\medskip
\noindent
$C_1$:\quad For every $x\in X$, there is \textit{some}
element $y\in Y$ for which $(x,y)\in R$.

\medskip
\noindent
$C_2$:\quad For every $x\in X$, there is \textit{at most one}
element $y\in Y$ for which $(x,y)\in R$.

\medskip
And this last condition is often stated in the following alternative
form:

\medskip
\noindent
$C'_2$: \quad If $x\in X$, $y_1,y_2\in Y$ and $(x,y_1),(x,y_2)\in R$, then
$y_1=y_2$.

\begin{example}\label{exa:function}

For example, let $X=[4]$ and $Y=[5]$.  Then let

$R_1=\{(2,1),(4,2),(1,1),(3,1)\}$, 

$R_2=\{(4,2),(1,5),(3,2)\}$, and 

$R_3=\{(3,2),(1,4),(2,2),(1,1),(4,5)\}$.

\noindent
Then only $R_1$ is a function from $X$ to $Y$.
\end{example}

In many settings (like calculus), it is customary to use letters like $f$,
$g$ and $h$ to denote functions.  So let $f$ be a function from a
set $X$ to a set $Y$. In view of the defining properties of functions, for
each $x\in X$, there is a unique element $y\in Y$ with $(x,y)\in f$.
And in this case, the convention is to write $y=f(x)$.  For example,
if $f=R_1$ is the function in Example~\ref{exa:function}, then
$2=f(4)$ and $f(3) =1$.

The shorthand notation $f:X\rightarrow Y$ is used to indicate
that $f$ is a function from the set $X$ to the set $Y$.  

In calculus, we study functions defined by algebraic rules.
For example, consider the function $f$ whose rule is $f(x) = 5x^3-8x+7$.
This short hand notation means that $X=Y=\reals$ and that
\[
f=\{(x,5x^3-8x+7):x\in\reals\}
\]
In combinatorics, we sometimes study functions defined
algebraically, just like in calculus, but we will frequently
describe functions by other kinds of rules.  For example, let
$f:\posints\rightarrow\posints$ be defined by
$f(n) = |n/2|$ if $n$ is even and $f(n)=3|n|+1$ when $n$ is odd.

A function $f:X\rightarrow Y$ is
called an \textit{injection from} $X$ \textit{to} $Y$ when

\medskip
\noindent
$I$:\quad For every $y\in Y$, there is \textit{at most} one element 
$x\in X$ with $y=f(x)$.

\medskip
When the meaning of $X$ and $Y$ is clear, we just say $f$ is an 
\textit{injection}.  An injection is also called a $1$--$1$ function
(read this as ``one to one'') and this is sometimes denoted as
$f:X\injection Y$.

A function $f:X\rightarrow Y$ is called a \textit{surjection from}
$X$ \textit{to} $Y$ when:

\medskip
\noindent
$S$:\quad For every $y\in Y$, there is \textit{at least} one $x\in X$ with
$y=f(x)$.

\medskip
Again, when the meaning of $X$ and $Y$ is clear, we just say $f$ is an 
\textit{surjection}.  A surjection is also called an \textit{onto} 
function and this is sometimes denoted as\\
$f:X\surjection Y$.

A function $f$ from $X$ to $Y$ which is both an injection and a surjection
is called a \textit{bijection}.  Alternatively, a bijection is referred
to as a $1$--$1$, onto function, and this is sometimes denoted as
$f:X \bijection Y$.  A bijection is also called a 
$1$--$1$-\textit{correspondence}.

\begin{example}
Let $X=Y=\reals$.  Then let $f$, $g$ and $h$ be the
functions defined by

\begin{enumerate}
\item $f(x)=3x-7$.
\item $g(x)=3(x-2)(x+5)(x-7)$.
\item $h(x)=6x^2-5x+13$.
\end{enumerate}
Then $f$ is a bijection; $g$ is a surjection but not an
injection (\textit{Why}?); and $h$ is neither an injection nor a
surjection (\textit{Why}?).
\end{example}

\begin{proposition}
Let $X$ and $Y$ be sets. Then there is a bijection from $X$ to $Y$
if and only if there is a bijection from $Y$ to $X$.
\end{proposition}

\section{Finite Sets}

A set $X$ is said to be \textit{finite} when either (1) $X=\emptyset$; or
(2) there exists positive integer $n$ and a bijection $f:[n]\bijection X$.
When $X$ is not finite, it is called \textit{infinite}.  For example,
$\{a,\emptyset,(3,2),\posints\}$ is a finite set as is
$\posints\times\emptyset$.  On the other hand, $\posints\times
\{\emptyset\}$ is infinite.  Of course, $[n]$ and $\bfn$ are
finite sets for every $n\in\posints$.

\begin{proposition}
Let $X$ be a non-empty finite set.  Then there is a unique
positive integer $n$ for which there is a bijection $f:[n]\bijection X$.
\end{proposition}

In some cases, it may take some effort to determine whether a set
is finite or infinite.  Here is a truly
classic result.

\begin{proposition}
The set $P$ of primes is infinite.
\end{proposition}
\begin{proof}
Suppose that the set $P$ of primes is finite.  It is non-empty
since $2\in P$.  Let $n$ be the unique positive integer for
which there exists a bijection $f:[n]\rightarrow P$.  Then let 

\[
p=1+f(1)\times f(2)\times f(3)\times \dots\times f(n)
\]
Then $p$ is prime (\textit{Why}?) yet larger than any element of $P$.
The contradiction completes the proof.
\end{proof}

Here's a famous example of a set where no one knows if
the set is finite or not.

\begin{conjecture}  It is conjectured that the following
set is infinite:
\[
T=\{n\in\posints:n \text{ and } n+2 \text{ are both
primes }\}. 
\]
\end{conjecture}
This conjecture is known as the \textit{Twin Primes Conjecture}.
Guaranteed $\text{A}++$ for any student who can settle it!
 
\begin{proposition}\label{exe:sb}
Let $X$ and $Y$ be finite sets.  If there exists an injection
$f:X\injection Y$ and an injection $g:Y \injection X$, then
there exists a bijection $h:X \bijection Y$.
\end{proposition}

When $X$ is a finite non-empty set, the \textit{cardinality} of $X$,
denoted $|X|$ is the unique positive integer $n$ for which
there is a bijection $f:[n]\bijection X$.  Intuitively,
$|X|$ is the number of elements in $X$.  For example,
\[
|\{(6,2), (8,(4,\emptyset)), \{3,\{5\}\}\}|=3.
\]
By convention, 
the cardinality of the empty set is taken to be zero, and we write
$|\emptyset|=0$.

\begin{proposition}\label{prop:xy} 
If $X$ and $Y$ are finite non-empty sets, then
$|X\times Y| =|X|\times |Y|$.
\end{proposition}

\begin{remark}
The statement in the last exercise is an example of ``operator
overloading'', a technique featured in several programming
languages.  Specifically, the times sign $\times$ is used
twice but has different meanings.  As part of $X\times Y$, it
denotes the cartesian product, while as part of $|X|\times |Y|$,
it means ordinary multiplication of positive integers.  Programming
languages can keep track of the data types of variables and
apply the correct interpretation of an operator like $\times$
depending on the variables to which it is applied.
\end{remark}

We also have the following general form of Proposition~\ref{prop:xy}:
\[
|X_1\times X_2\times\dots\times X_n|=
|X_1|\times |X_2|\times\dots\times |X_n|
\]

\begin{theorem}\hfill\mbox{}\\
\begin{enumerate}
\item There is a bijection between any two of the following
infinite sets $\posints$, $\ints$ and $\rats$. 
\item There is an injection from $\rats$ to $\reals$.
\item There is no surjection from $\rats$ to $\reals$.
\end{enumerate}
\end{theorem}

\section{Notation from Set Theory and Logic}

In set theory, it is common to deal with statements 
involving one or more elements from the universe as
variables.  Here are some examples:
\begin{enumerate}
\item For $n\in\posints$, $n^2-6n+8=0$.
\item For $A\subseteq[100]$, $\{2,8,25,58,99\}\subseteq A$.
\item For $n\in \ints$, $|n|$ is even.
\item For $x\in \reals$, $1+1=2$.
\item For $m,n\in \posints$, $m(m+1)+2n$ is even.
\item For $n\in \posints$, $2n+1$ is even.
\item For $n\in \posints$ and $x\in\reals$, $n+x$ is
irrational.
\end{enumerate}
These statements may be true for some values of the variables
and false for others.  The fourth and fifth statements
are true for \textit{all} values of the variables, while
the sixth is false for all values.

Implications are frequently abbreviated using with
a double arrow $\Longrightarrow$; the quantifier $\forall$ means ``for all'' 
(or ``for every''); and the quantifier $\exists$ means 
``there exists'' (or ``there is'').  Some
writers use $\wedge$ and $\vee$ for logical ``and'' and
``or'', respectively.  For example, 
\[
\forall A,B\subseteq[4]\quad \bigl((1,2\in A) \wedge |B|\ge 3)\bigr)
\Longrightarrow\bigl((A\subseteq B)\vee (\exists n\in A\cup B, 
 n^2=16)\bigr)
\]
The double arrow $\iff$ is used to denote logical equivalence
of statements (also ``if and only if'').  For example
\[
\forall A\subseteq[7]\quad A\cap\{1,3,6\}\neq\emptyset\iff
A\nsubseteq\{2,4,5,7\}
\]
We will use these notational shortcuts \textit{except} for
the use of $\wedge$ and $\vee$, as we will use these two symbols
in another context: binary operators in lattices. 

\section{Supplementary Notes}

Our treatment of set theory has been thoroughly intuitive \dots
an approach that it fraught with danger.  As was first discovered
more than 100 years ago, there are major conceptual hurdles
in formulating consistent systems of axioms for set theory.
And it is very easy to make statements that sound ``obvious''
but are not.

Here is one very famous example.  Let $X$ and $Y$ be
sets and consider the following two
statements:

\begin{enumerate}
\item There exists an injection $f:X\rightarrow Y$.
\item There exists a surjection $g:Y\rightarrow X$.
\end{enumerate}
If $X$ and $Y$ are finite sets, these statements are
equivalent, and it is perhaps natural to surmise that the
same is true when $X$ and $Y$ are infinite.  But that is not
the case.

A good source of additional (free) information on set theory is
the collection of Wikipedia articles.  Do a web search and
look up the following topics and people:

\begin{enumerate}
\item Zermelo Frankel set theory.
\item Axiom of Choice.
\item Peano postulates.
\item Georg Cantor, Augustus De Morgan, George Boole, Bertrand Russell
and Kurt G\"odel.
\end{enumerate}

\subsection{Decimal Representation of Real Numbers}

Every real number has a decimal expansion---although the
number of digits after the decimal point may be infinite.
A rational number $q=m/m$ from $\rats$ has an expansion
in which a certain block of digits repeats indefinitely.
For example, \[
\frac{2859}{35} = 81.6857142857142857142857142857142857142857142\dots
\]
In this case, the block $857142$ of size~$6$ is repeated forever.

Certain rational numbers have \textit{terminating} decimal expansions.
For example $385/8= 48.125$.  If we chose to do so, we could
write this instead as an infinite decimal by appending trailing $0$'s,
as a repeating block of size~$1$:
\[
\frac{385}{8} = 48.1250000000000000000000000000000000\dots
\]
On the other hand, we can also write the decimal expansion of
$385/8$ as:
\[
\frac{385}{8} = 48.12499999999999999999999999999999999\dots
\]
Here, we intend that the digit $9$, a block of size~$1$, be repeated forever.
Apart from this anomaly, the decimal expansion of real numbers is unique.

On the other hand, irrational numbers have non-repeating decimal 
expansions in which there is no block of repeating digits that
repeats forever.

You all know that $\sqrt{2}$ is irrational.  Here is the first part 
of its decimal expansion:
\[
\sqrt{2} =1.41421356237309504880168872420969807856967187537694807317667973\dots
\]
An irrational
number is said to be \textit{algebraic} if it is the root of
polynomial with integer coeffcients; else it is said to be
\textit{transcendental}.   
For example, $\sqrt{2}$ is \textit{algebraic} since it is the
root of the polynomial $x^2-2$.

Two other famous examples of irrational numbers are $\pi$ and $e$.
Here are their decimal expansions:
\[
\pi =3.14159265358979323846264338327950288419716939937510582097494459\dots
\]
and 
\[
e=2.7182818284590452353602874713526624977572470936999595749669676277\dots
\]
Both $\pi$ and $e$ are \textit{transcendental}.

\begin{example}
Amanda and Bilal, both students at a nearby university, have
been studying rational numbers that have
large blocks of repeating digits in their decimal expansions.
Amanda reports that she has found two positive integers
$m$ and $n$ with $n<500$ for which the decimal expansion
of the rational number $m/n$ has a block of 1961 digits which
repeats indefinitely.  Not to be outdone, Bilal brags that
he has found such a pair $s$ and $t$ of positive
integers with $t<300$ for which the
decimal expansion of $s/t$ has a block of $7643$ digits which
repeats indefinitely.  Bilal should be (politely) told to
do his arithmetic more carefully, as there is no such pair
of positive integers (\textit{Why}?).  On the other hand, Amanda may in fact
be correct---although, if she has done her work with more
attention to detail, she would have reported that the decimal
expansion of $m/n$ has a smaller block of repeating digits (\textit{Why}?).
\end{example}

\begin{proposition} 
There is no surjection from $\posints$ to the
set $X= \{x\in\reals:0<x<1\}$.
\end{proposition}

\begin{proof}
Let $f$ be a function from $\posints$ to $X$.
For each $n\in \posints$, consider the decimal
expanion(s) of the real number $f(n)$.  Then
choose a positive integer $a_n$ so that (1)~$a_n\le 8$, and
(2)~$a_n$ is not the $n^{th}$ digit after the decimal
point in any decimal expansion of $f(n)$.
Then the real number $x$ whose decimal expansion
is $x=.a_1a_2a_3a_4a_5\dots$ is an element of $X$
which is distinct from $f(n)$, for every $n\in\posints$.
This shows that $f$ is not a surjection.
\end{proof}

%%% Local Variables: 
%%% mode: latex
%%% TeX-master: "book"
%%% End: 
