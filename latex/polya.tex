% polya.tex
% Updated January 11, 2012

\chapter{P\'olya's Enumeration Theorem}\label{ch:polya}

In this chapter, we introduce a powerful enumeration technique
generally referred to as P\'olya's enumeration theorem\footnote{Like
  so many results of mathematics, the crux of the result was
  originally discovered by someone other than the mathematician whose
  name is associated with it. J.H.\ Redfield published this result in
  1927, 10 years prior to P\'olya's work. It would take until 1960 for
  Redfield's work to be discovered, by which time P\'olya's name was
  firmly attached to the technique.}. P\'olya's approach to counting
allows us to use symmetries (such as those of geometric objects like
polygons) to form generating functions. These generating functions can
then be used to answer combinatorial questions such as

\begin{enumerate}
\item How many different necklaces of six beads can be formed using
  red, blue and green beads? What about $500$-bead necklaces?
\item How many musical scales consisting of $6$ notes are there?
\item How many isomers of the compound xylenol,
  $\text{C}_6\text{H}_3(\text{CH}_3)_2(\text{OH})$, are there? What
  about $\text{C}_n \text{H}_{2n+2}$? (In chemistry, \emph{isomers}
  are chemical compounds with the same number of molecules of each
  element but with different arrangements of those molecules.)
\item How many nonisomorphic graphs are there on four vertices? How
  many of them have three edges? What about on $1000$ vertices with
  $257,000$ edges? How many $r$-regular graphs are there on $40$
  vertices? (A graph is $r$\emph{-regular} if every vertex has degree
  $r$.)
\end{enumerate}

To use P\'olya's techniques, we will require the idea of a permutation
group. However, our treatment will be self-contained and driven by
examples. We begin with a simplified version of the first question above.

\section{Coloring the Vertices of a Square}\label{s:polya:square}

Let's begin by coloring the vertices of a square using white and
gold. If we fix the position of the square in the plane, there are
$2^4=16$ different colorings. These colorings are shown in
\autoref{fig:polya:squares}.
\begin{figure}
  \centering
  \begin{overpic}[width=\linewidth]{polya-figs/square-colorings}
    \put(4.5,17){$C_1$}
    \put(16.5,17){$C_2$}
    \put(29.5,17){$C_3$}
    \put(42,17){$C_4$}
    \put(54.5,17){$C_5$}
    \put(67,17){$C_6$}
    \put(80,17){$C_7$}
    \put(92,17){$C_8$}

    \put(4.5,5){$C_9$}
    \put(16.3,5){$C_{10}$}
    \put(29.2,5){$C_{11}$}
    \put(41.8,5){$C_{12}$}
    \put(54.3,5){$C_{13}$}
    \put(67,5){$C_{14}$}
    \put(79.6,5){$C_{15}$}
    \put(91.6,5){$C_{16}$}

  \end{overpic}
  \caption{The $16$ colorings of the vertices of a square.}
  \label{fig:polya:squares}
\end{figure}
However, if we think of the square as a metal frame with a white bead
or a gold bead at each corner and allow the frame to be rotated and
flipped over, we realize that many of these colorings are
equivalent. For instance, if we flip coloring $C_7$ over about the
vertical line dividing the square in half, we obtain coloring
$C_9$. If we rotate coloring $C_2$ clockwise by $90^\circ$, we obtain
coloring $C_3$. In many cases, we want to consider such equivalent
colorings as a single coloring. (Recall our motivating example of
necklaces made of colored beads. It makes little sense to
differentiate between two necklaces if one can be rotated and flipped
to become the other.)

To systematically determine how many of the colorings shown in
\autoref{fig:polya:squares} are not equivalent, we must think about
the transformations we can apply to the square and what each does to
the colorings. Before examining the transformations' effects on the
colorings, let's take a moment to see how they rearrange the
vertices. To do this, we consider the upper-left vertex to be $1$, the
upper-right vertex to be $2$, the lower-right vertex to be $3$, and
the lower-left vertex to be $4$. We denote the clockwise rotation by
$90^\circ$ by $r_1$ and see that $r_1$ sends the vertex in position
$1$ to position $2$, the vertex in position $2$ to position $3$, the
vertex in position $3$ to position $4$, and the vertex in position $4$
to position $1$. For brevity, we will write $r_1(1) =2$, $r_1(2)=3$,
etc. We can also rotate the square clockwise by $180^\circ$ and denote
that rotation by $r_2$. In this case, we find that $r_2(1) = 3$,
$r_2(2)=4$, $r_2(3)= 1$, and $r_2(4)=2$. Notice that we can achieve
the transformation $r_2$ by doing $r_1$ twice in
succession. Furthermore, the clockwise rotation by $270^\circ$, $r_3$,
can be achieved by doing $r_1$ three times in
succession. (Counterclockwise rotations can be avoided by noting that
they have the same effect as a clockwise rotation, although by a
different angle.)

When it comes to flipping the square, there are four axes about which
we can flip it: vertical, horizontal, positive-slope diagonal, and
negative-slope diagonal. We denote these flips by $v$, $h$, $p$, and
$n$, respectively. Now notice that $v(1) = 2$, $v(2) = 1$, $v(3) = 4$,
and $v(4) = 3$. For the flip about the horizontal axis, we have $h(1)
= 4$, $h(2) = 3$, $h(3)=2$, and $h(4)= 1$. For $p$, we have $p(1) =
3$, $p(2) = 2$, $p(3)=1$, and $p(4)=4$. Finally, for $n$ we find $n(1)
= 1$, $n(2) = 4$, $n(3) = 3$, and $n(4)=2$. There is one more
transformation that we must mention; the transformation that does
nothing to the square is called the \emph{identity transformation},
denoted $\iota$. It has $\iota(1)=1$, $\iota(2)=2$, $\iota(3)=3$, and
$\iota(4)=4$.

Now that we've identified the eight transformations of the square,
let's make a table showing which colorings from
\autoref{fig:polya:squares} are left unchanged by the application of
each transformation. Not surprisingly, the identity transformation
leaves all of the colorings unchanged. Because $r_1$ moves the
vertices cyclically, we see that only $C_1$ and $C_{16}$ remain
unchanged when it is applied. Any coloring with more than one color
would have a vertex of one color moved to one of the other
color. Let's consider which colorings are fixed by $v$, the flip about
the vertical axis. For this to happen, the color at position $1$ must
be the same as the color at position $2$, and the color at position
$3$ must be the same as the color at position $4$. Thus, we would
expect to find $2\cdot 2 = 4$ colorings unchanged by $v$. Examining
\autoref{fig:polya:squares}, we see that these colorings are $C_1$,
$C_6$, $C_8$, and $C_{16}$. Performing a similar analysis for the
remaining five transformations leads to
\autoref{tab:polya:fixed-square}.

\begin{table}[b]
  \centering
  \begin{tabular}{c|l}
    Transformation & Fixed colorings\\\hline
    $\iota$ & All 16\\
    $r_1$ & $C_1$, $C_{16}$\\
    $r_2$ & $C_{1}$, $C_{10}$, $C_{11}$, $C_{16}$\\
    $r_3$ & $C_1$, $C_{16}$\\
    $v$ & $C_1$, $C_6$, $C_8$, $C_{16}$\\
    $h$ & $C_1$, $C_7$, $C_{9}$, $C_{16}$\\
    $p$ & $C_1$, $C_3$, $C_5$, $C_{10}$, $C_{11}$, $C_{13}$, $C_{15}$,
    $C_{16}$\\
    $n$ & $C_1$, $C_2$, $C_4$, $C_{10}$, $C_{11}$, $C_{12}$, $C_{14}$,
    $C_{16}$\\
  \end{tabular}
  \caption{Colorings fixed by transformations of the square}
  \label{tab:polya:fixed-square}
\end{table}

At this point, it's natural to ask where this is going. After all,
we're trying to count the number of \emph{nonequivalent} colorings,
and \autoref{tab:polya:fixed-square} makes no effort to group
colorings based on how a transformation changes one coloring to
another. It turns out that there is a useful connection between
counting the nonequivalent colorings and determining the number of
colorings fixed by each transformation. To develop this connection, we
first need to discuss the equivalence relation created by the action
of the transformations of the square on the set $\cgC$ of all
$2$-colorings of the square. (Refer to
\autoref{sec:numsys:equivalence} for a refresher on the definition of
equivalence relation.) To do this, notice that applying a
transformation to a square with colored vertices results in another
square with colored vertices. For instance, applying the
transformation $r_1$ to a square colored as in $C_{12}$ results in a
square colored as in $C_{13}$. We say that the transformations of the
square \emph{act} on the set $\cgC$ of colorings. We denote this
action by adding a star to the transformation name. For instance,
$r_1^*(C_{12})=C_{13}$ and $v^*(C_{10})=C_{11}$.

If $\tau$ is a transformation of the square with $\tau^*(C_i) = C_j$,
then we say colorings $C_i$ and $C_j$ are \emph{equivalent} and write
$C_i\sim C_j$. Since $\iota^*(C)=C$ for all $C\in\cgC$, $\sim$ is
reflexive. If $\tau_1^*(C_i) = C_j$ and $\tau_2^*(C_j) = C_k$, then
$\tau_2^*(\tau_1^*(C_i)) = C_k$, so $\sim$ is transitive. To complete
our verification that $\sim$ is an equivalence relation, we must
establish that it is symmetric. For this, we require the notion of the
\emph{inverse} of a transformation $\tau$, which is simply the
transformation $\tau\inv$ that undoes whatever $\tau$ did. For
instance, the inverse of $r_1$ is the \emph{counter}clockwise rotation
by $90^\circ$, which has the same effect on the location of the
vertices as $r_3$. If $\tau^*(C_i) = C_j$, then ${\tau^{-1}}^*(C_j) =
C_i$, so $\sim$ is symmetric. 

Before proceeding to establish the connection between the number of
nonequivalent colorings (equivalence classes under $\sim$) and the
number of colorings fixed by a transformation in full generality,
let's see how it looks for our example. In looking at
\autoref{fig:polya:squares}, you should notice that $\sim$ partitions
$\cgC$ into six equivalence classes. Two contain one coloring each
(the all white and all gold colorings). One contains two colorings
($C_{10}$ and $C_{11}$). Finally, three contain four colorings each
(one gold vertex, one white vertex, and the remaining four with two
vertices of each color). Now look again at
\autoref{tab:polya:fixed-square} and add up the number of colorings
fixed by each transformation. In doing this, we obtain $48$, and when
$48$ is divided by the number of transformations ($8$), we get $6$
(the number of equivalence classes)! It turns out that this is far
from a fluke, as we will soon see. First, however, we introduce the
concept of a permutation group to generalize our set of
transformations of the square.

\section{Permutation Groups}\label{s:polya:perm-groups}

Entire books have been written on the theory of the mathematical
structures known as \emph{groups}. However, our study of P\'olya's
enumeration theorem requires only a few facts about a particular class
of groups that we introduce in this section. First, recall that a
bijection from a set $X$ to itself is called a \emph{permutation}. A
\emph{permutation group} is a set $P$ of permutations of a set $X$ so
that
\begin{enumerate}
\item the identity permutation $\iota$ is in $P$;
\item if $\pi_1,\pi_2\in P$, then $\pi_2\circ \pi_1\in P$; and
\item if $\pi_1\in P$, then $\pi_1\inv\in P$.
\end{enumerate}
For our purposes, $X$ will always be finite and we will usually take
$X=[n]$ for some positive integer $n$. The \emph{symmetric group on
  $n$ elements}, denoted $S_n$, is the set of all permutations of
$[n]$. Every finite permutation group (and more generally every finite
group) is a subgroup of $S_n$ for some positive integer $n$.

As our first example of a permutation group, consider the set of
permutations we discussed in \autoref{s:polya:square}, called the
\emph{dihedral group of the square}. We will denote this group by
$D_8$. We denote by $D_{2n}$ the similar group of transformations for
a regular $n$-gon, using $2n$ as the subscript because there are $2n$
permutations in this group.\footnote{Some authors and computer algebra
  systems use $D_n$ as the notation for the dihedral group of the
  $n$-gon.} The first criterion to be a permutation group is clearly
satisfied by $D_8$. Verifying the other two is quite tedious, so we
only present a couple of examples. First, notice that $r_2\circ
r_1=r_3$. This can be determined by carrying out the composition of
these functions as permutations or by noting that rotating $90^\circ$
clockwise and then $180^\circ$ clockwise is the same as rotating
$270^\circ$ clockwise. For $v\circ r$, we find $v\circ r(1) = 1$,
$v\circ r(3)=3$, $v\circ r(2)=4$, and $v\circ r(4)=2$, so $v\circ
r=n$. For inverses, we have already discussed that $r_1\inv =
r_3$. Also, $v\inv = v$, and more generally, the inverse of \emph{any}
flip is that same flip.

\subsection{Representing permutations}\label{ss:polya:perm-groups:rep}

The way a permutation rearranges the elements of $X$ is central to
P\'olya's enumeration theorem. A proper choice of representation for a
permutation is very important here, so let's discuss how permutations
can be represented. One way to represent a permutation $\pi$ of $[n]$
is as a $2\times n$ matrix in which the first row represents the
domain and the second row represents $\pi$ by putting $\pi(i)$ in
position $i$. For example,
\[\pi=
\begin{pmatrix}
  1 & 2 & 3 & 4 & 5\\
  2 & 4 & 3 & 5 & 1
\end{pmatrix}
\]
is the permutation of $[5]$ with $\pi(1) =2$, $\pi(2)=4$, $\pi(3)=3$,
$\pi(4) = 5$, and $\pi(5) = 1$. This notation is rather awkward and
provides only the most basic information about the permutation. A more
compact (and more useful for our purposes) notation is known as
\emph{cycle notation}. One way to visualize how the cycle notation is
constructed is by constructing a digraph from a permutation $\pi$ of
$[n]$. The digraph has $[n]$ as its vertex set and a directed edge
from $i$ to $j$ if and only if $\pi(i) = j$. (Here we allow a directed
edge from a vertex to itself if $\pi(i) = i$.) The digraph
corresponding to the permutation $\pi$ from above is shown in
\autoref{fig:polya:perm-digraph}.
\begin{figure}[b]
  \centering
  \begin{overpic}[scale=0.9]{polya-figs/permutation-cycles}
    \put(0,50){$1$}
    \put(-1,5){$2$}
    \put(87,6){$3$}
    \put(60,5){$4$}
    \put(60,50){$5$}
  \end{overpic}
  \caption{The digraph corresponding to permutation $\pi=(1245)(3)$}
  \label{fig:polya:perm-digraph}
\end{figure}
Since $\pi$ is a permutation, every component of such a digraph is a
directed cycle. We can then use these cycles to write down the
permutation in a compact manner. For each cycle, we start at the
vertex with smallest label and go around the cycle in the direction of
the edges, writing down the vertices' labels in order. We place this
sequence of integers in parentheses. For the $4$-cycle in
\autoref{fig:polya:perm-digraph}, we thus obtain $(1245)$. (If $n\geq
10$, we place spaces or commas between the integers.) The component
with a single vertex is denoted simply as $(3)$, and thus we may write
$\pi=(1245)(3)$. By convention, the disjoint cycles of a permutation
are listed so that their first entries are in increasing order.

\begin{example}
  The permutation $\pi=(1483)(27)(56)$ has $\pi(1)=4$,
  $\pi(8)=3$, $\pi(3)=1$, and $\pi(5)=6$. The permutation
  $\pi'=(13)(2)(478)(56)$ has $\pi'(1)=3$, $\pi'(2) = 2$, and
  $\pi'(8)=4$. We say that $\pi$ consists of two cycles of length $2$
  and one cycle of length $4$. For $\pi'$, we have one cycle of length
  $1$, two cycles of length $2$, and one cycle of length $3$. A cycle
  of length $k$ will also called a $k$-cycle in this chapter.
\end{example}

\subsection{Multiplying permutations}\label{ss:polya:perm-groups:mult}

Because the operation in an arbitrary group is frequently called
multiplication, it is common to refer to the composition of
permutations as multiplication and write $\pi_2\pi_1$ instead of
$\pi_2\circ \pi_1$. The important thing to remember here, however, is
that the operation is simply function composition. Let's see a couple
of examples.

\begin{example}
  Let $\pi_1 = (1234)$ and $\pi_2 = (12)(34)$. (Notice that these are
  the permutations $r_1$ and $v$, respectively, from $D_8$.) Let
  $\pi_3=\pi_2\pi_1$. To determine $\pi_3$, we start by finding
  $\pi_3(1) = \pi_2\pi_1(1) = \pi_2(2) = 1$. We next find that
  $\pi_3(2) = \pi_2\pi_1(2) = \pi_2(3)=4$. Similarly, $\pi_3(3) = 3$
  and $\pi_3(4)=2$. Thus, $\pi_3=(1)(24)(3)$, which we called $n$
  earlier.

  Now let $\pi_4 = \pi_1\pi_2$. Then $\pi_4(1) = 3$, $\pi_4(2)=2$,
  $\pi_4(3)=1$, and $\pi_4(4)=4$. Therefore, $\pi_4=(13)(2)(4)$, which
  we called $p$ earlier. It's important to note that $\pi_1\pi_2\neq
  \pi_2\pi_1$, which hopefully does not surprise you, since function
  composition is not in general commutative. To further illustrate the
  lack of commutativity in permutation groups, pick up a book (Not
  this one! You need to keep reading directions here.) so that
  cover is up and the spine is to the left. First, flip the book over
  from left to right. Then rotate it $90^\circ$
  clockwise. Where is the spine? Now return the book to the cover-up,
  spine-left position. Rotate the book $90^\circ$ clockwise and then
  flip it over from left to right. Where is the spine this time?
  \end{example}

It quickly gets tedious to write down where the product of two (or
more) permutations sends each element. A more efficient approach would
be to draw the digraph and then write down the cycle structure. With
some practice, however, you can build the cycle notation as you go
along, as we demonstrate in the following example.

\begin{example}
  Let $\pi_1=(123)(487)(5)(6)$ and $\pi_2=(18765)(234)$. Let $\pi_3 =
  \pi_2\pi_1$. To start constructing the cycle notation for $\pi_3$,
  we must determine where $\pi_3$ sends $1$. We find that it sends it
  to $3$, since $\pi_1$ sends $1$ to $2$ and $\pi_2$ sends $2$ to
  $3$. Thus, the first cycle begins $13$. Now where is $3$ sent?  It's
  sent to $8$, which goes to $6$, which goes to $5$, which goes to
  $1$, completing our first cycle as $(13865)$. The first integer not
  in this cycle is $2$, which we use to start our next cycle. We find
  that $2$ is sent to $4$, which is set to $7$, which is set to
  $2$. Thus, the second cycle is $(247)$. Now all elements of $8$ are
  represented in these cycles, so we know that $\pi_3 = (13865)(247)$.
\end{example}

We conclude this section with one more example.

\begin{example}
  Let's find $[(123456)][(165432)]$, where we've written the two
  permutations being multiplied inside brackets. Since we work from
  \emph{right} to \emph{left}, we find that the first permutation
  applied sends $1$ to $6$, and the second sends $6$ to $1$, so our
  first cycle is $(1)$. Next, we find that the product sends $2$ to
  $2$. It also sends $i$ to $i$ for every other $i\leq 6$. Thus, the
  product is $(1)(2)(3)(4)(5)(6)$, which is better known as the
  identity permutation. Thus, $(123456)$ and $(165432)$ are inverses.
\end{example}

In the next section, we will use standard counting techniques we've
seen before in this book to prove results about groups acting ons
ets. We will state the results for arbitrary groups, but you may
safely replace ``group'' by ``permutation group'' without losing any
understanding required for the remainder of the chapter.

\section{Burnside's Lemma}\label{s:polya:burnside}

Burnside's lemma\footnote{Again, not originally proved by Burnside. It
  was known to Frobenius and for the most part by Cauchy. However, it
  was most easily found in Burnside's book, and thus his name came to
  be attached.} relates the number of equivalence classes of the
action of a group on a finite set to the number of elements of the set
fixed by the elements of the group. Before stating and proving it, we
need some notation and a proposition. If a group $G$ acts on a finite
set $\cgC$, let $\sim$ be the equivalence relation induced by this
action. (As before, the action of $\pi\in G$ on $\cgC$ will be denoted
$\pi^*$.) Denote the equivalence class containing $C\in \cgC$ by
$\langle C\rangle$. For $\pi\in G$, let $\fix_\cgC(\pi)=\{C\in
\cgC\colon \pi^*(C) = C\}$, the set of colorings fixed by $\pi$. For
$C\in\cgC$, let $\stab_G(C)=\{\pi\in G\colon \pi(C) = C\}$ be the
\emph{stabilizer} of $C$ in $G$, the permutations in $G$ that fix $C$.

To illustrate these concepts before applying them, refer back to
\autoref{tab:polya:fixed-square}. Using that information, we can
determine that $\fix_\cgC(r_2) =
\{C_1,C_{10},C_{11},C_{16}\}$. Determining the stabilizer of a
coloring requires finding the rows of the table in which it
appears. Thus, $\stab_{D_8}(C_7) = \{\iota,h\}$ and
$\stab_{D_8}(C_{11}) = \{\iota,r_2,p,n\}$.

\begin{proposition}\label{prop:polya:sum-stabs}
  Let a group $G$ act on a finite set $\cgC$. Then for all $C\in \cgC$,
  \[\sum_{C'\in\langle C\rangle} |\stab_G(C')| = |G|.\]
\end{proposition}

\begin{proof}
  Let $\stab_G(C) = \{\pi_1,\dots,\pi_k\}$ and $T(C,C') = \{\pi\in
  G\colon \pi^*(C) = C'\}$. (Note that $T(C,C) = \stab_G(C)$.) Take
  $\pi\in T(C,C')$. Then $\pi\circ \pi_i\in T(C,C')$ for $1\leq i\leq
  k$. Furthermore, if $\pi\circ \pi_i = \pi\circ \pi_j$, then
  $\pi^{-1}\circ\pi\circ \pi_i=\pi^{-1}\circ\pi\circ \pi_j$. Thus
  $\pi_i=\pi_j$ and $i=j$. If $\pi'\in T(C,C')$, then $\pi\inv\circ
  \pi'\in T(C,C)$. Thus, $\pi\inv\circ\pi' = \pi_i$ for some $i$, and
  hence $\pi' = \pi\circ \pi_i$. Therefore $T(C,C') =
  \{\pi\circ\pi_1,\dots,\pi\circ\pi_k\}$. Additionally, we observe
  that $T(C',C) = \{\pi\inv\colon \pi\in T(C,C')\}$. Now for all
  $C'\in\langle C\rangle$, \[|\stab_G(C')|=|T(C',C')|=|T(C',C)| =
  |T(C,C')| = |T(C,C)| = |\stab_G(C)|.\]
  Therefore, \[\sum_{C'\in\langle C\rangle}|\stab_G(C')| =
  \sum_{C'\in\langle C\rangle} |T(C,C')|.\] Now notice that each
  element of $G$ appears in $T(C,C')$ for precisely one $C'\in\langle
  C\rangle$, and the proposition follows.
\end{proof}

With \hyperref[prop:polya:sum-stabs]{Proposition
  \ref*{prop:polya:sum-stabs}} established, we are now prepared for
Burnside's lemma.

\begin{lemma}[Burnside's Lemma]\label{lem:polya:burnside}
  Let a group $G$ act on a finite set $\cgC$ and let $N$ be the
  number of equivalence classes of $\cgC$ induced by this
  action. Then
  \[N = \frac{1}{|G|} \sum_{\pi\in G} |\fix_\cgC(\pi)|.\]
\end{lemma}

Before we proceed to the proof, note that the calculation in
Burnside's lemma for the example of $2$-coloring the vertices of a
square is exactly the calculation we performed at the end of \autoref{s:polya:square}.

\begin{proof}
  Let $X=\{(\pi,C)\in G\times \cgC\colon \pi(C) = C\}$. Notice that
  $\sum_{\pi \in G} |\fix_\cgC(\pi)| = |X|$, since each term in the
  sum counts how many ordered pairs of $X$ have $\pi$ in their first
  coordinate. Similarly, $\sum_{C\in \cgC} |\stab_G(C)| = |X|$, with
  each term of this sum counting how many ordered pairs of $X$ have
  $C$ as their second coordinate. Thus, $\sum_{\pi \in G}
  |\fix_\cgC(\pi)|=\sum_{C\in \cgC} |\stab_G(C)|$. Now note that the
  latter sum may be rewritten
  as \[\sum_{\substack{\text{equivalence}\\\text{classes } \langle
      C\rangle}}\left( \sum_{C'\in\langle C\rangle}
    |\stab_G(C')|\right).\] By
  \hyperref[prop:polya:sum-stabs]{Proposition~\ref*{prop:polya:sum-stabs}},
  the inner sum is $|G|$. Therefore, the total sum is $N\cdot |G|$, so
  solving for $N$ gives the desired equation.
\end{proof}

\hyperref[lem:polya:burnside]{Burnside's lemma} helpfully validates
the computations we did in the previous section. However, what if
instead of a square we were working with a hexagon and instead of two
colors we allowed four? Then there would be $4^6=4096$ different
colorings and the dihedral group of the hexagon has $12$
elements. Assembling the analogue of \autoref{tab:polya:fixed-square}
in this situation would be a nightmare! This is where the genius of
P\'olya's approach comes into play, as we see in the next section.

\section{P\'olya's Theorem}\label{s:polya:polya}

Before getting to the full version of P\'olya's formula, we must
develop a generating function as promised at the beginning of the
chapter. To do this, we will return to our example of
\autoref{s:polya:square}.

\subsection{The cycle index}\label{ss:polya:polya:cycle-index}

Unlike the generating functions we encountered in
\autoref{ch:genfunction}, the generating functions we will develop in
this chapter will have more than one variable. We begin by associating
a monomial with each element of the permutation group involved. In
this case, it is $D_8$, the dihedral group of the square. To determine
the monomial associated to a permutation, we need to write the
permutation in cycle notation and then determine the monomial based on
the number of cycles of each length. Specifically, if $\pi$ is a
permutation of $[n]$ with $j_k$ cycles of length $k$ for $1\leq k\leq
n$, then the monomial associated to $\pi$ is $x_1^{j_1}x_2^{j_2}\cdots
x_n^{j_n}$. Note that $j_1 + 2j_2 + 3j_3 + \cdots + nj_n = n$. For
example, the permutation $r_1=(1234)$ is associated with the monomial
$x_4^1$ since it consists of a single cycle of length $4$. The
permutation $r_2=(13)(24)$ has two cycles of length $2$, and thus its
monomial is $x_2^2$. For $p=(14)(2)(3)$, we have two $1$-cycles and
one $2$-cycle, yielding the monomial $x_1^2x_2^1$. In
\autoref{tab:polya:square-cycles}, we show all eight permutations in
$D_8$ along with their associated monomials.

\begin{table}
  \centering
  \begin{tabular}{c|c|c}
    Transformation & Monomial & Fixed colorings\\\hline
    $\iota = (1)(2)(3)(4)$  & $x_1^4$& $16$\\[\smallskipamount]
    $r_1 = (1234)$ & $x_4^1$ & $2$\\[\smallskipamount]
    $r_2=(13)(24)$ & $x_2^2$ & $4$\\[\smallskipamount]
    $r_3=(1432)$ & $x_4^1$ & $2$\\[\smallskipamount]
    $v=(12)(34)$ & $x_2^2$ & $4$\\[\smallskipamount]
    $h=(14)(23)$ & $x_2^2$ & $4$\\[\smallskipamount]
    $p=(14)(2)(3)$ & $x_1^2x_2^1$ & $8$\\[\smallskipamount]
    $n=(1)(24)(3)$ & $x_1^2x_2^1$ & $8$
  \end{tabular}
  \caption{Monomials arising from the dihedral group of the square}
  \label{tab:polya:square-cycles}
\end{table}

Now let's see how the number of $2$-colorings of the square fixed by a
permutation can be determined from its cycle structure and associated
monomial. If $\pi(i)=j$, then we know that for $\pi$ to fix a coloring
$C$, vertices $i$ and $j$ must be colored the same in $C$. Thus, the
second vertex in a cycle must have the same color as the first. But
then the third vertex must have the same color as the second, which is
the same color as the first. In fact, all vertices appearing in a
cycle of $\pi$ must have the same color in $C$ if $\pi$ fixes $C$!
Since we are coloring with the two colors white and gold, we can
choose to color the points of each cycle uniformly white or gold. For
example, for the permutation $v=(12)(34)$ to fix a coloring of the
square, vertices $1$ and $2$ must be colored the same color ($2$
choices) and vertices $3$ and $4$ must be colored the same color ($2$
choices). Thus, there are $2\cdot 2=4$ colorings fixed by $v$. Since
there are two choices for how to uniformly color the elements of a
cycle, letting $x_i=2$ for all $i$ in the monomial associated with
$\pi$ gives the number of colorings fixed by $\pi$. In
\autoref{tab:polya:square-cycles}, the ``Fixed colorings'' column
gives the number of $2$-colorings of the square fixed by each
permutation. Before, we obtained this manually by considering the
action of $D_8$ on the set of all $16$ colorings. Now we only need the
cycle notation and the monomials that result from it to derive this!

Recall that \hyperref[lem:polya:burnside]{Burnside's lemma
  (\ref{lem:polya:burnside})} states that the number of colorings
fixed by the action of a group can be obtained by adding up the number
fixed by each permutation and dividing by the number of permutations
in the group. If we do that instead for the monomials arising from the
permutations in a permutation group $G$ in which every cycle of every
permutation has at most $n$ entries, we obtain a polynomial known as
the \emph{cycle index} $P_G(x_1,x_2,\dots,x_n)$. For our running
example, we find
\[P_{D_8}(x_1,x_2,x_3,x_4) = \frac{1}{8}\left(x_1^4 + 2x_1^2x_2^1 +
  3x_2^2 + 2x_4^1\right).\] To find the number of distinct
$2$-colorings of the square, we thus let $x_i=2$ for all $i$ and
obtain $P_{D_8}(2,2,2,2) = 6$ as before. Notice, however, that we have
something more powerful than \hyperref[lem:polya:burnside]{Burnside's
  lemma} here. We may substitute \emph{any} positive integer $m$ for
each $x_i$ to find out how many nonequivalent $m$-colorings of the
square exist. We no longer have to analyze how many colorings each
permutation fixes. For instance, $P_{D_8}(3,3,3,3) = 21$, meaning that
$21$ of the $81$ colorings of the vertices of the square using three
colors are distinct.

\subsection{The full enumeration formula}\label{ss:polya:polya:full}

Hopefully the power of the cycle index to count colorings that are
distinct when symmetries are considered is becoming apparent. In the
next section, we will provide additional examples of how it can be
used. However, we still haven't seen the full power of P\'olya's
technique. From the cycle index alone, we can determine how many
colorings of the vertices of the square are distinct. However, what if
we want to know how many of them have two white vertices and two gold
vertices? This is where P\'olya's enumeration formula truly plays the
role of a generating function.

Let's again consider the cycle index for the dihedral group $D_8$:
\[P_{D_8}(x_1,x_2,x_3,x_4) = \frac{1}{8}\left(x_1^4 + 2x_1^2x_2^1 +
  3x_2^2 + 2x_4^1\right).\]
Instead of substituting integers for the $x_i$, let's consider what
happens if we substitute something that allows us to track the colors
used. Since $x_1$ represents a cycle of length $1$ in a permutation,
the choice of white or gold for the vertex in such a cycle amounts to
a single vertex receiving that color. What happens if we substitute
$w+g$ for $x_1$? The first term in $P_{D_8}$ corresponds to the
identity permutation $\iota$, which fixes all colorings of the
square. Letting $x_1=w+g$ in this term gives
\[(w+g)^4 = g^4+4 g^3 w+6 g^2 w^2+4 g w^3+w^4,\]
which tells us that $\iota$ fixes one coloring with four gold
vertices, four colorings with three gold vertices and one white
vertex, six colorings with two gold vertices and two white vertices,
four colorings with one gold vertex and three white vertices, and one
coloring with four white vertices.

Let's continue establishing a pattern here by considering the variable
$x_2$. It represents the cycles of length $2$ in a permutation. Such a
cycle must be colored uniformly white or gold to be fixed by the
permutation. Thus, choosing white or gold for the vertices in that
cycle results in two white vertices or two gold vertices in the
coloring. Since this happens for every cycle of length $2$, we want to
substitute $w^2+g^2$ for $x_2$ in the cycle index. The $x_1^2x_2^1$
terms in $P_{D_8}$ are associated with the flips $p$ and $n$. Letting
$x_1=w+g$ and $x_2 = w^2+g^2$, we find
\[x_1^2x_2^1 = g^4+2 g^3 w+2 g^2 w^2+2 g w^3+w^4,\]
from which we are able to deduce that $p$ and $n$ each fix one
coloring with four gold vertices, two colorings with three gold
vertices and one white vertex, and so on. Comparing this with
\autoref{tab:polya:fixed-square} shows that the generating function is
right on.

By now the pattern is becoming apparent. If we substitute $w^i+g^i$
for $x_i$ in the cycle index for each $i$, we then keep track of how
many vertices are colored white and how many are colored gold. The
simplification of the cycle index in this case is then a generating
function in which the coefficient on $g^s w^t$ is the number of
distinct colorings of the vertices of the square with $s$ vertices
colored gold and $t$ vertices colored white. Doing this and
simplifying gives
\[P_{D_8}(w+g,w^2+g^2,w^3+g^3,w^4+g^4) = g^4+g^3 w+2 g^2 w^2+g
w^3+w^4.\]
From this we find one coloring with all vertices gold,
one coloring with all vertices white, one coloring with three gold
vertices and one white vertex, one coloring with one gold vertex and
three white vertices, and two colorings with two vertices of each
color.

As with the other results we've discovered in this chapter, this
property of the cycle index holds up beyond the case of coloring the
vertices of the square with two colors. The full version is P\'olya's
enumeration theorem:

\begin{theorem}[P\'olya's Enumeration Theorem]\label{theorem:polya:polya}
  Let $S$ be a set with $|S|=r$ and $\cgC$ the set of colorings of $S$
  using the colors $c_1,\dots,c_m$. Let a permutation group $G$ act on
  $S$ to induce an equivalence relation on $\cgC$. Then
  \[P_G\left(\sum_{i=1}^m c_i, \sum_{i=1}^m c_i^2, \dots,\sum_{i=1}^m
    c_i^r\right) \]
  is the generating function for the number of nonequivalent colorings
  of $S$ in $\cgC$.
\end{theorem}

If we return to coloring the vertices of the square but now allow the
color blue as well, we find
\begin{multline*}P_{D_8}(w+g+b,w^2+g^2+b^2,w^3+g^3+b^3,w^4+g^4+b^4) = b^4+b^3 g+2 b^2 g^2+b g^3+g^4\\+b^3 w+2 b^2 g w+2 b g^2 w+g^3 w+2 b^2 w^2+2 b g w^2+2 g^2 w^2+b w^3+g w^3+w^4.\end{multline*}
From this generating function, we can readily determine the number of
nonequivalent colorings with two blue vertices, one gold vertex, and
one white vertex to be $2$. Because the generating function of
\hyperref[theorem:polya:polya]{P\'olya's enumeration theorem} records
the number of nonequivalent patterns, it is sometimes called the
\emph{pattern inventory}. 

What if we were interested in making necklaces with $500$ (very small)
beads colored white, gold, and blue? This would be equivalent to
coloring the vertices of a regular $500$-gon, and the dihedral group
$D_{1000}$ would give the appropriate transformations. With a computer
algebra system\footnote{With some more experience in group theory, it
  is possible to give a general formula for the cycle index of the
  dihedral group $D_{2n}$, so the computer algebra system is a nice
  tool, but not required.}  such as
\emph{Mathematica}$^\text{\tiny\textregistered}$, it is possible to quickly
produce the pattern inventory for such a problem. In doing so, we find
that there are \begin{align*}
  &3636029179586993684238526707954331911802338502600162304034603583258060\\
  &0191583895484198508262979388783308179702534404046627287796430425271499\\
  &2703135653472347417085467453334179308247819807028526921872536424412922\\
  &79756575936040804567103229 \approx 3.6\times 10^{235}\end{align*}
possible necklaces. Of them,
\begin{align*}
  &2529491842340460773490413186201010487791417294078808662803638965678244\\
  &7138833704326875393229442323085905838200071479575905731776660508802696\\
  &8640797415175535033372572682057214340157297357996345021733060\approx
  2.5\times 10^{200}
\end{align*}
have $225$ white beads, $225$ gold beads, and $50$ blue beads.

The remainder of this chapter will focus on applications of P\'olya's
enumeration theorem and the pattern inventory in a variety of
settings.

\section{Applications of P\'olya's Enumeration
  Formula}\label{s:polya:apps}

This section explores a number of situations in which P\'olya's
enumeration formula can be used. The applications are from a variety
of domains and are arranged in increasing order of complexity,
beginning with an example from music theory and concluding with
counting nonisomorphic graphs.

\subsection{Counting musical scales}\label{ss:polya:scales}

Western music is generally based on a system of $12$ equally-spaced
\emph{notes}. Although these notes are usually named by letters of the
alphabet (with modifiers), for our purposes it will suffice to number
them as $0,1,\dots,11$. These notes are arranged into \emph{octaves}
so that the next pitch after $11$ is again named $0$ and the pitch
before $0$ is named $11$. For this reason, we may consider the system
of notes to correspond to the integers modulo $12$. With these
definitions, a \emph{scale} is a subset of $\{0,1,\dots,11\}$ arranged
in increasing order. A \emph{transposition} of a scale is a uniform
transformation that replaces each note $x$ of the scale by $x+a\pmod
12$ for some constant $a$. Musicians consider two scales to be
equivalent if one is a transposition of the other. Since a scale is a
subset, no regard is paid to which note starts the scale, either. The
question we investigate in this section is ``How many nonequivalent
scales are there consisting of precisely $k$ notes?''

Because of the cyclic nature of the note names, we may consider
arranging them in order clockwise around a circle. Selecting the notes
for a scale then becomes a coloring problem if we say that selected
notes are colored black and unselected notes are colored white. In
\autoref{fig:polya:scales}, we show three $5$-note scales using this
convention. Notice that since $S_2$ can be obtained from $S_1$ by
rotating it forward seven positions, $S_1$ and $S_2$ are equivalent by
the transposition of adding $7$. However, $S_3$ is not equivalent to
$S_1$ or $S_2$, as it cannot be obtained from them by rotation. (Note
that $S_3$ could be obtained from $S_1$ if we allowed flips in
addition to rotations. Since the only operation allowed is the
transposition, which corresponds to rotation, they are inequivalent.)
\begin{figure}[t]
  \centering
  \begin{overpic}[width=\linewidth]{polya-figs/scales}
    \put(14,13){$S_1$}
    \put(49.5,13){$S_2$}
    \put(84,13){$S_3$}
  \end{overpic}
  \caption{Three scales depicted by coloring}
  \label{fig:polya:scales}
\end{figure}

We have now mathematically modeled musical scales as discrete
structures in a way that we can use P\'olya's enumeration
theorem. What is the group acting on our black/white colorings of the
vertices of a regular $12$-gon? One permutation in the group is $\tau
= (0\ 1\ 2\ 3\ 4\ 5\ 6\ 7\ 8\ 9\ 10\ 11)$, which corresponds to the
transposition by one note. In fact, every element of the group can be
realized as some power of $\tau$ since only rotations are allowed and
$\tau$ is the smallest possible rotation. Thus, the group acting on
the colorings is the \emph{cyclic group of order $12$}, denoted
$C_{12} =
\{\iota,\tau,\tau^2,\dots,\tau^{11}\}$. \hyperref[ex:polya:cyclic12]{Exercise~\ref*{ex:polya:cyclic12}}
asks you to write all the elements of this group in cycle
notation. The best way to do this is by multiplying $\tau^{i-1}$ by
$\tau$ (i.e., compute $\tau\tau^{i-1}$) to find $\tau$. Once you've
done this, you will be able to easily verify that the cycle index is
\[P_{C_{12}}(x_1,\dots,x_{12}) = \frac{x_1^{12}}{12}+\frac{x_2^6}{12}+\frac{x_3^4}{6}+\frac{x_4^3}{6}+\frac{x_6^2}{6}+\frac{x_{12}}{3}.\]

Since we've chosen colorings using black and white, it would make
sense to substitute $x_i = b^i +w^i$ for all $i$ in $P_{C_{12}}$ now
to find the number of $k$-note scales. However, there is a convenient
shortcut we may take to make the resulting generating function look
more like those to which we grew accustomed in
\autoref{ch:genfunction}. The information about how many notes are
\emph{not} included in our scale (the number colored white) can be
deduced from the number that are included. Thus, we may eliminate the
use of the variable $w$, replacing it by $1$. We now find
\begin{multline*}P_{C_{12}}(1+b,1+b^2,\dots,1+b^{12}) = b^{12}+b^{11}+6 b^{10}+19 b^9+43 b^8+66 b^7+80 b^6\\+66 b^5+43 b^4+19 b^3+6 b^2+b+1.\end{multline*}
From this, we are able to deduce that the number of scales with $k$
notes is the coefficient on $b^k$. Therefore, the answer to our
question at the beginning of the chapter about the number of $6$-note
scales is $80$.

\subsection{Enumerating isomers}\label{ss:polya:isomers}

Benzene is a chemical compound with formula $\text{C}_6\text{H}_6$,
meaning it consists of six carbon atoms and six hydrogen atoms. These
atoms are bonded in such a way that the six carbon atoms form a
hexagonal ring with alternating single and double bonds. A hydrogen
atom is bonded to each carbon atom (on the outside of the ring). From
benzene it is possible to form other chemical compounds that are part
of a family known as \emph{aromatic hydrocarbons}. These compounds are
formed by replacing one or more of the hydrogen atoms by atoms of
other elements or functional groups such as $\text{CH}_3$ (methyl
group) or $\text{OH}$ (hydroxyl group). Because there are six choices
for which hydrogen atoms to replace, molecules with the same chemical
formula but different structures can be formed in this manner. Such
molecules are called \emph{isomers}. In this subsection, we will see
how P\'olya's enumeration theorem can be used to determine the number
of isomers of the aromatic hydrocarbon xylenol (also known as
dimethylphenol).

Before we get into the molecular structure of xylenol, we need to
discuss the permutation group that will act on a benzene ring. Much
like with our example of coloring the vertices of the square, we find
that there are rotations and flips at play here. In fact, the group we
require is the dihedral group of the hexagon, $D_{12}$. If we number
the six carbon atoms in clockwise order as $1,2,\dots,6$, then we find
that the clockwise rotation by $60^\circ$ corresponds to the
permutation $r=(123456)$. The other rotations are the higher powers of
$r$, as shown in \autoref{tab:polya:benzene}. The flip across the
vertical axis is the permutation $f=(16)(25)(34)$. The remaining
elements of $D_{12}$ (other than the identity $\iota$) can all be
realized as some rotation followed by this flip. The full list of
permutations is shown in \autoref{tab:polya:benzene}, where each
permutation is accompanied by the monomial it contributes to the cycle
index.
\begin{table}
  \centering
  \begin{tabular}{c|c|c|c}
    Permutation & Monomial & Permutation & Monomial\\\hline
    $\iota =(1)(2)(3)(4)(5)(6)$ & $x_1^6$ & $f=(16)(25)(34)$ & $x_2^3$\\[\smallskipamount]
    $r=(123456)$ & $x_6^1$  & $fr=(15)(24)(3)(6)$ & $x_1^2x_2^2$\\[\smallskipamount]
    $r^2=(135)(246)$ & $x_3^2$  & $fr^2=(14)(23)(56)$ & $x_2^3$\\[\smallskipamount]
    $r^3=(14)(25)(36)$ & $x_2^3$  & $fr^3=(13)(2)(46)(5)$ & $x_1^2x_2^2$\\[\smallskipamount]
    $r^4=(153)(264)$ & $x_3^2$  & $fr^4=(12)(36)(45)$ &$x_2^3$\\[\smallskipamount]
    $r^5=(165432)$ & $x_6^1$  & $fr^5=(1)(26)(35)(4)$ & $x_1^2x_2^2$
  \end{tabular}
  \caption{Cycle representation of permutations in $D_{12}$}
  \label{tab:polya:benzene}
\end{table}

With the monomials associated to the permutations in $D_{12}$ identified,
we are able to write down the cycle index
\[P_{D_{12}}(x_1,\dots,x_6) = \frac{1}{12}(x_1^6 + 2x_6^1 + 2x_3^2+4x_2^3 + 3x_1^2x_2^2).\]
With the cycle index determined, we now turn our attention to using it
to find the number of isomers of xylenol. This aromatic hydrocarbon
has three hydrogen molecules, two methyl groups, and a hydroxyl group
attached to the carbon atoms. Recalling that hydrogen atoms are the
default from benzene, we can more or less ignore them when choosing
the appropriate substitution for the $x_i$ in the cycle index. If we
let $m$ denote methyl groups and $h$ hydroxyl groups, we can then
substitute $x_i = 1+m^i+h^i$ in $P_{D_{12}}$. This substitution gives the
generating function
\begin{multline*}1+h+3 h^2+3 h^3+3 h^4+h^5+h^6+m+3 h m+6 h^2 m+6 h^3
  m\\+3 h^4 m+h^5 m+3 m^2+6 h m^2+11 h^2 m^2+6 h^3 m^2+3 h^4 m^2+3
  m^3+6 h m^3\\+6 h^2 m^3+3 h^3 m^3+3 m^4+3 h m^4+3 h^2 m^4+m^5+h
  m^5+m^6.\end{multline*}
Since xylenol has one hydroxyl group and two methyl groups, we are
looking for the coefficient on $hm^2$ in this generating function. The
coefficient is $6$, so there are six isomers of xylenol.

In his original paper, P\'olya used his techniques to enumerate the
number of isomers of the alkanes $\text{C}_n\text{H}_{2n+2}$. When
modeled as graphs, these chemical compounds are special types of
trees. Since that time, P\'olya's enumeration theorem has been used to
enumerate isomers for many different chemical compounds.

\subsection{Counting nonisomorphic graphs}\label{ss:polya:graphs}

Counting the graphs with vertex set $[n]$ is not difficult. There are
$C(n,2)$ possible edges, each of which can be included or
excluded. Thus, there are $2^{C(n,2)}$ \emph{labeled} graphs on $n$
vertices. It's only a bit of extra thought to determine that if you
only want to count the labeled graphs on $n$ vertices with $k$ edges,
you simply must choose a $k$-element subset of the set of all $C(n,2)$
possible edges. Thus, there are
\[\binom{\binom{n}{2}}{k}\]
graphs with vertex set $[n]$ and exactly $k$ edges.

A more difficult problem arises when we want to start counting
\emph{nonisomorphic} graphs on $n$ vertices. (One can think of these
as \emph{unlabeled} graphs as well.) For example, in
\autoref{fig:polya:graphs}, we show four different labeled graphs on
four vertices. The first three graphs shown there, however, are
isomorphic to each other. Thus, only two nonisomorphic graphs on four
vertices are illustrated in the figure. To account for isomorphisms,
we need to bring P\'olya's enumeration theorem into play.
\begin{figure}
  \centering
  \begin{overpic}[width=\linewidth]{polya-figs/graphs}
    \put(1.2,0.7){$1$}
    \put(19.7,0.7){$2$}
    \put(10.2,22){$3$}
    \put(10.2,7.2){$4$}

    \put(26.7,0.7){$1$}
    \put(45.2,0.7){$2$}
    \put(35.7,22){$3$}
    \put(35.7,7.2){$4$}

    \put(51.7,0.7){$1$}
    \put(70.2,0.7){$2$}
    \put(60.7,22){$3$}
    \put(60.7,7.2){$4$}

    \put(76.7,0.7){$1$}
    \put(95.2,0.7){$2$}
    \put(85.7,22){$3$}
    \put(85.7,7.2){$4$}

  \end{overpic}
  \caption{Four lalbeled graphs on four vertices}
  \label{fig:polya:graphs}
\end{figure}

We begin by considering all $2^{C(n,2)}$ graphs with vertex set $[n]$
and choosing an appropriate permutation group to act in the
situation. Since any vertex can be mapped to any other vertex, the
symmetric group $S_4$ acts on the vertices. However, we have to be
careful about how we find the cycle index here. When we were working
with colorings of the vertices of the square, we realized that all the
vertices appearing in the same cycle of a permutation $\pi$ had to be
colored the same color. Since we're concerned with edges here and not
vertex colorings, what we really need for a permutation to fix a graph
is that every edge be sent to an edge and every non-edge be sent to a
non-edge. To be specific, if $\{1,2\}$ is an edge of some $\bfG$ and
$\pi\in S_4$ fixes $\bfG$, then $\{\pi(1),\pi(2)\}$ must also be an
edge of $\bfG$. Similarly, if vertices $3$ and $4$ are not adjacent in
$\bfG$, then $\pi(3)$ and $\pi(4)$ must also be nonadjacent in
$\bfG$.

To account for edges, we move from the symmetric group $S_4$ to its
\emph{pair group} $S_4^{(2)}$. The objects that $S_4^{(2)}$ permutes
are the $2$-element subsets of $\{1,2,3,4\}$. For ease of notation, we
will denote the $2$-element subset $\{i,j\}$ by $e_{ij}$. To find the
permutations in $S_4^{(2)}$, we consider the vertex permutations in
$S_4$ and see how they permute the $e_{ij}$. The identity permutation
$\iota=(1)(2)(3)(4)$ of $S_4$ corresponds to the identity permutation
$\iota=(e_{12})(e_{13}) (e_{14}) (e_{23}) (e_{24}) (e_{34})$ of
$S_4^{(2)}$. Now let's consider the permutation $(12)(3)(4)$. It fixes
$e_{12}$ since it sends $1$ to $2$ and $2$ to $1$. It also fixeds
$e_{34}$ by fixing $3$ and $4$. However, it interchanges $e_{13}$ with
$e_{23}$ ($3$ is fixed and $1$ is swapped with $2$) and $e_{14}$ with
$e_{24}$ ($1$ is sent to $2$ and $4$ is fixed). Thus, the
corresponding permutation of pairs is
$(e_{12})(e_{13}e_{23})(e_{14}e_{24})(e_{34})$. For another example,
consider the permutation $(123)(4)$. It corresponds to the permutation
$(e_{12}e_{23}e_{13})(e_{14}e_{24}e_{34})$ in $S_4^{(2)}$.

Since we're only after the cycle index of $S_4^{(2)}$, we don't need
to find all $24$ permutations in the pair group. However, we do need
to know the types of those permutations in terms of cycle lengths so
we can associate the appropriate monomials. For the three examples
we've considered, the cycle structure of the permutation in the pair
group doesn't depend on the original permutation in $S_4$ other than
for \emph{its} cycle structure. Any permutation in $S_4$ consisting of
a $2$-cycle and two $1$-cycles will correspond to a permutation with
two $2$-cycles and two $1$-cycles in $S_4^{(2)}$. A permutation in
$S_4$ with one $3$-cycle and one $1$-cycle will correspond to a
permutation with two $3$-cycles in the pair group. By considering an
example of a permutation in $S_4$ consisting of a single $4$-cycle, we
find that the corresponding permutation in the pair group has a
$4$-cycle and a $2$-cycle. Finally, a permutation of $S_4$ consisting
of two $2$-cycles corresponds to a permutation in $S_4^{(2)}$ having
two $2$-cycles and two
$1$-cycles. (\hyperref[ex:polya:perm-pairs]{Exercise~\ref*{ex:polya:perm-pairs}}
asks you to verify these claims using specific permutations.)

Now that we know the cycle structure of the permutations in
$S_4^{(2)}$, the only task remaining before we can find its cycle
index of is to determine how many permutations have each of the
possible cycle structures. For this, we again refer back to
permutations of the symmetric group $S_4$. A permutation consisting of
a single $4$-cycle begins with $1$ and then has $2$, $3$, and $4$ in
any of the $3!=6$ possible orders, so there are $6$ such
permutations. For permutations consisting of a $1$-cycle and a
$3$-cycle, there are $4$ ways to choose the element for the $1$-cycle
and then $2$ ways to arrange the other three as a $3$-cycle. (Remember
the smallest of them must be placed first, so there are then $2$ ways
to arrange the remaining two.) Thus, there are $8$ such
permutations. For a permutation consisting of two $1$-cycles and a
$2$-cycle, there are $C(4,2)=6$ ways to choose the two elements for
the $2$-cycle. Thus, there are $6$ such permutations. For a
permutation to consist of two $2$-cycles, there are $C(4,2)=6$ ways to
choose two elements for the first $2$-cycle. The other two are then
put in the second $2$-cycle. However, this counts each permutation
twice, once for when the first $2$-cycle is the chosen pair and once
for when it is the ``other two.'' Thus, there are $3$ permutations
consisting of two $2$-cycles. Finally, only $\iota$ consists of four
$1$-cycles.

Now we're prepared to write down the cycle index of the pair group
\[P_{S_4^{(2)}}(x_1,\dots,x_6) = \frac{1}{24}\left( x_6^1 + 9x_1^2x_2^2 + 8 x_3^2 +
  6x_2x_4\right).\]
To use this to enumerate graphs, we can now make the substitution $x_i
= 1+x^i$ for $1\leq i\leq 6$. This allows us to account for the two
options of an edge not being present or being present. In doing so, we
find
\[P_{S_4^{(2)}}(1+x,\dots,1+x^6)= 1+x+2 x^2+3 x^3+2 x^4+x^5+x^6\]
is the generating function for the number of $4$-vertex graphs with
$m$ edges, $0\leq m\leq 6$. To find the total number of nonisomorphic
graphs on four vertices, we substitute $x=1$ into this
polynomial. This allows us to conclude there are $11$ nonisomorphic
graphs on four vertices, a marked reduction from the $64$ labeled
graphs.

The techniques of this subsection can be used, given enough computing
power, to find the number of nonisomorphic graphs on any number of
vertices. For $30$ vertices, there are 
\begin{align*}
  &334494316309257669249439569928080028956631479935393064329967834887217\\
  &734534880582749030521599504384\approx 3.3\times 10^{98}
\end{align*}
nonisomorphic graphs, as compared to $2^{435}\approx 8.9\times
10^{130}$ labeled graphs on $30$ vertices. The number of nonisomorphic
graphs with precisely $200$ edges is
\begin{align*}
  &313382480997072627625877247573364018544676703365501785583608267705079\\
  &9699893512219821910360979601\approx 3.1\times 10^{96}.
\end{align*}
The last part of the question about graph enumeration at the beginning
of the chapter was about enumerating the graphs on some number of
vertices in which every vertex has degree $r$. While this might seem
like it could be approached using the techniques of this chapter, it
turns out that it cannot because of the increased dependency between
where vertices are mapped.

\section{Exercises}\label{s:polya:ex}

\begin{enumerate}
\item Write the permutations shown below in cycle notation.\label{ex:polya:perm-rep}
  \begin{align*}
    \pi_1 &=
    \begin{pmatrix}
      1 & 2 & 3 & 4 & 5 & 6\\
      4 & 2 & 5 & 6 & 3 & 1
    \end{pmatrix}
   &
    \pi_2 &=
    \begin{pmatrix}
      1 & 2 & 3 & 4 & 5 & 6\\
      5 & 6 & 1 & 3 & 4 & 2
    \end{pmatrix}\\
   \pi_3 &=
    \begin{pmatrix}
      1 & 2 & 3 & 4 & 5 & 6 & 7 & 8\\
      3 & 1 & 5 & 8 & 2 & 6 & 4 & 7
    \end{pmatrix}
    &
   \pi_4 &=
    \begin{pmatrix}
      1 & 2 & 3 & 4 & 5 & 6 & 7 & 8\\
      3 & 7 & 1 & 6 & 8 & 4 & 2 & 5 
    \end{pmatrix}
\end{align*}
\item Compute $\pi_1\pi_2$, $\pi_2\pi_1$, $\pi_3\pi_4$, and
  $\pi_4\pi_3$ for the permutations $\pi_i$ in
  exercise~\ref{ex:polya:perm-rep}.
\item Find $\stab_{D_8}(C_3)$ and $\stab_{D_8}(C_{16})$ for the
  colorings of the vertices of the square shown in
  \autoref{fig:polya:squares} by referring to
  \autoref{tab:polya:fixed-square}.
\item In \autoref{fig:polya:pentagon}, we show a regular pentagon with
  its vertices labeled. Use this labeling to complete this exercise.
  \begin{figure}[h]
    \centering
    \begin{overpic}[scale=0.65]{polya-figs/pentagon}
      \put(9,2){$4$}
      \put(84,4){$3$}
      \put(102,53){$2$}
      \put(-8,53){$5$}
      \put(46,100){$1$}
    \end{overpic}
   \caption{A pentagon with labeled vertices}
    \label{fig:polya:pentagon}
  \end{figure}
  \begin{enumerate}
  \item The dihedral group of the pentagon, $D_{10}$, contains $10$
    permutations. Let $r_1=(12345)$ be the clockwise rotation by
    $72^\circ$ and $f_1=(1)(25)(34)$ be the flip about the line
    passing through $1$ and perpendicular to the opposite side. Let
    $r_2$, $r_3$, and $r_4$ be the other rotations in $D_{10}$. Denote
    the flip about the line passing through vertex $i$ and
    perpendicular to the other side by $f_i$, $1\leq i\leq 5$. Write
    all $10$ elements of $D_{10}$ in cycle notation.
  \item Suppose we are coloring the vertices of the pentagon using
    black and white. Draw the colorings fixed by $r_1$. Draw the
    colorings fixed by $f_1$.
  \item Find $\stab_{D_{10}}(C)$ where $C$ is the coloring of the
    vertices of the pentagon in which vertices $1$, $2$, and $5$ are
    colored black and vertices $3$ and $4$ are colored white.
  \item Find the cycle index of $D_{10}$.
  \item Use the cycle index to determine the number of nonequivalent
    colorings of vertices of the pentagon using black and white.
  \item Making an appropriate substitution for the $x_i$ in the cycle
    index, find the number of nonequivalent colorings of the vertices
    of the pentagon in which two vertices are colored black and three
    vertices are colored white. Draw these colorings.
  \end{enumerate}
\item Write all permutations in $C_{12}$, the cyclic group of order
  $12$, in cycle notation.\label{ex:polya:cyclic12}
\item The $12$-note western scale is not the only system on which
  music is based. In classical Thai music, a scale with seven
  equally-spaced notes per octave is used. As in western music, a
  scale is a subset of these seven notes, and two scales are
  equivalent if they are transpositions of each other. Find the number
  of $k$-note scales in classical Thai music for $1\leq k\leq 7$.
\item Xylene is an aromatic hydrocarbon having two methyl groups (and
  four hydrogen atoms) attached to the hexagonal carbon ring. How many
  isomers are there of xylene?
\item Find the permutations in $S_4^{(2)}$ corresponding to the
  permutations $(1234)$ and $(12)(34)$ in $S_4$. Confirm that the
  first consists of a $4$-cycle and a $2$-cycle and the second
  consists of two $2$-cycles and two $1$-cycles.\label{ex:polya:perm-pairs}
\item Draw the three nonisomorphic graphs on four vertices with $3$
  edges and the two nonisomorphic graphs on four vertices with $4$ edges.
\item
  \begin{enumerate}
  \item Use the method of \autoref{ss:polya:graphs} to find the cycle
    index of the pair group $S_5^{(2)}$ of the symmetric group on five
    elements.\label{ex:polya:graphs5-pt1}
  \item Use the cycle index from \ref{ex:polya:graphs5-pt1} to
    determine the number of nonisomorphic graphs on five vertices. How
    many of them have $6$ edges?
  \end{enumerate}
\item Tic-tac-toe is a two-player game played on a $9\times 9$
  grid. The players mark the squares of the grid with the symbols X
  and O. This exercise uses P\'olya's enumeration theorem to
  investigate the number of different tic-tac-toe boards. (The
  analysis of \emph{games} is more complex, since it requires
  attention to the order the squares are marked and stopping when one
  player has won the game.)
  \begin{enumerate}
  \item Two tic-tac-toe boards are equivalent if one may be obtained
    from the other by rotating the board or flipping it over. (Imagine
    that it is drawn on a clear piece of plastic.) Since the $9\times
    9$ grid is a square, the group that acts on it in this manner is
    the dihedral group $D_8$ that we have studied in this
    chapter. However, as with counting nonisomorphic graphs, we have
    to be careful to choose the way this group is represented in terms
    of cycles. Here we are interested in how permutations rearrange
    the nine squares of the tic-tac-toe board as numbered in
    \autoref{fig:polya:TTT}. For example, the effect of the
    transformation $r_1$, which rotates the board $90^\circ$
    clockwise, can be represented as a permutation of the nine squares
    as $(13971)(2684)(5)$.
   \begin{figure}
      \centering
          \begin{picture}(100,100)
      \thicklines
      \Line(33,0)(33,100)
      \Line(67,0)(67,100)
      \Line(0,33)(100,33)
      \Line(0,67)(100,67)
      \put(15,80){$1$}
      \put(48,80){$2$}
      \put(83,80){$3$}
      \put(15,45){$4$}
      \put(48,45){$5$}
      \put(83,45){$6$}
      \put(15,13){$7$}
      \put(48,13){$8$}
      \put(83,13){$9$}
    \end{picture}
    \caption{Numbered squares of a tic-tac-toe board}\label{fig:polya:TTT}
    \end{figure}

    Write each of the eight elements of $D_8$ as permutations of the
    nine squares of a tic-tac-toe board.
  \item Find the cycle index of $D_8$ in terms of these permutations.
  \item Make an appropriate substitution for $x_i$ in the cycle index
    to find a generating function $t(X,O)$ in which the coefficient on
    $X^iO^j$ is the number of nonequivalent tic-tac-toe boards having
    $i$ squares filled by symbol X and $j$ squares filled by symbol
    O. (Notice that some squares might be blank!)
  \item How many nonequivalent tic-tac-toe boards are there?
  \item How many nonequivalent tic-tac-toe boards have three X's and
    three O's?
  \item When playing tic-tac-toe, the players alternate turns, each
    drawing their symbol in a single unoccupied square during a
    turn. Assuming the first player marks her squares with X and the
    second marks his with O, then at each stage of the game there are
    either the same number of X's and O's or one more X than there are
    O's. Use this fact and $t(X,O)$ to determine the number of
    nonequivalent tic-tac-toe boards that can actually be obtained in
    playing a game, assuming the players continue until the board is
    full, regardless of whether one of them has won the game.
  \end{enumerate}
\item Suppose you are painting the faces of a cube and you have white,
  gold, and blue paint available. Two painted cubes are equivalent if
  you can rotate one of them so that all corresponding faces are
  painted the same color. Determine the number of nonequivalent ways
  you can paint the faces of the cube as well as the number having two
  faces of each color. \emph{Hint}: It may be helpful to label the
  faces as $U$ (``up''), $D$ (``down''), $F$ (``front''), $B$
  (``back''), $L$ (``left''), and $R$ (``right'') instead of using
  integers. Working with a three-dimensional model of a cube will also
  aid in identifying the permutations you require.
\end{enumerate}

%%% Local Variables: 
%%% mode: latex
%%% TeX-master: "chap-skel-mtk"
%%% End: 
