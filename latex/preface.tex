% preface.tex
% Updated January 11, 2012

\chapter{Preface}\label{ch:preface}

At Georgia Tech, MATH 3012: Applied Combinatorics, is a junior-level
course targeted primarily at students pursuing the B.S. in
Computer Science.  The purpose of the course is to give students a
broad exposure to combinatorial mathematics, using applications to
emphasize fundamental concepts and techniques.
Applied Combinatorics is also required of students seeking the B.S. in
Applied Mathematics or the B.S in Discrete Mathematics, and it is one
of two discrete mathematics courses that computer engineering students may select to
fulfill a breadth requirement.  The course will also often contain a selection
of other engineering and science majors who are interested in learning
more mathematics.   As a consequence, in a typical semester, some 250 Georgia
Tech students are enrolled in Applied Combinatorics. Students enrolled
in Applied Combinatorics at Georgia Tech have already completed the
three semester calculus sequence---with many students bypassing
one or more of the these courses on the basis of advanced placement
scores.  Also, the students will know some linear algebra and can
at least have a reasonable discussion about vector spaces, bases and
dimension.  

Our approach to the course is to show students the beauty of
combinatorics and how combinatorial problems naturally arise in many
settings, particularly in computer science. While proofs are
periodically presented in class, the course is not intended to teach
students how to write proofs; there are other required courses in our
curriculum that meet this need. Students may occasionally be asked to
prove small facts, but these arguments are closer to the kind we
expect from students in second or third semester calculus as contrasted
with proofs we expect from a mathematics major in an upper-division course.
Regardless, we cut very few corners, and our text can readily be
used by instructors who elect to be even more rigorous in their
approach.

This book arose from our feeling that a text that met our approach to
Applied Combinatorics was not available. Because of the diverse set of
instructors assigned to the course, the standard text was one that
covered every topic imaginable (and then some), but provided little
depth. We've taken a different approach, attacking the central
subjects of the course description to provide exposure, but taking the
time to go into greater depth in select areas to give the students a
better feel for how combinatorics works. We have also included some
results and topics that are not found in other texts at this level but
help reveal the nature of combinatorics to students. We want 
students to understand that combinatorics is a subject
that you must feel ``in the gut'', and we hope that our presentation
achieves this goal. The emphasis throughout remains on applications,
including algorithms. We do not get deeply into the details of what it
means for an algorithm to be ``efficient'', but we do include an informal
discussion of the basic principles of complexity, intended 
to prepare students in computer science, engineering and applied mathematics
for subsequent coursework.

The materials included in this book have evolved over time.  Early
versions of a few chapters date from 2004, but the pace quickened in
2006 when the authors team taught a large section of Applied 
Combinatorics.  In the last five years, existing chapters have been
updated and expanded, while new chapters
have been added.  As matters now stand, our book includes more material
than we can cover in a single semester. We feel that the topics of
Chapters~1 through~9 plus Chapters 12, 13 and 14 are the core of
a one semester course in Applied Combinatorics. 
Additional topics can then be selected from the remaining chapters 
based on the interests of the instructor and students. 

We are grateful to our colleagues Alan Diaz, Thang Le, Noah Strebi, Prasad Tetali
and Carl Yerger, who have taught Applied Combinatorics from preliminary 
versions and have given valuable feedback.  As this text is freely
available on the internet, we welcome comments, criticisms, suggestions
and corrections from anyone who takes a look at our work. 

\vspace{.5in}
\noindent
\textbf{About the Authors}

\medskip
\noindent
William T. Trotter is a Professor in the School of Mathematics at
Georgia Tech.  He was first exposed to combinatorial mathematics through 
the 1971 Bowdoin Combinatorics Conference which featured an array of 
superstars of that era, including Gian Carlo Rota, Paul Erd\H{o}s, 
Marshall Hall, Herb Ryzer, Herb Wilf, William Tutte, Ron Graham, 
Daniel Kleitman and Ray Fulkerson.
Since that time, he has published more than 120 research papers on graph theory,
discrete geometry, ramsey theory and extremal combinatorics.  Perhaps
his best known work is in the area of combinatorics and partially ordered
sets, and his 1992 research monograph on this topic has been very influential.
(He takes some pride in the fact that this monograph is still in print
and copies are being sold in 2014.) He has more than 70 co-authors, but
considers his extensive joint work with Graham Brightwell, 
Stefan Felsner, Peter Fishburn, Hal Kierstead and Endre Szemer\`{e}di 
as representing his best work.  His career includes invited 
presentations at more than 50 international conferences
and more than 30 meetings of professional societies.  He was the founding
editor of the \textit{SIAM Journal on Discrete Mathematics} and has served on
the Editorial Board of \textit{Order} since the journal was launched
in 1984, and his service includes an eight year stint as Editor-in-Chief.  Currently,
he serves on the editorial boards of three other journals in combinatorial
mathematics.

Still he has his quirks. First, he insists on being called ``Tom'',
as Thomas is his middle name, while continuing to sign as William T. Trotter.
Second, he has invested time and energy serving five terms as department/school
chair, one at Georgia Tech, two at Arizona State University and two at
the University of South Carolina.  In addition, he has served as
a Vice Provost and as an Assistant Dean.  Third, he is fascinated by
computer operating systems and is always installing new ones.  In one
particular week, he put eleven different flavors of Linux on the
same machine, interspersed with four complete installs of Windows~7.
Incidentally, the entire process started and ended with Windows~7.  Fourth, he
likes to hit golf balls, not play golf, just hit balls.  Without these
diversions, he might even have had enough time to settle the Riemann hypothesis.

He has had eleven Ph.D. students, one of which is now his co-author on this
text.

\bigskip
\noindent
Mitchel T. Keller is a super-achiever (this description is written by WTT) 
extraordinaire from North Dakota.  As a graduate student at Georgia Tech,
he won a lengthy list of honors and awards, including a Vigre Graduate
Fellowship, an IMPACT Scholarship, a John R.~Festa Fellowship and
the 2009 Price Research Award.  Mitch is a natural leader and was
elected President (and Vice President) of the Georgia Tech Graduate
Student Governance Association, roles in which he served with distinction.
Indeed, after completing his terms, his student colleagues voted to
establish a continuing award for distinguished leadership, to be named
the Mitchel T.~Keller award, with Mitch as the first recipient.  Very
few graduate students win awards in the first place, but Mitch is the
only one I know who has an award \textit{named} after them.

Mitch is also a gifted teacher of mathematics, receiving the prestigious
Georgia Tech 2008 Outstanding Teacher Award, a campus-wide competition.  
He is quick to experiment with the latest approaches to teaching mathematics, 
adopting what works for him while refining and polishing things along the
way.  He really understands the literature behind active learning
and the principles of engaging students in the learning process.  
Mitch has even taught his more senior (some say ancient)
co-author a thing or two and got him to try personal response
systems in a large calculus section this fall.

Mitch is off to a fast start in his own research career, and is already
an expert in the subject of linear discrepancy.  Mitch has also made
substantive contributions to a topic known as Stanley depth,
which is right at the boundary of combinatorial mathematics and
algebraic combinatorics.

%\newpage
After finishing his Ph.D., Mitch received another signal honor,
a Marshall Sherfield Postdoctoral Fellowship and spent 
two years at the London School of Economics.  He is presently an
Assistant Professor of Mathematics at Washington and Lee University, and a few years down the road,
he'll probably be president of something.

On the personal side, Mitch is the keeper of the Mathematical
Genealogy Project, and he is a great cook.  His desserts are to
die for.

\begin{flushright}
  Mitch Keller, \texttt{kellermt@wlu.edu}\\
  Tom Trotter, \texttt{trotter@math.gatech.edu}\\
  \textit{Lexington, Virginia and Atlanta, Georgia}
\end{flushright}

%%% Local Variables: 
%%% mode: latex
%%% TeX-master: "book"
%%% End: 
