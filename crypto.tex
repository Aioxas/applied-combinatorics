\chapter{Cryptography}\label{ch:crypto}

We begin with a quick tour of encryption/decryption
techniques---substitution ciphers (frequency analysis vulnerability),
one-time pads (implementation nightmare), modern encryption algorithms
like AES. Modern algorithms are usually designed using advanced
algebraic techniques to allow them to run quickly while still
providing security.

The secure exchange of keys for use in algorithms such as AES is
difficult, especially if the two parties who wish to communicate have
never met. One of the most important places this problem arises is
secure communication over the Internet. How can Alice securely
communicate with TBTF Bank's website? TBTF has millions of customers,
so it's unreasonable for them to pre-arrange an encryption key to use
with each one of them. Instead, a scheme known as public-key
encryption is used. Public-key encryption uses the current difficulty
of a mathematical problem such as factoring very large integers to
encrypt data. Alice publishes a \emph{public key} for anyone to use to
encrypt a message they send her. She retains her secret \emph{private
  key} for use in decrypting messages she receives. Bob uses Alice's
public key to encrypt his message and sends it to Alice, who uses her
private key to decrypt it. An eavesdropper could decrypt the message
if they had either Alice's private key or could solve the difficult
mathematical problem involved. However, as long as Alice practices
good security with her private key and the hard problem remains hard,
the data is secure.

Let's take a look at the RSA algorithm as an example of a public-key
cryptosystem. (Uses Euler $\phi$, Euclidean algorithm, and difficulty
of factoring.)

So why did we spend time talking about algorithms like AES at the
beginning of the chapter? RSA using keys of $512$ bits is considered
compromised, as $n$ can be factored in a matter of weeks or
less. There is speculation that even $1024$-bit keys will become
vulnerable in the near future, and so it is recommended that any RSA
implementation use keys of at least $2048$ bits. The use of such large
keys slows encryption and decryption, making RSA difficult to use for
secure transmission of large amounts of data. Thus, in reality, TBTF
Bank's web site uses $2048$-bit RSA encryption to communicate with
Alice only for purposes of exchanging a secure key, usually between
$128$ bits and $256$ bits, for use in an algorithm such as AES.

It's also not out of the realm of possibility to talk about
Diffie-Helman key exchange, which is similar to a public-key system
like RSA. However, it is not itself a method of encryption. Instead,
it is merely a mechanism for using information transmitted in the open
to agree on an encryption key for use in a symmetric key cryptosystem
like AES. Diffie-Helman key exchange relies on the discrete logarithm
problem in finite groups such as the multiplicative group of integers
modulo a large integer.

%%% Local Variables: 
%%% mode: latex
%%% TeX-master: "chap-skel-mtk"
%%% End: 