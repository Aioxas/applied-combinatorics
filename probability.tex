% probability.tex
% Updated January 11, 2012

\chapter{Probability}\label{ch:probability}

It was a slow day and Dave said he was bored.  It was just after lunch,
and he complained that there was nothing to do.  Nobody really seemed
to be listening, although Alice said that Dave might consider
studying, even reading ahead in the chapter.  Undeterred, Dave said
``Hey Alice, how about we play a game.  We could take turns tossing a
coin, with the other person calling heads or tails.  We could keep
score with the first one to a hundred being the winner.''  Alice
rolled her eyes at such a lame idea.  Sensing Alice's lack of
interest, Dave countered ``OK, how about a hundred games of Rock, Paper
or Scissors?''  Zori said ``Why play a hundred times?  If that's what
you're going to do, just play a single game.''

Now it was Alice's turn.  ``If you want to play a game, I've got a
good one for you.  Just as you wanted, first one to score a hundred
wins.  You roll a pair of dice.  If you roll doubles, I win $2$
points.  If the two dice have a difference of one, I win $1$ point.
If the difference is $2$, then it's a tie.  If the difference is $3$,
you win one point; if the difference is~$4$, you win two points; and if
the difference is $5$, you win three points.  Xing interrupted to say
``In other words, if the difference is $d$, then Dave wins $d-2$
points.''  Alice continues ``Right!  And there are three ways Dave can
win, with one of them being the biggest prize of all.  Also, rolling
doubles is rare, so this has to be a good game for Dave.''

Zori's ears perked up with Alice's description.  She had a gut feeling
that this game wasn't really in Dave's favor and that Alice knew what
the real situation was.  The idea of a payoff with some
uncertainty involved seemed very relevant.  Carlos was scribbling on a piece of paper,
then said politely ``Dave, you really should be reading ahead in the
chapter''.

So what do you think?  Is this a fair game?  What does it mean for a
game to be fair?  Should Dave play---independent of the question of
whether such silly stuff should occupy one's time?  And what does any
of this conversation have to do with combinatorics?

\section{An Introduction to Probability}

We continue with an informal discussion intended to motivate
the more structured development that will follow.  Consider
the ``spinner'' shown in \autoref{fig:spinner}.  Suppose we give it 
a good thwack so that the arrow goes round and round.  We then 
record the number of the region in which the pointer comes to rest.  
Then observers, none of whom have studied combinatorics, might
make the following comments: 

\begin{figure}
\begin{center}
\includegraphics*[scale=.4]{probability-figs/spinner.pdf}
\caption{A Spinner for Games of Chance}
\label{fig:spinner}
\end{center}
\end{figure}

\begin{enumerate}
\item The odds of landing in region~$1$ are the same as those for
landing in region~$3$.
\item You are twice as likely to land in region~$2$ as in region~$4$.
\item When you land in an odd numbered region, then 60\% of the time,
it will be in region~$5$.
\end{enumerate}

We will now develop a more formal framework that will enable us to
make such discussions far more precise. We will also see whether Alice
is being entirely fair to Bob in her proposed game to one hundred.

We begin by defining a \textit{probability space} as a pair $(S,P)$ 
where $S$ is a finite set and $P$ is a function that whose domain is
the family of all subsets of $S$ and whose range is the set $[0,1]$ of
all real numbers which are non-negative and at most one.  Furthermore,
the following two key properties must be satisfied:

\begin{enumerate}
\item $P(\emptyset)=0$ and $P(S)=1$.
\item If $A$ and $B$ are subsets of $S$, and $A\cap B=\emptyset$,
then $P(A\cup B)= P(A)+P(B)$.
\end{enumerate}

When $(S,P)$ is a probability space, the function $P$ is called a 
\textit{probability measure}, the subsets of $S$ are called
\textit{events}, and when $E\subseteq S$, the quantity $P(E)$ is 
referred to as the \textit{probability} of the event $E$.  

Note that we can consider
$P$ to be extended to a mapping from $S$ to $[0,1]$ by
setting $P(x)=P(\{x\})$ for each element $x\in S$.
We call the elements of $S$ \textit{outcomes}
(some people prefer to say the elements are \textit{elementary
outcomes}) and the quantity $P(x)$ is called the \textit{probability}
of $x$.  It is important to realize that if you know $P(x)$ for
each $x\in S$, then you can calculate $P(E)$ for any event $E$,
since (by the second property), $P(E)=\sum_{x\in X}P(x)$. 

\begin{example}
For the spinner, we can take $S=\{1,2,3,4,5\}$, with $P(1)=P(3)=P(4)=1/8$,
$P(2)=2/8=1/4$ and $P(5)=3/8$.  So $P(\{2,3\})=1/8+2/8=3/8$.
\end{example}

\begin{example}
Let $S$ be a finite, nonempty set and let $n=|S|$.  
For each $E\subseteq S$, set $P(E)=|E|/n$.  In particular, $P(x)=1/n$ 
for each element $x\in S$.  In this trivial example, all outcomes 
are equally likely.
\end{example}

\begin{example}
If a single six sided die is rolled and the number of dots on the
top face is recorded, then the ground set is $S=\{1,2,3,4,5,6\}$ and
$P(i)=1/6$ for each $i\in S$.  On the other hand, if a pair of dice
are rolled and the sum of the dots on the two top faces is recorded,
then $S=\{2,3,4,\dots,11,12\}$ with $P(2)=P(12) =1/36$, $P(3)=P(11)=2/36$,
$P(4)=P(10)=3/36$, $P(5)=P(9)=4/36$, $P(6)=P(8)=5/36$ and $P(7)=6/36$.
To see this, consider the two die as distinguished, one die red and the
other one blue.  Then each of the pairs $(i,j)$ with $1\le i,j\le 6$, 
the red die showing $i$ spots and the blue die showing $j$ spots is
equally likely.  So each has probability $1/36$.  Then, for example,
there are three pairs that yield a total of four, namely $(3,1)$, $(2,2)$
and $(1,3)$.  So the probability of rolling a four is $3/36=1/12$.
\end{example}

\begin{example}
In Alice's game as described above, the set $S$ can be taken
as $\{0,1,2,3,4,5\}$, the set of possible differences when a pair
of dice are rolled.  In this game, we will see that the correct
definition of the function $P$ will set $P(0)=6/36$; $P(1)=10/36$;
$P(2)=8/36$; $P(3)=6/36$; $P(4)=4/36$; and $P(5)=2/36$.  Using
Xing's more compact notation, we could say that $P(0)=1/6$ and
$P(d)= 2(6-d)/36$ when $d>0$.
\end{example}

\begin{example}
  A jar contains twenty marbles, of which six are red, nine are blue
  and the remaining five are green.  Three of the twenty marbles are
  selected at random.\footnote{This is sometimes called
    \textit{sampling without replacement}.  You should imagine a jar
    with opaque sides---so you can't see through them.  The marbles
    are stirred/shaken, and you reach into the jar blind folded and
    draw out three marbles.}  Let $X=\{0,1,2,3,4,5\}$, and for each
  $x\in X$, let $P(x)$ denote the probability that the number of blue
  marbles among the three marbles selected is $x$.  Then
  $P(i)=C(9,i)C(11,3-i)/C(20,3)$ for $i=0,1,2,3$, while $P(4)=P(5)=0$.
  Bob says that it doesn't make sense to have outcomes with
  probability zero, but Carlos says that it does.
\end{example}

\begin{example}
In some cards games, each player receives five cards from a
standard deck of~$52$ cards---four suits (spades, hearts, diamonds
and clubs) with~$13$ cards, ace though king in each suit.  A player 
has a \textit{full house} if there are two values $x$ and $y$ for
which he has three of the four $x$'s and two of the four $y$'s, e.g.
three kings and two eights.  If five cards are drawn at random from
a standard deck, the probability of a full house is 
\[
\frac{\binom{13}{1}\binom{12}{1}\binom{4}{3}\binom{4}{2}}{\binom{52}{5}}\approx 0.00144.
\]
\end{example}

\section{Conditional Probability and Independent Events}

A jar contains twenty marbles of which six are red, nine are blue
and the remaining five are green.  While blindfolded, Xing selects two of the twenty 
marbles random (without replacement) and puts one in his left pocket and
one in his right pocket.  He then takes off the blindfold.

The probability that the marble in his left pocket is red is $6/20$.
But Xing first reaches into his right pocket, takes this marble out and
discovers that it is is blue.  Is the probability that the marble in
his left pocket is red still $6/20$?  Intuition says that it's slightly
higher than that.  Here's a more formal framework for answering such
questions.

Let $(S,P)$ be a probability space and let $B$ be an event for
which $P(B)>0$.  Then for every event $A\subseteq S$, we define
the \textit{probability of $A$, given $B$}, denoted $P(A|B)$, by setting
$P(A|B)=P(A\cap B)/P(B)$.  

\begin{discussion}
  Returning to the question raised at the beginning of the section,
  Bob says that this is just conditional probability. He says let $B$
  be the event that the marble in the right pocket is blue and let $A$
  be the event that the marble in the left pocket is red.  Then
  $P(B)=9/20$, $P(A) = 6/20$ and $P(A\cap B)=(9\cdot6)/380$, so that
  $P(A|B)= \frac{54}{380}\frac{20}{9}=6/19$, which is of course
  slightly larger than $6/20$.  Alice is impressed.
\end{discussion}

\begin{example}\label{exa:twojars}
  Consider the jar of twenty marbles from the preceding example.  A
  second jar of marbles is introduced.  This jar has eighteen marbles:
  nine red, five blue and four green.  A jar is selected at random and
  from this jar, two marbles are chosen at random.  What is the
  probability that both are green?  Bob is on a roll.  He says ``Let $G$
  be the event that both marbles are green, and let $J_1$ and $J_2$ be
  the event that the marbles come from the first jar and the second
  jar, respectively.  Then $G= (G\cap J_1)\cup (G\cap J_2)$, and
  $(G\cap J_1)+(G\cap J_2)=\emptyset$.  Furthermore,
  $P(G|J_1)=\binom{5}{2}/\binom{20}{2}$ and
  $P(G|J_2)=\binom{4}{2}/\binom{18}{2}$, while $P(J_1)=P(J_2)=1/2$.
  Also $P(G\cap J_i)=P(J_i)P(G|J_i)$ for each $i=1,2$.  Therefore,
\[
P(G)=\frac{1}{2}\frac{\binom{5}{2}}{\binom{20}{2}}+
\frac{1}{2}\frac{\binom{4}{2}}{\binom{18}{2}}=\frac{1}{2}\bigl(\frac{20}{380}+
\frac{12}{306}\bigr).\text{''}
\]

Now Alice is speechless.
\end{example}

\subsection{Independent Events}

Let $A$ and $B$ be events in a proability space $(S,P)$.  We say $A$
and $B$ are \textit{independent} if $P(A\cap B)=P(A)P(B)$.  Note that
when $P(B)\neq 0$, $A$ and $B$ are independent if and only if
$P(A)=P(A|B)$. Two events that are not independent are said to be
\emph{dependent}. Returning to our earlier example, the two events
($A$: the marble in Xing's left pocket is red and $B$: the marble in
his right pocket is blue) are dependent.

\begin{example}
  Consider the two jars of marbles from
  \hyperref[exa:twojars]{Example~\ref*{exa:twojars}}.  One of the two
  jars is chosen at random and a single marble is drawn from that jar.
  Let $A$ be the event that the second jar is chosen, and let $B$ be
  the event that the marble chosen turns out to be green.  Then
  $P(A)=1/2$ and $P(B)=\frac{1}{2}\frac{5}{20}+
  \frac{1}{2}\frac{4}{18}$.  On the other hand, $P(A\cap
  B)=\frac{1}{2} \frac{4}{18}$, so $P(A\cap B)\neq P(A)P(B)$, and the
  two events are not independent.  Intuitively, this should be clear,
  since once you know that the marble is green, it is more likely that
  you actually chose the first jar.
\end{example}

\begin{example}\label{exa:twodie}
A pair of dice are rolled, one red and one blue. Let $A$ be the event
that the red die shows either a $3$ or a $5$, and let $B$ be the event that you
get doubles, i.e., the red die and the blue die show the same number.
Then $P(A)=2/6$, $P(B)=6/36$, and $P(A\cap B) = 2/36$.  So $A$ and $B$
are independent. 
\end{example}

\section{Bernoulli Trials}

Suppose we have a jar with $7$ marbles, four of which are red
and three are blue.  A marble is drawn at random and we record whether it
is red or blue.  The probability $p$ of getting a red marble
is $4/7$; and the probability of getting a blue is $1-p=3/7$.

Now suppose the marble is put back in the jar, the marbles in the
jar are stirred, and the experiment is repeated.  Then the probability
of getting a red marble on the second trial is again $4/7$, 
and this pattern holds regardless of the number of times the experiment 
is repeated.  

It is customary to call this situation a series of \textit{Bernoulli
trials}.  More formally, we have an experiment with only two
outcomes: \textit{success} and \textit{failure}.  The probability
of success is $p$ and the probability of failure is $1-p$.  Most importantly,
when the experiment is repeated, then the probability of success on any
individual test is exactly $p$.  

We fix a positive integer $n$ and consider the case that the experiment
is repeated $n$ times.  The outcomes are then the binary strings of
length $n$ from the two-letter alphabet $\{S,F\}$, for success and failure, 
respectively.  If $x$ is a string with $i$ sucesses and $n-i$ failures, then 
$P(x)=\binom{n}{i}p ^i(1-p)^{n-i}$.  Of course, in applications, success
and failure may be replaced by: head/tails, up/down, good/bad, forwards/backwards,
red/blue, etc.

\begin{example}
  When a die is rolled, let's say that we have a success if the result
  is a two or a five.  Then the probability $p$ of success is
  $2/6=1/3$ and the probability of failure is $2/3$.  If the die is
  rolled ten times in succession, then the probability that we get
  exactly exactly four successes is $C(10,4)(1/3)^4 (2/3)^{6}$.
\end{example}

\begin{example}
A fair coin is tossed $100$ times and the outcome (heads or tails)
is recorded.  Then the probability of getting heads $40$ times
and tails the other $60$ times is 
\[
\binom{100}{40}\left(\frac{1}{2}\right)^{40}\left(\frac{1}{2}\right)^{60}=\frac{\binom{100}{40}}{2^{100}}.
\]
\end{example}

\begin{discussion}
Bob says that if a fair coin is tossed $100$ times, it is fairly likely
that you will get exactly $50$ heads and $50$ tails.  Dave is not so certain
this is right.  Carlos fires up his computer and in few second, he reports
that the probability of getting exactly $50$ heads when a fair coin is
tossed $100$ times is
\[
\frac{12611418068195524166851562157}{158456325028528675187087900672}
\]
which is $.079589$, to six decimal places. 
In other words, not very likely at all.  Xing is doing a modestly more complicated
calculation, and he reports that you have a $99$\% chance that the number
of heads is at least $20$ and at most $80$.  Carlos adds that when $n$ 
is very large, then it is increasingly certain that the number of heads
in $n$ tosses will be close to $n/2$.  Dave asks what do you mean by
close, and what do you mean by very large?
\end{discussion}

\section{Discrete Random Variables}

Let $(S,P)$ be a probability space and let
$X:S\longrightarrow\mathbb{R}$ be any function that maps the outcomes
in $S$ to real numbers (all values allowed, positive, negative and
zero).  We call\footnote{For historical reasons, capital letters, like
  $X$ and $Y$ are used to denote random variables.  They are just
  functions, so letters like $f$, $g$ and $h$ might more seem more
  natural---but maybe not.} $X$ a \textit{random variable}.  The
quantity $\sum_{x\in S} X(x)P(x)$, denoted $E(X)$, is called the
\textit{expectation} (also called the \textit{mean} or \emph{expected
  value}) of the random variable $X$.  As the suggestive name
reflects, this is what one should expect to be the average behavior of
the result of repeated Bernoulli trials.

Note that since we are dealing only with probability spaces $(S,P)$ where
$S$ is a finite set, the range of the probability measure $P$ is actually
a finite set.  Accordingly, we can rewrite the formula for $E(X)$ as 
$\sum_y y\cdot \prob(X(x)=y)$, where the summation extends over a finite
range of values for $y$.

\begin{example}
For the spinner shown in \autoref{fig:spinner}, let $X(i)=i^2$ where
$i$ is the number of the region.  Then 
\[
E(X)=\sum_{i\in S} i^2P(i)=1^2\frac{1}{8}+2^2\frac{2}{8}+3^2\frac{1}{8}+
   4^2\frac{1}{8}+5^2\frac{3}{8}=\frac{109}{8}.
\]
Note that $109/8=13.625$.  The significance of this quantity is
captured in the following statement.  If we record the result from the
spinner $n$ times in succession as $(i_1,i_2,\dots,i_n)$ and Xing
receives a prize worth $i_j^2$ for each $j=1,2,\dots,n)$, then Xing
should ``expect'' to receive a total prize worth $109n/8=13.625n$.
Bob asks how this statement can possibly be correct, since $13.625n$
may not even be an integer, and any prize Xing receives will have
integral value.  Carlos goes on to explain that the concept of
expected value provides a formal definition for what is meant by a
fair game.  If Xing pays $13.625$ cents to play the game and is then
paid $i^2$ pennies where $i$ is the number of the region where the
spinner stops, then the game is fair.  If he pays less, he has an
unfair advantage, and if he pays more, the game is biased against him.
Bob says ``How can Xing pay $13.625$ pennies?''  Brushing aside Bob's
question, Carlos says that one can prove that for every $\epsilon >0$,
there is some $n_0$ (which depends on $\epsilon$) so that if $n>n_0$,
then the probability that Xing's total winnings minus $13.625n$,
divided by $n$ is within $\epsilon$ of $13.625$ is at least
$1-\epsilon$.  Carlos turns to Dave and explains politely that this
statement gives a precise meaning of what is meant by ``close'' and
``large''.
\end{example}

\begin{example}
For Alice's game as detailed at the start of the chapter,
$S=\{0,1,2,3,4,5\}$, we could take $X$ to be the function defined by
$X(d)= 2-d$.  Then $X(d)$ records the amount that Bob wins when
the difference is $d$ (a negative win for Bob is just a win for Alice
in the same amount).  We calculate the expectation of $X$ as follows:
\[
E(X)=\sum_{d=0}^{5}X(d)p(d)= -2\frac{1}{6} -1\frac{10}{36}+0
     \frac{8}{36}+1\frac{6}{36}+2\frac{4}{36}+3{2}{36}=\frac{-2}{36}.
\]
Note that $-2/36=-.055555\dots$.  So if points were dollars, each
time the game is played, Bob should expect to lose slightly more
than a nickel.  Needless to say, Alice likes to play this game and
the more times Bob can be tricked into playing, the more she likes it.
On the other hand, by this time in the chapter, Bob should be getting
the message and telling Alice to go suck a lemon.
\end{example}

\subsection{The Linearity of Expectation}

The following fundamental property of expectation is an immmediate 
consequence of the definition, but we state it formally because it 
is so important to discussions to follow.

\begin{proposition}\label{prop:linearexpect}
Let $(S,P)$ be a probability space and let $X_1,X_2,\dots,X_n$ be
random variables.  Then
\[
E(X_1+X_2+\dots+X_t)=E(X_1)+E(X_2)+\dots+E(X_n).
\]
\end{proposition}


\subsection{Implications for Bernoulli Trials}

\begin{example}
Consider a series of $n$ Bernoulli trials with $p$, the probability
of success, and let $X$ count the number of successes.  Then, we claim
that
\[
E(X)=\sum_{i=0}^n i\binom{n}{i}p^i(1-p)^{n-i}=np
\]
To see this, consider the function $f(x)=[px+(1-p)]^n$. Taking the
derivative by the chain rule, we find that $f'(x)=np[px+(1-p)]^{n-1}$.
Now when $x=1$,  the derivative has value $np$.

On the other hand, we can use the binomial theorem to expand the function $f$.
\[
f(x)=\sum_{i=0}^n \binom{n}{i}x^ip^i(1-p)^{n-i}
\]
It follows that
\[
f'(x)=\sum_{i=0}^n i \binom{n}{i}x^{i-}p^i(1-p)^{n-i}
\]
And now the claim follows by again setting $x=1$.  Who says calculus isn't
useful!
\end{example}

\begin{example}
Many states have lotteries to finance college scholarships
or other public enterprises judged to have value to the public
at large.  Although far from a scientific investigation, it seems on
the basis of our investigation that many of the games have an 
expected value of approximately fifty cents when one
dollar is invested.  So the games are far from fair, and no
one should play them unless they have an intrinsic desire
to support the various causes for which the lottery profits
are targeted.

By contrast, various games of chance played in gambling
centers have an expected return of slightly less than ninety
cents for every dollar wagered.  In this setting, we can
only say that one has to place a dollar value on the 
enjoyment derived from the casino environment.  From a mathematical
standpoint, you are going to lose.  That's how they get the money
to build those exotic buildings.
\end{example}

\section{Central Tendency}

Consider the following two situations.

Situation 1.\quad A small town decides to hold a lottery to raise funds for
charitable purposes.  A total of $10,001$ tickets are sold, and the
tickets are labeled with numbers from the set $\{0,1,2,\dots,10,000\}$. 
At a public ceremony, duplicate tickets are placed in a big box, and
the mayor draws the winning ticket from out of the box.  Just to heighten 
the suspense as to who has actually won the prize, the mayor reports that 
the winning number is at least $7,500$.  The citizens ooh and aah and they
can't wait to see who among them will be the final winner.

Situation 2.\quad Behind a curtain, a fair coin is tossed $10,000$ times, 
and the number of heads is recorded by an observer, who is reputed to
be honest and impartial.   Again, the outcome is an integer in the 
set $\{0,1,2,\dots,10,000\}$.  The observer then emerges from behind 
the curtain and announces that the number of heads is at least than $7,500$.  
There is a pause and then someone says ``What?  Are you out of your mind?''

So we have two probability spaces, both with sample space
$S=\{0,1,2,\dots,10,000\}$.  For each, we have a random variable $X$,
the winning ticket number in the first situation, and the number of
heads in the second.  In each case, the expected value, $E(X)$, of the
random variable $X$ is $5,000$.  In the first case, we are not all
that surprised at an outcome far from the expected value, while in the
second, it seems intuitively clear that this is an extraordinary
occurrence.  The mathematical concept here is referred to as
\textit{central tendency}, and it helps us to understand just how
likely a random variable is to stray from its expected value.

For starters, we have the following elementary result which is
called Markov's inequality.

\begin{theorem}\label{thm:markov}
Let $X$ be a random variable in a probability
space $(S,P)$.  Then for every $k>0$,
\[
P(|X|\ge k) \le E(|X|)/k.
\]
\end{theorem}
\begin{proof}
Of course, the inequality holds trivially unless
$k> E(|X|)$.  For $k$ in this range, we
establish the equivalent inequality: $k P(|X|\ge k)\le E(|X|)$.
\begin{align*}
k P(|X|\ge k) &   = \sum_{r\ge k} k P(|X|=r)\\
              & \le \sum_{r \ge k} r P(|X|=r)\\
              & \le \sum_{r> 0} r P(|X|=r)\\
              &= E(|X|).
\end{align*}
\end{proof}

To make Markov's inequality more concrete, we see that on the
basis of this trivial  result, the probability that either the
winning lottery ticket or the number of heads is at least $7,500$
is at most $5000/7500=2/3$.  So nothing alarming here in either
case.  Since we still feel that the two cases are quite different,
a more subtle measure will be required.

\subsection{Variance and Standard Deviation}

Again, let $(S,P)$ be a probability space and let $X$ be a random variable.
The quantity $E((X-E(X))^2)$ is called the \textit{variance} of $X$ and
is denoted $\var(X)$.  Evidently, the variance of $X$ is a non-negative
number.  The \textit{standard deviation} of $X$, denoted $\sigma_X$ is then
defined as the quantity $\sqrt{\var(x)}$, i.e., $\sigma_X^2 =\var(X)$.

\begin{example}
For the spinner shown at the beginning of the chapter, let $X(i)=i^2$ when
the pointer stops in region~$i$.  Then we have already noted that the
expectation $E(X)$ of the random variable $X$ is $109/8$.  It follows that
the variance $\var(X)$ is:
\begin{align*}
\var(X)&=(1^2-\frac{109}{8})^2\frac{1}{8}+(2^2-\frac{109}{8})^2\frac{1}{4}+
          (3^2-\frac{109}{8})^2\frac{1}{8}+(4^2-\frac{109}{8})^2\frac{1}{8}+
          (5^2-\frac{109}{8})^2\frac{3}{8}\\
       &=(108^2+105^2+100^2+93^2+84^2)/512\\
       &=48394/512
\end{align*}
It follows that the standard deviation $\sigma_X$ of $X$ is then
$\sqrt{48394/512}\approx 9.722$.
\end{example}

\begin{example}
Suppose that $0<p<1$ and consider a series of $n$ Bernoulli trials with 
the probability of success being $p$, and let $X$ count the number of 
successes.  We have already noted that $E(X)=np$.  Now we claim the the 
variance of $X$ is given by:

\[
\var(X)=\sum_{i=0}^n (i-np)^2\binom{n}{i}p^i(1-p)^{n-i} = np(1-p)
\]

There are several ways to establish this claim.  One way is to proceed
directly from the definition, using the same method we used previously 
to obtain the expectation. But now you need also to calculate the second 
derivative.  Here is a second approach, one that capitalizes on the
fact that separate trials in a Bernoulli series are independent.

Let $\cgF=\{X_1,X_2,\dots,X_n\}$ be a family of random variables
in a probability space $(S,P)$.  We say the family $\cgF$ is
\textit{independent} if for each $i$ and $j$ with 
$1\le i<j\le n$, and for each pair $a,b$ of real numbers with
$0\le a,b\le 1$, the follwing two events are independent:
$\{x\in S: X_i(x)\le a\}$ and $\{x\in S:X_j(x)\le b\}$.  When
the family is independent, it is straightforward to verify that
\[
\var(X_1+X_2+\dots+X_n)=\var(X_1)+\var(X_2)+\dots+\var(X_n).
\]  

With the aid of this observation, the calculation of the variance of
the random variable $X$ which counts the number of successes becomes
a trivial calculation.  But in fact, the entire treatment we have outlined
here is just a small part of a more complex subject which
can be treated more elegantly and ultimately much more compactly---provided
you first develop additional background material on families of
random variables.   For this we will refer
you to suitable probability and statistics texts, such as those given
in our references.
\end{example}

\begin{proposition}\label{prop:altvar}
Let $X$ be a random variable in a probability space $(S,P)$.
Then $\var(X)= E(X^2)-E^2(X)$.
\end{proposition}
\begin{proof}
Let $E(X)=\mu$.  From its definition, we note that
\begin{align*}
 \var(X) &= \sum_{r} (r -\mu)^2\prob(X=r)\\
         &= \sum_{r} (r^2 -2r \mu+\mu^2)\prob(X=r)\\
         &= \sum_r r^2\prob(X=r) - 2 \mu\sum_r r\prob(X=r) +\mu^2\sum_r\prob(X=r)\\
         &= E(X^2) -2\mu^2+\mu^2\\
         &= E(X^2) - \mu^2\\
         &= E(X^2) - E^2(X).
\end{align*}
\end{proof}

Variance (and standard deviation) are quite useful tools in discussions of 
just how likely a random variable is to be near its expected value.
This is reflected in the following theorem, known as Chebychev's
inequality.

\begin{theorem}
Let $X$ be a random variable in a probability space $(S,P)$, and
let $k>0$ be a positive real number.  If the expectation $E(X)$ of
$X$ is $\mu$ and the standard deviation is $\sigma_X$, then
\[
\prob(|X- E(X)|\le k\sigma_X)\ge 1-\frac{1}{k^2}.
\]
\end{theorem}
\begin{proof}
Let $A=\{r\in \mathbb{R}:|r-\mu|>k\sigma_X\}$.

Then we have:

\begin{align*}
\var(X) &= E((X-\mu)^2)\\
        &= \sum_{r\in \mathbb{R}}(r-\mu)^2\prob(X=r)\\
        &\ge \sum_{r\in A}(r-\mu)^2\prob(X=r)\\
        &\ge k^2\sigma_X^2\sum_{r\in A}\prob(X=r)\\
        &\ge k^2\sigma_X^2\prob(|X-\mu|>k\sigma_X).
\end{align*}
Since $\var(X)=\sigma_X^2$, we may now deduce that $1/k^2\geq
\prob(|X-\mu|)>k\sigma_X)$.  Therefore, since $\prob(|X-\mu|\le
k\sigma_X)=1-\prob(|X-\mu|> k\sigma_X)$, we conclude that
\[
\prob(|X- \mu|\le k\sigma_X)\ge 1-\frac{1}{k^2}.
\]
\end{proof}

\begin{example}
Here's an example of how this inequality can be applied.
Consider $n$ tosses of a fair coin with $X$ counting the
number of heads.  As noted before, $\mu=E(X)=n/2$ and
$\var(X)=n/4$, so $\sigma_X=\sqrt{n}/2$.   When $n=10,000$ and
$\mu=5,000$ and $\sigma_X=50$.   Setting $k=50$ so that
$k\sigma_X=2500$, we see that the probability that $X$ is within 
$2500$ of the expected value of $5000$ is at least $0.9996$.
So it seems very unlikely indeed that the number of heads 
is at least $7,500$.

Going back to lottery tickets, if we make the rational
assumption that all ticket numbers are equally likely,
then the probability that the winning number is at least
$7,500$ is exactly $2501/100001$, which is very close to $1/4$.
\end{example}

\begin{example}
In the case of Bernoulli trials, we can use basic properties
of binomial coefficients to make even more accurate 
estimates.  Clearly, in the case of coin tossing,
the probability that the number of heads in $10,000$ tosses
is at least $7,500$ is given by

\[
\sum_{i = 7,500}^{10,000} \binom{10,000}{i}/2^{10,000}
\]
Now a computer algebra system can make this calculation exactly, and
you are encouraged to check it out just to see how truly small this
quantity actually is.
\end{example}

\section{Probability Spaces with Infinitely Many Outcomes}

To this point, we have focused entirely on probability spaces
$(S,P)$ with $S$ a finite set.  More generally, probability
spaces are defined where $S$ is an infinite set.  When $S$ is
countably infinite, we can still define $P$ on the members
of $S$, and now $\sum_{x\in S} P(x)$ is an infinite sum
which converges absolutely (since all terms are non-negative)
to $1$.  When $S$ is uncountable, $P$ is not defined on $S$.
Instead, the probabilty function is defined on a family of
subsets of $S$.  Given our emphasis on finite sets and combinatorics,
we will discuss the first case briefly and refer students to
texts that focus on general concepts from probability and statistics
for the second.

\begin{example}
  Consider the following game.  Nancy rolls a single die.  She wins
  if she rolls a six.  If she rolls any other number, she then rolls
  again and again until the first time that one of the following two
  situations occurs: (1) she rolls a six, which now this results in a
  loss or (2) she rolls the same number as she got on her first roll,
  which results in a win. As an example, here are some sequences of
  rolls that this game might take:
\begin{enumerate}
\item $(4,2,3,5,1,1,1,4)$.  Nancy wins!
\item $(6)$.  Nancy wins!
\item $(5,2, 3,2,1,6)$. Nancy loses. Ouch.
\end{enumerate}
So what is the probability that Nancy will win this game?

Nancy can win with a six on the first roll.  That has probability $1/6$.
Then she might win on round~$n$ where $n\ge2$.  To accomplish this, she
has a $5/6$ chance of rolling a number other than six on the first roll;
a $4/6$ chance of rolling something that avoids a win/loss decision on
each of the rolls, $2$ through $n-1$ and then a $1/6$ chance of rolling
the matching number on round~$n$.
So the probability of a win is given by:
\[
 \frac{1}{6}+\sum_{n\ge 2}\frac{5}{6}\left(\frac{4}{6}\right)^{n-2}\frac{1}{6} = \frac{7}{12}.
\]

\end{example}
\begin{example}
You might think that something slightly more general is lurking
in the background of the preceding example---and it is.  Suppose we
have two disjoint events $A$ and $B$ in a probability space $(S,P)$ and
that $P(A)+P(B)<1$.  Then suppose
we make repeated samples from this space with each sample independent
of all previous ones.  Call it a win if event $A$ holds and a loss if
event $B$ holds.  Otherwise, it's a tie and we sample again.  Now the
probability of a win is:
\[
P(A)+P(A)\sum_{n\ge 1}(1-P(A)-P(B))^n=\frac{P(A)}{P(A)+P(B)}.
\]
\end{example}

\section{Discussion}

Bob was late for morning coffee and the group was
well into dissecting today's applied combinatorics class.
As he approached the table, he blurted out
``Ok guys, here's a problem that doesn't make any sense to
me, except that Nadja, my friend from biology, says that if I 
have a good feel for probability, then it is transparent.'' Alice not very
softly interjected ``Not much goes through six inches
of iron.'' Bob didn't bite ``A guy eats lunch
at the same diner every day.  After lunch, the waiter asks
if he wants dessert.  He asks for the choices and the waiter
replies `We have three kinds of pie: apple, cherry and pecan.'
Then the guy always says `I'll have pecan pie.'  This goes
on for six months.  Then one
day, the waiter says `I have bad news.  Today, we don't
have any apple pie, so your only choices are cherry and pecan.'
Now the guy says `In this case, I'll have the cherry pie.'  I
have to tell you all that this doesn't make any sense to me.
Why would the guy ask for cherry pie in preference to pecan
pie when he consistently takes pecan pie over both cherry
pie and apple pie?''

Zori was the first to say something `Ok guys, I've
finally willing to accept the premise that big integer arithmetic,
and things that reflect the same flavor, might and I emphasize
might, have some relevance in the real world, but this conversation
about dessert in some stupid diner is too much.''  Xing was 
hesitant but still offered ``There's something here. That much 
I'm sure.''  Dave said ``Yeah, a great dessert.  Especially 
the pecan pie.''  Alice was not amused.  All the while 
Carlos was thinking. Finally, he said ``I think it has 
something to do with conditional probability.  The patron's 
preference for pecan pie was conditioned on the
fact that there were three choices.  When there were only two
choices, his preferences changed.''

Now Yolanda saw more ``Doesn't this happen all the time in
presidential politics?  People prefer candidate $A$ when
$A$, $B$ and $C$ are running, but when candidate $C$ drops out,
they shift their preference to candidate $B$.''
Alice said ``You could say the same thing about close
personal relationships.''  Although she didn't say it,
she was thinking that it wouldn't matter how many dropped out
if Bob was one of the remaining.


\section{Exercises}\label{s:probability:exercises}
\begin{enumerate}
\item Our gang of seven (Alice, Bob, Carlos, Dave, Xing, Yolanda and Zori) are
students in a class with a total enrollment of 35.  The professor
chooses three students at random to go to the board to work
challenge problems.  
\begin{enumerate}
\item What is the probability that Yolanda is chosen?
\item What is the probability that Yolanda is chosen and Zori is not? 
\item What is the probability that exactly two members of the club are chosen?
\item What is the probability that none of the seven members of club are chosen?
\end{enumerate}
\item Bob says to no one in particular, ``Did you know that the
  probability that you will get at least one ``7'' in three rolls of a
  pair of dice is slightly less than $1/2$.  On the other hand, the
  probability that you'll get at least one ``5'' in six rolls of the
  dice is just over $1/2$.''  Is Bob on target, or out to lunch?
\item Consider the spinner shown in \autoref{fig:spinner} at the
  beginning of the chapter.
\begin{enumerate}
\item What is the probability of getting at least one ``5'' in three spins?
\item What is the probability of getting at least one ``3'' in three spins?
\item If you keep spinning until you get either a ``2'' or a ``5'', what is
the probability that you get a ``2'' first?
\item If you receive $i$ dollars when the spinner halts in region~$i$,
what is the expected value?  Since three is right in the middle of the 
possible outcomes, is it reasonable to pay three dollars to play this game?
\end{enumerate}

\item  Alice proposes to Bob the following game.  Bob pays one dollar to play.
Fifty balls marked $1,2,\dots,50$ are placed in a big jar, stirred around,
and then drawn out one by one by Zori, who is wearing a blindfold.  The result
is a random permutation $\sigma$ of the integers $1$, $2,\dots,50$.  Bob wins
with a payout of two dollars and fifty cents if the permutation $\sigma$ is 
a derangement, i.e., $\sigma(i)\neq i$ for all $i=1,2,\dots,n$. Is this a
fair game for Bob?  If not how should the payoff be adjusted to make it fair?

\item A random graph with vertex set $\{1,2,\dots,10\}$ is constructed by
the following method.  For each two element subset $\{i,j\}$ from $\{1,2,\dots,10\}$,
a fair coin is tossed and the edge $\{i,j\}$ then belongs to the graph when
the result is ``heads.''  For each $3$-element subset $S\subseteq\{1,2,\dots,n\}$,
let $E_S$ be the event that $S$ is a complete subgraph in our random graph.
\begin{enumerate}  
\item Explain why $P(E_S)= 1/8$ for each $3$-element subset $S$.
\item Explain why $E_S$ and $E_T$ are independent when $|S\cap T|\le 1$.
\item Let $S=\{1,2,3\}$, $T=\{2,3,4\}$ and $U=\{3,4,5\}$.  Show that
$P(E_S|E_T)\neq P(E_S|E_TE_U)$.
\end{enumerate}

\item Ten marbles labeled $1,2,\dots,10$ are placed in a big jar and
  then stirred up.  Zori, wearing a blindfold, pulls them out of the
  jar two at a time.  Players are allowed to place bets as to whether
  the sum of the two marbles in a pair is $11$.  There are $C(10,2)=
  45$ different pairs and exactly $5$ of these pairs sums to eleven.

  Suppose Zori draws out a pair; the results are observed; then
  she returns the two balls to the jar and all ten balls are stirred
  before the next sample is taken.  Since the probability that the sum
  is an ``11'' is $5/45=1/9$, then it would be fair to pay one dollar
  to play the game if the payoff for an ``11'' is nine dollars.
  Similarly, the payoff for a wager of one hundred dollars should be
  nine hundred dollars.

  Now consider an alternative way to play the game.  Now Zori
  draws out a pair; the results are observed; and the marbles are set
  aside.  Next, she draws another pair from the remaining eight
  marbles, followed by a pair selected from the remaining six, etc.
  Finally, the fifth pair is just the pair that remains after the
  fourth pair has been selected.  Now players may be free to wager on
  the outcome of any or all or just some of the five rounds.  Explain
  why either everyone should or no one should wager on the fifth round.
  Accordingly, the last round is skipped and all marbles are returned
  to the jar and we start over again.

  Also explain why an observant player can make lots of money with a payout
  ratio of nine to one.  Now for a more challenging problem, what is
  the minimum payout ratio above which a player has a winning
  strategy?
\end{enumerate}

% \end{enumerate}
%%% Local Variables: 
%%% mode: latex
%%% TeX-master: "chap-skel-mtk"
%%% End: 
