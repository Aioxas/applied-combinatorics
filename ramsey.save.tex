\chapter{Ramsey Theory}

% %               Abstract and Maketitle

% \begin{abstract}
% Ramsey's theorem asserts that if the $k$-element subsets
% of a large set are colored with a small number of colors,
% then there is a homoegeneous (monochromatic) set $H$ all of
% whose $k$-element subsets are assigned the same color.
% This important generalization of the Pigeon Hole Principle
% implies that complete disorder is impossible.  Instead,
% buried inside large systems are substems with complete
% uniformity.
% \end{abstract}
% \maketitle

\section{Basic Notation and Terminology}\label{s:ramsey:intro}

For a positive integer $n$, we let $[n]=\{1,2,\dots,n\}$.
For a set $X$ and a non-negative integer $k$, $\binom{X}{k}$
denotes the set of all $k$-element subsets of $X$.  In Ramsey
theoretic settings, it is common to refer to a map
$\phi:\binom{X}{k}:\longrightarrow R$ as a \textit{coloring} of
the $k$-element subsets of $X$, and the elements of $R$ will
be referred to as \textit{colors}.  Typically, we will just take
$R=[r]$ for some positive integer $r$, so $\phi$ will also
be called an $r$-coloring.

When $\phi:\binom{X}{k}\longrightarrow [r]$ is a $r$-coloring, a 
subset $H\subseteq S$ is called a \textit{homogeneous} set 
(also a \textit{monochromatic} set) when there exists a color 
$\alpha\in[r]$ so that $\phi(A)=\alpha$ for every $A\in\binom{H}{k}$.

\begin{theorem}[Ramsey's theorem]
If $k$ and $r$ are positive integers, and $(h_1,h_2,\dots,h_r)$
are integers with $h_i\ge k$ for $i=1,2,\dots,r$, then
there exists a least positive integer $t_0=R(k:h_1,h_2,\dots,h_r)$
so that if $X$ is any set with $|X|\ge t_0$,
then for every $r$-coloring $\phi:\binom{S}{k}\longrightarrow [r]$
of the $k$-element subssets of $X$,
there exists an $\alpha\in X$ and a subset $H\subseteq X$ with
$|H|\ge h_i$ so that $\phi(A) = \alpha$ for every $A\in
\binom{H}{k}$.
\end{theorem}

\begin{proof}
We use a double induction.  The first induction is on $k$,
and the second is on $r$.  When $k=1$, the result holds for
all $r$ as the theorem reduces to a restatement of the Pigeonhole
Principle.  As an aside, we note that
\[
R(1:h_1,h_2,\dots,h_r)= 1+\sum_{i=1}^r h_i-1.
\]
However, in general, we will not be able to say much about
the exact value of ramsey numbers.  Instead, the emphasis is
on the fact that they \textit{exist}!

Now assume validity for some $k\ge1$ and consider the next value
of $k$. Now the induction is on $r$.  When $r=1$, it is easy
to see that $R(k:h_1)=h_1$.  Now consider the case $r=2$.

Let $q=R(1;h_1,h_2)$ and define a sequence of numbers 
$(s_0,s_1,s_2,\dots,s_{q}$ as follows.  First set $s_0=k-1$.  
If $s_i$ has been defined, and $1\le i<q$, set $s_{i+1}=1+R(k-1,s_i,s_i)$.  
Note that $s_1=1+R(k-1;k-1,k-1) = 1 + (k-1)=k$.
We show that $R(k:h_1,h_2)$ exists and it at most $s_q$.  

Now let $X$ be any set with $|X|\ge s_q$ and let $\phi:
\binom{X}{k}\longrightarrow [2]$ be a $2$-coloring of the
$k$-element subsets of $X$. Without loss of generality, we
may assume that $X$ is a set of positive integers.
Let $x_1$ be the least integer in $X$ and set $X_1=X-\{x_1\}$.
Then $\phi$ determines a coloring $\phi_1$ of the $k-1$-element subsets of 
$X_1$ by setting $\phi_1(B)=\phi(B\cup\{x_1\}$.
It follows that there is some $\alpha_1\in X$ and a subset
$H_1\subset X_1$ with $|H_1|=s_{q-1}$ so that $\phi_1(B)=\alpha_1$ for
every $B\in\binom{X_1}{k-1}$.  The let $x_2$ be the least integer
in $H_1$ and set $X_2=H_1-\{x_2\}$.  As before, $\phi$ determines
a coloring $\phi_2$ of the $k-1$ element subsets of $X_2$ by
setting $\phi_2(B)=\phi(B\cup\{x_2\}$.  Again, there is some
$\alpha_2\in X$ and a subset $H_2\subset X_2$ so
that $\phi_2(B)=\alpha_2$ for every $B\in\binom{H_2}{k-1}$. Note
that $\alpha_2$ may in fact be the same as $\alpha_1$.

Repeat this process $q$ times and consider the 
set $\{x_1,x_2,\dots,x_q\}$ obtained as an end result.
Also consider the elements $\{\alpha_i:1\le i\le q\}$.  
It follows from the Pigeon-hole Principle that there
is an element $\alpha\in\{1,2,\dots,r\}$ and subset  
$S\subseteq \{1,2,\dots,q\}$ with $|S|=h_\alpha$ so
that $\alpha_i=\alpha$ for every $i\in S$.  Then
set $H=\{x_i:i\in S\}$.  Then $|H|=h_\alpha$.  Furthermore
$\phi(A)=\alpha$ for every $k$-element subset of $H$.
This completes the argument when $r=2$.

Now suppose that $r>2$.  We derive a $2$-coloring from an
$r$-coloring by considering the last $r-1$ colors collectively
as just one color.  Now  it follows easily that
the ramsey number $R(k;h_1,h_2,\dots,h_r)$ exists and
satisfies:

\[
R(k;h_1,h_2,\dots,h_r)\le R(k;h_1,R(k;h_2,h_3,\dots,h_r))
\]
\end{proof}

\begin{corollary}
$R(2;m,n)\le \binom{m+n-2}{m-1}=\binom{m+n-2}{n-1}$.
\end{corollary}

\begin{proof}
The argument when $r=2$ gives the inequality
\[
R(2;m,n)\le R(2;m-1,n)+R(2;m,n-1).
\]
However, the binomial coefficients satisfy
\[
\binom{m+n-2}{m-1}=\binom{m+n-3}{m-2}+\binom{m+n-3}{m-1} =
\binom{m+n-3}{m-2}+\binom{m+n-3}{n-2}
\]
So the desired inequality follows by induction after noting
that it holds for $m=1$ and for $n=1$.
\end{proof} 

\section{Small Ramsey Numbers}\label{s:size}

In the following table, we provide information about
the ramsey numbers $R(2;m,n)$ when $m$ and $n$ are at
most $7$.   When a cell contains a single number, that
is the precise answer.  When there are two numbers, they
represent upper and lower bounds.

\begin{center}
\begin{tabular}{||cc||c|c|c|c|c|c|c||}
\hline\hline
&n&3&4&5&6&7&8&9\\
m&&&&&&&&\\
\hline\hline
3&&6&9&14&18&23&36&39\\
4&&&18&25&35, 41&49, 61&56, 84&69, 115\\
5&&&&43, 49&58, 87&80, 143&95, 216&121, 316\\
6&&&&&102, 165&111, 298&127, 495&153, 780\\
7&&&&&&205, 540&216, 1031&216, 1713\\
8&&&&&&&282, 1870&282, 3583\\
9&&&&&&&&565, 6588\\
\hline\hline
\end{tabular}
\end{center}

For additional data, do a web search and look for
Stanley Radziszowski, who maintains the most current information
on his web site.

\section{Estimating Ramsey Numbers}\label{s:estimatesize}

We will find it convenient to utilize the following approximation
due to Stirling.  You can find a proof in almost any
advanced calculus book.

\[
n!\equiv \sqrt{2\pi n} \bigl( \frac{n}{e}\bigr)^n\bigl(1+
  \frac{1}{12n}+\frac{1}{288n^2}-\frac{139}{51840n^3} +O(\frac{1}{n^4})\bigr).
\]
Of course, we will normally be satisfied with the
first term:
\[
n!\equiv \sqrt{2\pi n} \bigl( \frac{n}{e}\bigr)^n
\]
Using Stirling's approximation, we have the following
upper bound:
\[
R(2;n,n) \le \binom{2n-2}{n-1} \equiv \frac{2^{2n}}{4\sqrt{\pi n}} 
\]
Here is an exponential lower bound.

\begin{theorem}
\[
R(2;n,n) \ge \bigl(1+o(1)\bigr) \frac{n}{e\sqrt2} 2^{\frac{1}{2}n}
\]
\end{theorem}

\begin{proof}
Let $t$ be an integer with $t>n$ and consider the following
probability space.  The outcomes in the probability space
are graphs with vertex set $\{1,2,\dots,t\}$.  For each
$i$ and $j$ with $1\le i < j\le t$, edge $ij$ is present
in the graph with probability $1/2$.  Furthermore, the
events for distinct pairs are independent.

Let $X_1$ denote the random variable which counts the number
of $n$-element subsets of $\{1,2,\dots,t\}$ for which all
$\binom{n}{2}$ pairs are edges in the graph.  Similarly,
$X_2$ is the random variable which counts the number
of $n$-element subsets of $\{1,2,\dots,t\}$ for which all
$\binom{n}{2}$ pairs are edges are \textit{not} in the graph.  
Then set $X=X_1+X_2$.

By linearity of expectation, $E(X)=E(X_1)+E(X_2)$ while

\[
E(X_1)=E(X_2) = \binom{t}{n} \frac{1}{2^{\binom{n}{2}}}.
\]

If $E(X)<1$, then there must exist a graph with vertex
set $\{1,2,\dots,t\}$ without a $K_n$ or an $I_n$.
We then consider the inequality
\[
2\binom{t}{n}\frac{1}{2^{\binom{n}{2}}} < 1
\]
Using the Stirling approximation, we see that this
inequality holds when
\[
t \ge \bigl(1+o(1)\bigr) \frac{n}{e\sqrt2} 2^{\frac{1}{2}n}
\]
\end{proof}

%%% Local Variables: 
%%% mode: latex
%%% TeX-master: "book"
%%% End: 
