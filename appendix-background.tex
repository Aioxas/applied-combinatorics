% appendix-background.tex
% Updated January 11, 2012

\chapter{Background Material for Combinatorics}\label{app:background}

This appendix treats background material essential to the
study of combinatorial mathematics.  Many students will find that
most---and perhaps all---of this material has been covered somewhere
in their prior course work, and we expect that very few instructors
will include this appendix in the syllabus.  Nevertheless, students
may find it convenient to consult this appendix from time to
time, and we suspect that many instructors will encourage
students to read this material to refresh their memories of key
concepts.  

\section{Introduction}

Set theory is concerned with \textit{elements}, certain collections
of elements called \textit{sets} and a concept of \textit{membership}.
For each element $x$ and each set $X$, \textit{exactly} one of the 
following two statements holds:

\begin{enumerate}
\item $x$ is a member of $X$. 
\item $x$ is \textit{not} a member of $X$. 
\end{enumerate}

It is important to note that membership cannot be ambiguous.

When $x$ is an element and $X$ is a set, we write $x\in X$ when
$x$ is a member of $X$.  Also, the statement $x$ belongs to $X$ means
exactly the same thing as $x$ is a member of $X$.  Similarly, when
$x$ is not a member of $X$, we write $x\notin X$ and say $x$ does
not belong to $X$.

Certain sets will be defined explicitly by listing the elements.
For example, let $X=\{a,b,d,g,m\}$.  Then $b\in X$ and $h\notin X$.
The order of elements in such a listing is irrelevant, so we could
also write $X=\{g,d,b,m,a\}$.  In other situations, sets will be 
defined by giving a rule for membership.
As examples, let $\posints$ denote the set of positive integers.  
Then let $X=\{n\in\posints:5\le n\le 9\}$.   Note that $6,8\in X$ while
$4,10,238\notin X$.

Given an element $x$ and a set $X$, it may at times be tedious and
perhaps very difficult to determine which of the statements
$x\in X$ and $x\notin X$ holds.  But if we are discussing sets,
it must be the case that \textit{exactly} one is true.

\begin{example}
Let $X$ be the set consisting of the following $12$ positive
integers:

\begin{align*}
&13232112332\\
&13332112332\\
&13231112132\\
&13331112132\\
&13232112112\\
&13231112212\\
&13331112212\\
&13232112331\\
&13231112131\\
&13331112131\\
&13331112132\\
&13332112111\\
&13231112131
\end{align*}
Note that one number is listed twice.  Which one is it?
Also, does $13232112132$ belong to $X$?  Note that the
apparent difficulty of answering these questions stems from
(1) the size of the set $X$; and (2) the size of the integers 
that belong to $X$.  Can you think of circumtances in which
it is difficult to answer whether $x$ is a member of $X$
even when it is known that $X$ contains exactly one element?
\end{example}

\begin{example}
Let $P$ denote the set of primes.  Then $35\notin P$ since
$35= 5\times 7$.  Also, $19\in P$.  Now consider the number
\[
n = 77788467064627123923601532364763319082817131766346039653933
\]
Does $n$ belong to $P$?  Alice says yes while Bob says no.
How could Alice justify her affirmative answer?  How could Bob
justify his negative stance?  In this specific case, I know that
Alice is right.  Can you explain why?
\end{example}

\section{Intersections and Unions}

When $X$ and $Y$ are sets, the \textit{intersection}
of $X$ and $Y$, denoted $X\cap Y$, is defined by
\[
X\cap Y = \{x: x\in X, x\in Y\}
\]
Note that this notation uses the convention followed by
many programming languages. Namely, the ``comma'' in the
definition means that \textit{both} requirements for
membership be satisfied.  For example, if $X=\{b,c,e,g,m\}$
and $Y=\{a,c,d,h,m,n,p\}$, then $X\cap Y=\{c,m\}$.

\subsection{The Meaning of $2$-Letter Words}

In the not too distant past, there was considerable discussion
in the popular press on the meaning of the $2$-letter word \textit{is}.
For mathematicians and computer scientists, it would have
been far more significant to have a discussion of the $2$-letter
word \textit{or}.  The problem is that the English language uses
\textit{or} in two fundamentally different ways. Consider the
following sentences:

\begin{enumerate}
\item  A nearby restaurant has a dinner special featuring two choices for
dessert: flan de casa \textbf{or} tirami-su.
\item  A state university accepts all students who have graduated
from in-state high schools and have SAT scores above $1000$ \textbf{or} 
have grade point averages above $3.0$.
\item A local newpaper offers 
customers the option of paying their for their newspaper bills on a monthly
\textbf{or} semi-annual basis.
\end{enumerate}
In the first and third statement, it is clear that there are
two options but that \textit{only} one of them is allowed.  However,
in the second statement, the interpretation is that admission will
be granted to students who satisfy \textit{at least one} of the
two requirements.  These interpretations are called 
respectively the \textit{exclusive}
and \textit{inclusive} versions of \textbf{or}.  In this class, we
will assume that whenever the word ``or'' is used, 
the inclusive interpretation is intended---unless otherwise
stated.

For example, when $X$ and $Y$ are sets, the \textit{union}
of $X$ and $Y$, denoted $X\cup Y$, is defined by
\[
X\cup Y = \{x: x\in x \text{ or } x\in Y\}.
\]
membership be satisfied.  For example, if $X=\{b,c,e,g,m\}$
and $Y=\{a,c,d,h,m,n,p\}$, then $X\cup Y=\{a,b,c,d,e,g,h,m,n,p\}$.

Note that $\cap$ and $\cup$ are \textit{commutative} and
\textit{associative} binary operations, as is the case with
addition and multiplication for the set $\posints$ of positive
integers, i.e., if $X$, $Y$ and $Z$ are sets, then
\[
X\cap Y = Y\cap X \quad\text{ and }\quad X\cup Y = Y\cup X. 
\]
Also,
\[
X\cap(Y\cap Z)= (X\cap Y)\cap Z\quad\text{ and }\quad
X\cup(Y\cup Z)= (X\cup Y)\cup Z.
\]
Also, note that each of $\cap$ and $\cup$ distributes
over the other, i.e., 
\[
X\cap(Y\cup Z)= (X\cap Y)\cup (X\cap Z)\quad\text{ and }\quad
X\cup(Y\cap Z)= (X\cup Y)\cap (X\cup Z)
\]
On the other hand, in $\posints$, multiplication distributes
over addition but not vice-versa.

\subsection{The Empty Set: Much To Do About Nothing}

The \textit{empty set}, denoted $\emptyset$ is the set for
which $x\notin \emptyset$ for every element $x$.  Note that
$X\cap \emptyset =\emptyset$ and $X\cup \emptyset=X$, for every
set $X$. 

The empty set is unique in the sense that if $x\notin X$ for every
element $x$, then $X=\emptyset$.

\subsection{The First So Many Positive Integers}

In our course, we will use the symbols $\posints$, $\ints$,
$\rats$ and $\reals$ to denote respectively the
set of positive integers, the set of all integers (positive, negative
and zero), the set of rational numbers (fractions) and the set of real
numbers (rationals and irrationals).  On occasion, we will discuss
the set $\nonnegints$ of \textit{non-negative integers}.
When $n$ is a positive integer, we will use the 
abbreviation $[n]$ for the set $\{1,2,\dots,n\}$ of the
first $n$ positive integers.  For example, $[5]=\{1,2,3,4,5\}$.
For reasons that may not be clear at the moment but hopefully
will be transparent later in the semester, we use the
notation $\bfn$ to denote the $n$-element set $\{0,1,2,\dots,n-1\}$.
Of course, $\bfn$ is just the set of the first $n$ non-negative
integers.  For example, $\mathbf{5}=\{0,1,2,3,4\}$.

\subsection{Subsets, Proper Subsets and Equal Sets} 

When $X$ and $Y$ are sets, we say $X$ is a \textit{subset} of $Y$ and
write $X\subseteq Y$ when $x\in Y$ for every $x\in X$.  When $X$ is a
subset of $Y$ and there exists at least one element $y\in Y$ with
$y\notin X$, we say $X$ is a proper subset of $Y$ and write
$X\subsetneq Y$.  For example, the $P$ of primes is a proper subset of
the set $\posints$ of positive integers.

Surprisingly often, we will encounter a situation where
sets $X$ and $Y$ have different rules for membership
yet both are in fact the same set.  For example, let
$X=\{0,2\}$ and $Y=\{z\in\ints: z+z=z\times z\}$. Then $X=Y$.
For this reason, it is useful to have a test when sets are
equal.  If $X$ and $Y$ are sets, then

\[
X = Y \quad\text{ if and only if }\quad X\subseteq Y \text{ and } Y\subseteq X.
\]  

\section{Cartesian Products} 

When $X$ and $Y$ are sets, the \textit{cartesian product}
of $X$ and $Y$, denoted $X\times Y$, is defined by
\[
X\times Y=\{(x,y): x\in X \text{ and } y\in Y\}
\]
For example, if $X=\{a,b\}$ and $Y=[3]$, then
$X\times Y=\{(a,1),(b,1),(a,2),(b,2),(a,3),(b,3)\}$.
Elements of $X\times Y$ are called \textit{ordered pairs}.
When $p=(x,y)$ is an ordered pair, the element $x$ is referred
to as the \textit{first coordinate} of $p$ while $y$ is the
\textit{second coordinate} of $p$.
Note that if either $X$ or $Y$ is the empty set, then
$X\times Y=\emptyset$.

\begin{example}
Let $X=\{\emptyset,(1,0),\{\emptyset\}\}$ and $Y=\{(\emptyset,0)\}$.
Is $((1,0),\emptyset)$ a member of $X\times Y$?
\end{example}

Cartesian products can be defined for more than two factors.
When $n\ge 2$ is a positive integer and $X_1,X_2,\dots,X_n$ are
non-empty sets, their \textit{cartesian product} is defined
by
\[
X_1\times X_2\times\dots\times X_n=\{(x_1,x_2,\dots,x_n): x_i\in X_i
\text{ for } i = 1,2,\dots,n\}
\]


\section{Binary Relations and Functions}

A subset $R\subseteq X\times Y$ is called a \textit{binary
relation} on $X\times Y$, and a binary relation $R$ on $X\times Y$ 
is called a \textit{function from} $X$ \textit{to} $Y$ when the
following condition is satisfied:

\medskip
\noindent
$C$:\quad For every $x\in X$, there is \textit{exactly one}
element $y\in Y$ for which $(x,y)\in R$.

\medskip
Many authors prefer to write Condition~\textbf{C} in two parts:

\medskip
\noindent
$C_1$:\quad For every $x\in X$, there is \textit{some}
element $y\in Y$ for which $(x,y)\in R$.

\medskip
\noindent
$C_2$:\quad For every $x\in X$, there is \textit{at most one}
element $y\in Y$ for which $(x,y)\in R$.

\medskip
And this last condition is often stated in the following alternative
form:

\medskip
\noindent
$C'_2$: \quad If $x\in X$, $y_1,y_2\in Y$ and $(x,y_1),(x,y_2)\in R$, then
$y_1=y_2$.

\begin{example}\label{exa:function}

For example, let $X=[4]$ and $Y=[5]$.  Then let

$R_1=\{(2,1),(4,2),(1,1),(3,1)\}$, 

$R_2=\{(4,2),(1,5),(3,2)\}$, and 

$R_3=\{(3,2),(1,4),(2,2),(1,1),(4,5)\}$.

\noindent
Then only $R_1$ is a function from $X$ to $Y$.
\end{example}

In many settings (like calculus), it is customary to use letters like $f$,
$g$ and $h$ to denote functions.  So let $f$ be a function from a
set $X$ to a set $Y$. In view of the defining properties of functions, for
each $x\in X$, there is a unique element $y\in Y$ with $(x,y)\in f$.
And in this case, the convention is to write $y=f(x)$.  For example,
if $f=R_1$ is the function in Example~\ref{exa:function}, then
$2=f(4)$ and $f(3) =1$.

The shorthand notation $f:X\rightarrow Y$ is used to indicate
that $f$ is a function from the set $X$ to the set $Y$.  

In calculus, we study functions defined by algebraic rules.
For example, consider the function $f$ whose rule is $f(x) = 5x^3-8x+7$.
This short hand notation means that $X=Y=\reals$ and that
\[
f=\{(x,5x^3-8x+7):x\in\reals\}
\]
In combinatorics, we sometimes study functions defined
algebraically, just like in calculus, but we will frequently
describe functions by other kinds of rules.  For example, let
$f:\posints\rightarrow\posints$ be defined by
$f(n) = |n/2|$ if $n$ is even and $f(n)=3|n|+1$ when $n$ is odd.

A function $f:X\rightarrow Y$ is
called an \textit{injection from} $X$ \textit{to} $Y$ when

\medskip
\noindent
$I$:\quad For every $y\in Y$, there is \textit{at most} one element 
$x\in X$ with $y=f(x)$.

\medskip
When the meaning of $X$ and $Y$ is clear, we just say $f$ is an 
\textit{injection}.  An injection is also called a $1$--$1$ function
(read this as ``one to one'') and this is sometimes denoted as
$f:X\injection Y$.

A function $f:X\rightarrow Y$ is called a \textit{surjection from}
$X$ \textit{to} $Y$ when:

\medskip
\noindent
$S$:\quad For every $y\in Y$, there is \textit{at least} one $x\in X$ with
$y=f(x)$.

\medskip
Again, when the meaning of $X$ and $Y$ is clear, we just say $f$ is an 
\textit{surjection}.  A surjection is also called an \textit{onto} 
function and this is sometimes denoted as\\
$f:X\surjection Y$.

A function $f$ from $X$ to $Y$ which is both an injection and a surjection
is called a \textit{bijection}.  Alternatively, a bijection is referred
to as a $1$--$1$, onto function, and this is sometimes denoted as
$f:X \bijection Y$.  A bijection is also called a 
$1$--$1$-\textit{correspondence}.

\begin{example}
Let $X=Y=\reals$.  Then let $f$, $g$ and $h$ be the
functions defined by

\begin{enumerate}
\item $f(x)=3x-7$.
\item $g(x)=3(x-2)(x+5)(x-7)$.
\item $h(x)=6x^2-5x+13$.
\end{enumerate}
Then $f$ is a bijection; $g$ is a surjection but not an
injection (\textit{Why}?); and $h$ is neither an injection nor a
surjection (\textit{Why}?).
\end{example}

\begin{proposition}
Let $X$ and $Y$ be sets. Then there is a bijection from $X$ to $Y$
if and only if there is a bijection from $Y$ to $X$.
\end{proposition}

\section{Finite Sets}

A set $X$ is said to be \textit{finite} when either (1) $X=\emptyset$; or
(2) there exists positive integer $n$ and a bijection $f:[n]\bijection X$.
When $X$ is not finite, it is called \textit{infinite}.  For example,
$\{a,\emptyset,(3,2),\posints\}$ is a finite set as is
$\posints\times\emptyset$.  On the other hand, $\posints\times
\{\emptyset\}$ is infinite.  Of course, $[n]$ and $\bfn$ are
finite sets for every $n\in\posints$.

\begin{proposition}
Let $X$ be a non-empty finite set.  Then there is a unique
positive integer $n$ for which there is a bijection $f:[n]\bijection X$.
\end{proposition}

In some cases, it may take some effort to determine whether a set
is finite or infinite.  Here is a truly
classic result.

\begin{proposition}
The set $P$ of primes is infinite.
\end{proposition}
\begin{proof}
Suppose that the set $P$ of primes is finite.  It is non-empty
since $2\in P$.  Let $n$ be the unique positive integer for
which there exists a bijection $f:[n]\rightarrow P$.  Then let 

\[
p=1+f(1)\times f(2)\times f(3)\times \dots\times f(n)
\]
Then $p$ is prime (\textit{Why}?) yet larger than any element of $P$.
The contradiction completes the proof.
\end{proof}

Here's a famous example of a set where no one knows if
the set is finite or not.

\begin{conjecture}  It is conjectured that the following
set is infinite:
\[
T=\{n\in\posints:n \text{ and } n+2 \text{ are both
primes }\}. 
\]
\end{conjecture}
This conjecture is known as the \textit{Twin Primes Conjecture}.
Guaranteed $\text{A}++$ for any student who can settle it!
 
\begin{proposition}\label{exe:sb}
Let $X$ and $Y$ be finite sets.  If there exists an injection
$f:X\injection Y$ and an injection $g:Y \injection X$, then
there exists a bijection $h:X \bijection Y$.
\end{proposition}

When $X$ is a finite non-empty set, the \textit{cardinality} of $X$,
denoted $|X|$ is the unique positive integer $n$ for which
there is a bijection $f:[n]\bijection X$.  Intuitively,
$|X|$ is the number of elements in $X$.  For example,
\[
|\{(6,2), (8,(4,\emptyset)), \{3,\{5\}\}\}|=3.
\]
By convention, 
the cardinality of the empty set is taken to be zero, and we write
$|\emptyset|=0$.

\begin{proposition}\label{prop:xy} 
If $X$ and $Y$ are finite non-empty sets, then
$|X\times Y| =|X|\times |Y|$.
\end{proposition}

\begin{remark}
The statement in the last exercise is an example of ``operator
overloading'', a technique featured in several programming
languages.  Specifically, the times sign $\times$ is used
twice but has different meanings.  As part of $X\times Y$, it
denotes the cartesian product, while as part of $|X|\times |Y|$,
it means ordinary multiplication of positive integers.  Programming
languages can keep track of the data types of variables and
apply the correct interpretation of an operator like $\times$
depending on the variables to which it is applied.
\end{remark}

We also have the following general form of Proposition~\ref{prop:xy}:
\[
|X_1\times X_2\times\dots\times X_n|=
|X_1|\times |X_2|\times\dots\times |X_n|
\]

\begin{theorem}\hfill\mbox{}\\
\begin{enumerate}
\item There is a bijection between any two of the following
infinite sets $\posints$, $\ints$ and $\rats$. 
\item There is an injection from $\rats$ to $\reals$.
\item There is no surjection from $\rats$ to $\reals$.
\end{enumerate}
\end{theorem}

\section{Notation from Set Theory and Logic}

In set theory, it is common to deal with statements 
involving one or more elements from the universe as
variables.  Here are some examples:
\begin{enumerate}
\item For $n\in\posints$, $n^2-6n+8=0$.
\item For $A\subseteq[100]$, $\{2,8,25,58,99\}\subseteq A$.
\item For $n\in \ints$, $|n|$ is even.
\item For $x\in \reals$, $1+1=2$.
\item For $m,n\in \posints$, $m(m+1)+2n$ is even.
\item For $n\in \posints$, $2n+1$ is even.
\item For $n\in \posints$ and $x\in\reals$, $n+x$ is
irrational.
\end{enumerate}
These statements may be true for some values of the variables
and false for others.  The fourth and fifth statements
are true for \textit{all} values of the variables, while
the sixth is false for all values.

Implications are frequently abbreviated using with
a double arrow $\Longrightarrow$; the quantifier $\forall$ means ``for all'' 
(or ``for every''); and the quantifier $\exists$ means 
``there exists'' (or ``there is'').  Some
writers use $\wedge$ and $\vee$ for logical ``and'' and
``or'', respectively.  For example, 
\[
\forall A,B\subseteq[4]\quad \bigl((1,2\in A) \wedge |B|\ge 3)\bigr)
\Longrightarrow\bigl((A\subseteq B)\vee (\exists n\in A\cup B, 
 n^2=16)\bigr)
\]
The double arrow $\iff$ is used to denote logical equivalence
of statements (also ``if and only if'').  For example
\[
\forall A\subseteq[7]\quad A\cap\{1,3,6\}\neq\emptyset\iff
A\nsubseteq\{2,4,5,7\}
\]
We will use these notational shortcuts \textit{except} for
the use of $\wedge$ and $\vee$, as we will use these two symbols
in another context: binary operators in lattices. 

\section{Formal Development of Number Systems}

Up to this point, we have been discussing number systems
in an entirely informal manner, assuming everyone knew
all that needed to be known.  Now let's pause and put
things on a more firm foundation.  So 
for the time being, do a memory dump and forget everything you have ever
learned about numbers and arithmetic. The set
of natural numbers has just been delivered on our
door step in a big box with a warning label saying ``Assembly
Required.''  We open the box and find a single piece of paper
on which the following ``instructions'' are printed.  These
defining properties of the natural numbers are known as the \textit{Peano 
Postulates}.

\begin{enumerate}
\item[(i).] There is a non-empty set of elements called 
\textit{natural numbers}.  There is natural number
called \textit{zero} which is denoted~$0$.  The set of all natural
numbers is denoted $\mathbb{N}_0$
\item[(ii).] There is a one-to-one function $s:\mathbb{N}_0
\injection \mathbb{N}_0$
called the \textit{successor} function.  For each $n\in \mathbb{N}_0$,
$s(n)$ is called the \textit{successor} of $n$.
\item[(iii).] There is no natural number $n$ for which $0=s(n)$.
\item[(iv).] Let $\mathbb{M}\subseteq \mathbb{N}_0$.  
Then $\mathbb{M} =\mathbb{N}_0$ if and only if
\begin{enumerate}
\item[(a).] $0\in \mathbb{M}$; and 
\item[(b).] $\forall k\in \mathbb{N}_0\quad(k\in \mathbb{M}) 
  \Longrightarrow(s(k)\in \mathbb{M})$. 
\end{enumerate}
\end{enumerate}

Property~(iv) in the list of Peano Postulates is called 
the \textit{Principle of Mathematical Induction}, or
just the \textit{Principle of Induction}.
As a first application of the Principle of Induction, we prove the following
basic property of the natural numbers.

\begin{proposition}
Let $n$ be a natural number with $n\neq0$.  Then there is a
natural number $m$ so that $n=s(m)$.
\end{proposition}
\begin{proof}
Let $\mathbb{S}=\{n\in\mathbb{N}_0:\exists m\in\mathbb{N}_0,
n=s(m)\}$.  Then set $\mathbb{M}=\{0\}\cup\mathbb{S}$.
We show that $\mathbb{M}=\mathbb{N}_0$.  First, note that
$0\in\mathbb{M}$.  Next, we will show that for all $k\in\mathbb{N}_0$,
if $k\in\mathbb{M}$, then $s(k)\in\mathbb{M}$. However, this is trivial
since for all $k\in\mathbb{N}_0$, we have 
$s(k)\in\mathbb{S}\subseteq\mathbb{M}$.
We conclude that $\mathbb{M}=\mathbb{N}_0$.
\end{proof}

\subsection{Addition as a Binary Operation}\label{s:add}

Recall that a \textit{binary operator} $*$ on set $X$ is just
a function $*:X\times X\rightarrow X$.  So the image of
the ordered pair $(x,y)$ would normally be denoted $*((x,y))$.
However, this is usually abbreviated as $*(x,y)$ or
even more compactly as $x*y$.  With this convention,
we now define a binary operation $+$ 
on the the set $\mathbb{N}_0$ of natural numbers.
This operation is defined by:

For every natural number $n\in\mathbb{N}_0$:
\begin{enumerate}
\item [(i).]  $n+0=n$.
\item [(ii).] For all $k\in\mathbb{N}_0$, $n+s(k) = s(n+k)$.
\end{enumerate}

We pause to make it clear why the preceding two statements
define $+$.  Let $n$ be
an arbitrary natural number.  Then let $\mathbb{M}$ denote the
set of all natural numbers $m$ for which $n+m$ is
defined.  Note that $0\in \mathbb{M}$ by part~(i).  Also note
that for all $k\in \mathbb{N}_0$, $s(k)\in \mathbb{M}$ whenever 
$k\in \mathbb{M}$ by part~(ii).  This
shows that $\mathbb{M}=\mathbb{N}_0$.  Since $n$ was arbitrary, this allows us
to conclude that $n+m$ is defined for all $n,m\in \mathbb{N}_0$.

We read $n+m$ as $n$ \textit{plus} $m$.  The operation $+$
is also called \textit{addition}.  

Among the natural numbers, the successor of zero plays a very
important role, so important that it deserves its own special
symbol.  Here we follow tradition and call the natural number
$s(0)$ \textit{one} and denote it by~$1$.
Note that for every natural number
$n$, we have $n+1=n+s(0)=s(n)$.  In particular, $0+1=1$.

With this notation, the Principle of Induction can be 
restated in the following form, which many of you may already have seen.

\medskip
\noindent
\textbf{The Principle of Induction.}\quad
Let $\mathbb{M}\subseteq \mathbb{N}_0$.  
Then $\mathbb{M} =\mathbb{N}_0$ if and only if
\begin{enumerate}
\item[(a).] $0\in \mathbb{M}$; and 
\item[(b).] $\forall k\in \mathbb{N}_0\quad(k\in \mathbb{M}) 
  \Longrightarrow(k+1\in \mathbb{M})$. 
\end{enumerate}

\begin{theorem}{\upshape [Associative Property of Addition]}\\
$m+(n+p) = (m+n)+p$, for all $m,n,p\in \mathbb{N}_0$.
\end{theorem}

\begin{proof}
Let $m,n\in \mathbb{N}_0$. Then let 
$\mathbb{M}$ denote the set of all natural numbers $p$ for
which $m+(n+p)=(m+n)+p$.  We show that $\mathbb{M}=\mathbb{N}_0$.

Note that 
\[
m+(n+0) = m+n = (m+n)+0 
\]
which shows that $0\in \mathbb{M}$.

Now assume that $k\in \mathbb{M}$, i.e., $m+(n+k) = (m+n)+k$.
Then
\[
m+[n+(k+1)]=m+[(n+k)+1]=[m+(n+k)]+1= [(m+n)+k]+1=(m+n)+(k+1). 
\]
Notice here that the first, second, and fourth equalities follow from
the second part of the definition of addition while the third uses our
inductive assumption that $m+(n+k)=(m+n)+k)$. This shows that $k+1\in
\mathbb{M}$.  Therefore, $\mathbb{M}= \mathbb{N}_0$.  Since $m$ and
$n$ were arbitrary elements of $\mathbb{N}_0$, the theorem follows.
\end{proof}

In proofs to follow, we will trim out some of the wording
and leave only the essential mathematical steps intact.
In particular, we will (i)~omit reference to the set $\mathbb{M}$,
and (ii)~drop the phrase ``For all $k\in\mathbb{N}_0$'' 
For example, to define addition, we will just
write (i)~$n+0=n$, and (ii)~$n+(k+1) = (n+k)+1$.

\begin{lemma}
$m+(n+1) = (m+1)+n$, for all $m,n\in\mathbb{N}_0$.
\end{lemma}
\begin{proof}
Fix $m\in \mathbb{N}_0$.  Then
\[
m+(0+1)=m+1= (m+0)+1.
\]
Now assume that $m+(k+1)=(m+1)+k$.
Then
\[
m+[(k+1)+1]=[m+(k+1)]+1=[(m+1)+k]+1=(m+1)+(k+1).
\]
\end{proof}

We next prove the commutative property, a task that takes
two steps. First, we prove the following special case.

\begin{lemma}
$n+0 = 0+n=n$, for all $n\in \mathbb{N}_0$.
\end{lemma}

\begin{proof}
The statement is trivially true when $n= 0$.
Now suppose that $k+0=0+k=k$ for some $k\in \mathbb{N}_0$.
Then
\[
(k+1) +0 = k+1 = (0+k)+1= 0+(k+1).
\]
\end{proof}

\begin{theorem}{\upshape [Commutative Law of Addition]}\\
$m+n=n+m$ for all $m,n\in \mathbb{N}_0$.
\end{theorem}

\begin{proof}
Let $m\in \mathbb{N}_0$. Then $m+0=0+m$ from the preceding lemma.
Assume $m+k=k+m$. Then
\[
m+(k+1)=(m+k)+1=(k+m)+1= k+(m+1) = (k+1)+m.
\]
\end{proof}

\begin{lemma}\label{lem:sum0}
If $m,n\in\mathbb{N}_0$ and $m+n=0$, then $m=n=0$.
\end{lemma}

\begin{proof}
Suppose that either of $m$ and $n$ is not zero. Since addition
is commutative, we may assume without loss of generality that
$n\neq0$.  Then there exists a natural number
$p$ so that $n=s(p)$.  This implies that $m+n=m+s(p)=s(m+p)=0$,
which is impossible since $0$ is not the successor of any
natural number.
\end{proof}

\begin{theorem}{\upshape [Cancellation Law of Addition]}\\
If $m,n,p\in \mathbb{N}_0$ and $m+p=n+p$, then $m=n$.
\end{theorem}

\begin{proof}
Let $m,n\in \mathbb{N}_0$.  Suppose that $m+0=n+0$.  Then $m=n$.
Now suppose that $m=n$ whenever $m+k=n+k$.
If $m+(k+1)=n+(k+1)$, then 
\[
s(m+k)=(m+k)+1=m+(k+1)=n+(k+1)=(n+k)+1=s(n+k).
\]
Since $s$ is an
injection, this implies $m+k=n+k$.  Thus $m=n$.
\end{proof}

\section{Multiplication as a Binary Operation}\label{s:mult}

We define a binary operation $\times$, called \textit{multiplication}, 
on the set of natural numbers.  When $m$ and $n$ are natural
numbers, $m\times n$ is also called the \textit{product}
of $m$ and $n$, and it sometimes denoted 
$m*n$ and even more compactly as $mn$.  We will use this
last convention in the material to follow. Let $n\in \mathbb{N}_0$. 

Then define:

\begin{enumerate}
\item[(i).] $n0=0$, and
\item[(ii).] $n(k+1)=nk +n$.
\end{enumerate}

Note that $10=0$ and $01=00+0=0$.  Also, note that $11=10+1=0+1=1$.
More generally, from~(ii) and Lemma~\ref{lem:sum0}, we 
conclude that if $m,n\neq0$, then $mn\neq0$.

\begin{theorem}{\upshape [Left Distributive Law]}\\
$m(n+p)=mn + mp$, for all $m,n,p\in \mathbb{N}_0$.
\end{theorem}

\begin{proof}
Let $m,n\in \mathbb{N}_0$.  Then 
\[
m(n+0)=mn = mn +0 = mn+ m0.
\]
Now assume $m(n+k) = mn + mk$.
Then
\begin{align*}
m[n+(k+1)] &= m[(n+k)+1]=m(n+k)+m\\
&=(mn+mk)+m=mn+(mk+m)= mn+m(k+1).
\end{align*}
\end{proof}

\begin{theorem}{\upshape [Right Distributive Law]}\\
$(m+n)p=mp + np$, for all $m,n,p\in \mathbb{N}_0$.
\end{theorem}

\begin{proof}
Let $m,n\in \mathbb{N}_0$.  Then
\[
(m+n)0 =0 = 0+0 = m0 + n0.
\]
Now assume $(m+n)k = mk + nk$.
Then
\begin{align*}
(m+n)(k+1)&=(m+n)k+(m+n)= (mk+nk) +(m+n)\\
&=(mk+m)+(nk+n)=m(k+1)+n(k+1).
\end{align*}
\end{proof}

\begin{theorem}{\upshape [Associative Law of Multiplication]}\\
$m(np) = (mn)p$, for all $m,n,p\in \mathbb{N}_0$.
\end{theorem}

\begin{proof}
Let $m,n\in \mathbb{N}_0$.  Then
\[
m(n0)= m0 = 0 = (mn)0.
\]
Now assume that $m(nk)=(mn)k$.  Then
\[
m[n(k+1)]= m(nk + n)= m(nk) + mn =(mn)k + mn = (mn)(k+1).
\]
\end{proof}

The commutative law requires  some preliminary work.

\begin{lemma}
$n0= 0n=0$, for all $n\in \mathbb{N}_0$.
\end{lemma}

\begin{proof}
The lemma holds trivially when $n=0$.  Assume $k0=
0k=0$.  Then
\[
(k+1)0 =0 = 0+0= 0k+0=0(k+1).
\]
\end{proof}

\begin{lemma}
$n1 =1n=n$, for every $n\in \mathbb{N}_0$.
\end{lemma}
\begin{proof}
$01=00+0=0 =10$.  Assume
$k1=1k=k$.  Then
\[
(k+1)1=k1+11=1k+1=1(k+1).
\]
\end{proof}

\begin{theorem}{\upshape [Commutative Law of Multiplication]}\\
$mn=nm$, for all $m,n\in \mathbb{N}_0$.
\end{theorem}

\begin{proof} Let $m\in \mathbb{N}_0$.
Then $m0=0m$.
Assume $mk=km$. Then
\[
m (k+1) = mk +m = km+m= km +1m=(k+1)m.
\]
\end{proof}

\section{Exponentiation}

We now define a binary operation called \textit{exponentiation}
which is defined only on those ordered pairs $(m,n)$ of natural
numbers where not both are zero.  The notation for exponentiation
is non-standard.  In books, it is written $m^n$ while the notations
$m**n$, $m\wedge n$ and $\exp(m,n)$ are used in-line.  We will use
the $m^n$ notation for the most part.

When $m=0$, we set $0^n=0$ for all $n\in\mathbb{N}_0$
with $n\neq0$.  Now let $m\neq0$.  We define $m^n$ by
(i)~$m^0=1$ and (ii)~$m^{k+1}=mm^k$.

\begin{theorem}
For all $m,n,p\in\mathbb{N}_0$ with $m\neq0$,
$m^{n+p}=m^n\,m^p$.
\end{theorem}

\begin{proof}
Let $m,n\in\mathbb{N}_0$ with $m\neq0$.  Then
$m^{n+0}=m^n=m^n\,1=m^n\,m^0$.
Now suppose that $m^{n+k}=m^n\,m^k$.
Then
\[
m^{n+(k+1)}=m^{(n+k)+1}=m\,m^{n+k}
 = m(m^n\,m^k)=m^n(m\,m^k)=m^n\,m^{k+1}.
\]
\end{proof}

\begin{theorem}
For all $m,n,p\in\mathbb{N}_0$ with $m\neq0$,
$(m^n)^p=m^{np}$.
\end{theorem}
\begin{proof}
Let $m,n\in\mathbb{N}_0$ with $m\neq0$.  Then
$(m^n)^0=1=m^0=m^{n0}$.  Now suppose that $(m^n)^k=m^{nk}$.
Then 
\[
(m^n)^{k+1}=m^n(m^n)^k=m^n(m^{nk})=m^{n+nk}=m^{n(k+1)}.
\]
\end{proof}

\section{Partial Orders and Total Orders}\label{s:order}

A binary relation $R$ on a set $X$ is just a subset of
the cartesian product $X\times X$.  In discussions of
binary relations, the notation $(x,y)\in R$ is sometimes
written as $xRy$.

A binary relation $R$ is:

\begin{enumerate} 
\item[(i).] \textit{reflexive} if $(x,x)\in R$ for
all $x\in X$.
\item[(ii).] \textit{antisymmetric} if $x=y$ whenever $(x,y)\in R$ and $(y,x)\in R$, for all $x,y\in X$.
\item[(iii).] \textit{transitive} if $(x,y)\in R$ and $(y,z)\in R$
imply $(x,z)\in R$, for all $x,y,z\in X$.
\end{enumerate}

A binary relation $R$ on a set $X$ is called a \textit{partial order} on
$X$ when it is reflexive, antisymmetric and transitive.
Traditionally, symbols like $\le$ and $\subseteq$ are used to denote
partial orders.  As an example, recall that if $X$ is a family of sets, 
we write $A\subseteq B$ when $A$ is a subset of $B$.  

When using the ordered pair notation for binary relations,
to indicate that a pair $(x,y)$ is not in the relation,
we simply write $(x,y)\notin R$. When using the alternate
notation, this is usually denoted by using the negation
symbol from logic and writing $\lnot (xRy)$.  Most of the
special symbols used to denote partial orders 
come with negative versions, e.g., $x\not\le y$, $x\nsubseteq y$.

A partial order is called a \textit{total order} on $X$ when for
all $x,y\in X$, $(x,y)\in R$ or $(y,x)\in R$.  For example,
if 
\[
X=\{\emptyset,\{\emptyset\},\{\emptyset,\{\emptyset\}\}\}
\]
then $\subseteq$ is a total order on $X$.

When $\le$ is a partial order on a set $X$, we write
$x<y$ when $x\le y$ and $x\neq y$.

\section{A Total Order on Natural Numbers}

Let $m,n\in \mathbb{N}_0$. Define a binary relation $\le$ on
$\mathbb{N}_0$ by setting $m\le n$ if and only if
there exists a natural number $p$ so
that $m+p=n$.  

\begin{proposition}
$\le$ is a total order on $\mathbb{N}_0$.
\end{proposition}
\begin{proof}
$\le$ is reflexive since $n+0=n$ and therefore $n\le n$, for all $n\in 
\mathbb{N}_0$.  Next, we show that $\le$ is antisymmetric.  
Let $m,n\in\mathbb{N}_0$ and suppose that $m\le n$ and $n\le m$.  
Then there exist natural numbers $p$ and $q$ so that $m+p=n$ and $n+q=m$.
It follows that
\[
m+(p+q)= (m+p)+q=n+q =m=m+0\]
Therefore $p+q=0$, which implies that $p=q=0$.  Thus $m+p=m+0=m=n$.

Next, we show that $\le$ is transitive.  Suppose that $m,n,p\in
\mathbb{N}_0$, $m\le n$ and $n\le p$.  Then there exist
natural numbers $q$ and $r$ so that $m+q=n$ and $n+r=p$.
Then
\[
m+(q+r)=(m+q)+r=n+r=p.
\]
Thus $m\le p$, and we have now shown that $\le$ is a partial
order on $\mathbb{N}_0$.


Finally, we show that $\le$ is a total order.
To accomplish this, we choose an arbitrary element $m\in\mathbb{N}_0$
and show that for every $n\in\mathbb{N}_0$, either $m\le n$ or
$n\le m$. We do this by induction on $n$.  
Suppose first that $n=0$.  Since $0+m=m$, we conclude that
$0\le m$.
Now suppose that for some $k\in\mathbb{N}_0$, we have
$m\le k$.  Then there is a natural number $p$ so that $m+p=k$.  Then
$m+(p+1) =(m+p)+1=k+1$, so $m\le k+1$.

On the other hand, suppose that for some $k\in\mathbb{N}_0$,
we have $k\le m$.  If $k=m$, then $m\le k$ and
$m\le k+1$ as above.  Now suppose that $k\le m$ and $k\neq m$.
Since $k\le m$, there exists a natural number $p$ so that $k+p=m$.
Since $k\neq m$, we know $p\neq 0$. Therefore, there is a 
natural number $q$ so that
$p=q+1$.  Then $m=k+p=k+(q+1)=(k+1)+q$ which shows that $k+1\le m$.
\end{proof}

Note that if $m,n\in\mathbb{N}_0$, then $m<n$ if and only if
there exists a natural number $p\neq 0$ so that $m+p=n$.

\begin{theorem}{\upshape [Monotonic Law for Addition]}\\ 
Let $m,n,p\in \mathbb{N}_0$.  If $m\le n$, then
$m+p\le n+p$. Furthermore, if $m<n$, then $m+p<n+p$.
\end{theorem}

\begin{proof}
It suffices to prove that if $m,n\in \mathbb{N}_0$ with $m<n$, then
$m+p<n+p$ for every $p\in\mathbb{N}_0$.  Let $q\neq0$ be
the natural number so that $m+q =n$.  
Now let $p\in \mathbb{N}_0$.
Then $(m+p)+q=(m+q)+p=n+p$, so $m+p < n+p$.
\end{proof} 

\begin{lemma}
If $m,n\in \mathbb{N}_0$, $m\neq0$ and $n\neq0$, then $mn\neq0$.
\end{lemma}
\begin{proof}
Assume to the contrary, that $m,n\in \mathbb{N}_0$, $m\neq0$, $n\neq0$
and $mn=0$.  Let $n=s(p)$.  Then $0=mn= ms(p)+m$ which requires
$m=0$.  This is a contradiction.
\end{proof}

\begin{theorem}{\upshape [Monotonic Law for Multiplication]}\\  
Let $m,n,p\in \mathbb{N}_0$.  If
$m\le n$, then $mp\le np$.  Furthermore, if $m<n$ and $p\neq0$,
then $mp<np$.
\end{theorem}

\begin{proof} Only the last statement requires proof.  Let 
$m,n\in\mathbb{N}_0$ with $m< n$.  Then $m+q=n$ for some $q\neq0$.
Then $np=(m+q)p=mp+pq$.  Since $pq\neq0$, we conclude $mp<np$.
\end{proof}

\begin{corollary}{\upshape [Cancellation Law of Multiplication]}\\
If $m,n,p\in \mathbb{N}_0$, $mp=np$, and $p\neq 0$, then $m=n$.
\end{corollary}

\begin{proof}
If $m<n$, then $mp<np$, and if $n<m$, then
$np<mp$.  We conclude that $m=n$. 
\end{proof}

\section{Notation for Natural Numbers}\label{s:decimal}

In some sense, we already have a workable notation for natural numbers.
In fact, we really didn't need a special symbol for $s(0)$.
The natural number $0$ and the sucessor function $s$ are enough.
For example, the positive integer associated
with the number of fingers (including the thumb) 
on one hand is
$s(s(s(s(s(0)))))$, our net worth is $0$, and the age of Professor
Trotter's son in years when these notes were first prepared was
\[
s(s(s(s(s(s(s(s(s(s(s(s(s(s(s(s(s(s(0)))))))))))))))))).
\]
Admittedly, this is not very practical, especially if some day
we win the lottery or want to discuss the federal deficit. 
So it is natural (ugh!) to consider alternative notations.  

Here is one such scheme.  First, let's
decide on a natural $b>s(0)$ as \textit{base}.  We will
then develop a notation which is called the \textit{base $b$
notation}.  We already have a special symbol for zero, namely $0$,
but we need additional symbols for each natural number $n$ with
$0<n<b$. These symbols
are called \textit{digits}.  For example,
the positive integer $b=s(s(s(s(s(s(s(s(0))))))))$ is called 
\textit{eight}, and it makes a popular choice as a base.
Here are the symbols (digits) customarily chosen for this base: 
$1=s(0)$, $2=s(1)$;
$3=s(2)$; $4=s(3)$; $5=s(4)$; $6=s(5)$; and $7=s(6)$.  Technically
speaking, it is not necessary to have a separate symbol for
$b$, but it might be handy regardless. In this case, most
people prefer the symbol $8$.  We like this symbol, unless and until
it gets lazy and lays down sideways.

So the first $8$ natural numbers are then $0$, $1$, $2$, $3$, $4$,
$5$, $6$ and $7$.  To continue with our representation, we want 
to use the following basic theorem.

\begin{theorem}
Let $n,d\in \mathbb{N}_0$ with $d>0$.  Then there exist unique natural
numbers $q$ and $r$ so that $n=qd+r$ and $0\le r < d$.
\end{theorem}

\begin{proof}
Let $d\in \mathbb{N}_0$ with $d>0$.  We first show that for each 
$n\in\mathbb{N}_0$, there exists $q,r\in\mathbb{N}_0$ so that
$n=qd+r$ and $0\le r<d$.  If $n=0$, we can take
$q=0$ and $r=0$.  Now suppose that $k=qd+r$ and $0\le r<m$ for 
some $k\in \mathbb{N}_0$.

Note that $r<d$ implies $r+1\le d$.
If $r+1<d$, then $k+1=qd+(r+1)$. On the other hand, if
$r+1=d$, then $k+1=(q+1)d+0$.

Now that existence has been settled, we note that the uniqueness 
of $q$ and $r$ follow immediately from the cancellation properties.
\end{proof}

Now suppose that for some $k\in \mathbb{N}_0$, with
$k\ge 7$, we have defined a base eight notation for the representation
of $k$, for all $n$ with $0\le n\le k$, and that in each case, 
this representation consists of a string of digits,
written left to right, and selected from $\{0,1,2,3,4,5,6,7\}$.
Write $k+1=qb+r$ where $0\le r<b$.  Note that $q\le k$, so that
we already have a representation for $q$.  To obtain a representation
of $k+1$, we simply append $r$ at the (right) end.  

For example, consider the age of Professor Trotter's son. It is then written
as~22.  And to emphasize the base eight notation, most people would
say $22$, base~$8$ and write $(22)_8$.  

Among the more popular bases are base~2, where only the digits~0 and~1
are used, and base~sixteen, where sixteen is the popular word for 
$(20)_8$.  Here the digit symbols are
\[
0,1,2,3,4,5,6,7,8,9,A,B,C,D,E,F
\]
Another popualar choice, in fact the one in most widespread use
in banks, shopping centers and movie theatres, is base \textit{ten}.
Ten is the natural number~A, base sixteen.  Also, ten is $(12)_8$.
Most folks use the digits $0,1,2,3,4,5,6,7,8,9$ for base ten
notation.  And when no other designation is made, then it is assumed
that the natural number is written base ten.  So of course, Professor
Trotter's son is 18 and is a freshman at Georgia Tech.  Which explains
why his hair is as white as it is.

For any base $b>1$, caution must be exercised when discussing 
multiplication, since writing the product $m\times n$ in the 
abbreviated form $mn$ causes us some grief.  For example, if $b = 8$,
then writing the product $372\times4885$ as $3724885$ is
ambiguous.  For this reason, when using base~$b$ notation, the
product symbol $\times$ (or some variation of $\times$)
is always used.

\subsection{Alternate Versions of Induction}

Many authors prefer to start the development of number systems
with the set of \textit{positive integers} and defer the
introduction of the concept of zero.  In this setting,
you have a non-empty set $\posints$, a one-to-one 
\textit{successor} function $s:\posints\injection\posints$ and
a positive integer called \textit{one} and denoted~$1$ that
is not the successor of any positive integer.  The Principle of
Induction then becomes:  If $\mathbb{M}\subseteq\posints$, 
then $\mathbb{M} =\posints$ if and only if
\begin{enumerate}
\item[(a).] $1\in \mathbb{M}$; and 
\item[(b).] $\forall k\in \mathbb{N}_0\quad(k\in \mathbb{M}) 
  \Longrightarrow(s(k)\in \mathbb{M})$. 
\end{enumerate}

More generally, to show that a set $\mathbb{M}$ contains
all integers greater than or equal to an integer~$n$, it is
sufficient to show that (i)~$n\in\mathbb{M}$, and (ii)~For
all $k\in\ints,\,(k\in\mathbb{M}\Longrightarrow(k+1\in\mathbb{M})$.

Here is another version of induction, one that is particularly useful
in combinatorial arguments.

\begin{theorem}
Let $\mathbb{M}\subseteq\posints$.  If $\mathbb{M}\neq\posints$,
then there is a unique least positive integer $n$ that does not
belong to $\mathbb{M}$.
\end{theorem}

\section{Equivalence Relations}\label{sec:numsys:equivalence}

A binary relation $R$ is:

\begin{enumerate}
\item[(iv).] \textit{symmetric} if $(x,y)\in R$ implies $(y,x)\in R$ for
all $x,y\in X$.
\end{enumerate}

A binary relation $R$ on a set $X$ is called an
\textit{equivalence relation} when it is reflexive, symmetric
and transitive.  Typically, symbols like, $=$, $\cong$,
$\equiv$ and $\sim$ are used to
denote equivalence relations.  An equivalence relation, say
$\cong$, defines
a partition on the set $X$ by setting
\[
\langle x\rangle =\{y\in X: x\cong y\}
\]
Note that if $x,y\in X$ and $\langle x\rangle\cap\langle y\rangle
\neq\emptyset$, then $\langle x\rangle=\langle y\rangle$.
The sets in this partition are called \textit{equivalence classes}.

When using the ordered pair notation for binary relations,
to indicate that a pair $(x,y)$ is not in the relation,
we simply write $(x,y)\notin R$. When using the alternate
notation, this is usually denoted by using the negation
symbol from logic and writing $\lnot (xRy)$.  Many of the
special symbols used to denote equivalence relations
come with negative versions: $x\neq y$, $x\ncong y$,
$x\nsim y$, etc.

\section{The Integers as Equivalence Classes of Ordered Pairs}\label{s:integers}

Define a binary relation $\cong$ on the set $Z=\mathbb{N}_0
 \times\mathbb{N}_0$ by
\[
(a,b)\cong (c,d)\quad\textit{iff}\quad a+d=b+c.
\]

\begin{lemma}
$\cong$ is reflexive.
\end{lemma}

\begin{proof}
Let $(a,b)\in Z$.  Then $a+b=b+a$, so $(a,b)\cong(b,a)$.
\end{proof}

\begin{lemma}
$\cong$ is symmetric.
\end{lemma}

\begin{proof} Let $(a,b),(c,d)\in Z$ and suppose
that $(a,b)\cong (c,d)$.  Then $a+d=b+c$, so that
$c+b=d+a$.  Thus $(c,d)\cong (a,b)$.
\end{proof}

\begin{lemma}
$\cong$ is transitive.
\end{lemma}

\begin{proof}
Let $(a,b), (c,d), (e,f)\in Z$.  Suppose that
\[
(a,b)\cong(c,d)\quad\text{and}\quad (c,d)\cong (e,f).
\]
Then $a+d=b+c$ and $c+f=d+e$.  Therefore,
\[
(a+d)+(c+f) =(b+c)+(d+e).
\]
It follows that
\[
(a+f)+(c+d) =(b+e)+(c+d).
\]
Thus $a+f = b+e$ so that $(a,b)\cong(e,f)$.
\end{proof}

Now that we know that $\cong$ is an equivalence relation on
$Z$, we know that $\cong$ partitions $Z$ into equivalence classes.
For an element $(a,b)\in Z$, we denote the equivalence class
of $(a,b)$ by $\langle (a,b)\rangle$.

Let $\ints$ denote the set of all equivalence classes of $Z$
determined by the equivalence relation $\cong$.  The elements
of $\ints$ are called \textit{integers}.

\section{Properties of the Integers}\label{s:moreintegers}

For the remainder of this chapter, most statements will be given
without proof.  Students are encouraged to fill in the
details.  

We define a binary operation $+$ on $\ints$ by the following rule:
\[
\langle(a,b)\rangle+\langle(c,d)\rangle = \langle(a+c,b+d)\rangle.  
\]

Note that the definition of addition is made in terms of 
representatives of the class, so we must pause to make sure
that $+$ is \textit{well defined}, i.e., independent of
the particular representatives.

\begin{lemma}
If $\langle(a,b)\rangle =\langle(c,d)\rangle$ and 
$\langle(e,f)\rangle=\langle(g,h)\rangle$, then
$\langle(a,b)\rangle+\langle(e,f)\rangle=
\langle(c,d)\rangle+\langle(g,h)\rangle$.
\end{lemma}

\begin{proof}
Since $(a,b)\cong(c,d)$, we know
$a+d=b+c$.  Since $(e,f)\cong(g,h)$, 
we know $e+h=f+g$. It follows that $(a+d)+(e+h) = (b+c)+ (f+g)$.
Thus $(a+e)+(d+h)= (b+f)+(c+g)$, which implies that
$\langle(a,b)\rangle+\langle(e,f)\rangle=
\langle(c,d)\rangle+\langle(g,h)\rangle$.
\end{proof}

In what follows, we use a single symbol, like $x$, $y$ or
$z$ to denote an integer, but remember that each integer
is in fact an entire equivalence class whose elements are 
ordered pairs of natural numbers.

\begin{theorem}
For all $x,y,z\in\ints$,
\begin{enumerate}
\item $x+y=y+x$;
\item $x+(y+z)= (x+y)+z$; and
\item $x+y= x+z$ implies $y=z$.
\end{enumerate}
\end{theorem}

Next, we define a second binary operation called
\textit{multiplication}, and denoted $x\times y$, $x*y$ or
just $xy$.  When $x=\langle(a,b)\rangle$ and $y=\langle(c,d)\rangle$, 
we define:
\[
xy =\langle(a,b)\rangle\langle(c,d)\rangle= \langle(ac+bd, ad+bc)\rangle.
\]

\begin{theorem}  Multiplication is well defined.  Furthermore,
\begin{enumerate}
\item $xy=yx$,  for every $x,y\in \ints$.
\item $x(yz)=(xy)z$, for every $x,y,z\in \ints$.
\item $x(y+z)=xy+xz$, for every $x,y,z\in \ints$.
\end{enumerate}
\end{theorem}

The integer $\langle(0,0)\rangle$ has a number of special properties. 
Note that for all $x\in\ints$, $x+\langle(0,0)\rangle= x$ 
and $x\langle(0,0)\rangle=\langle(0,0)\rangle$.
So most folks call $\langle(0,0)\rangle$ \textit{zero}
and denote it by~$0$.  This is a terrible abuse of notation, 
since we have already used the word zero and the symbol $0$ to 
denote a particular natural number.

But mathematicians, computer scientists and even
real people do this all the time.  We use the same
word and even the same phrase in many different settings 
expecting that the listener
will make the correct interpretation.  For example, how many
different meanings do you know for \textit{You're so bad}?


If $x=\langle(a,b)\rangle$ is an integer and $y=
\langle(b,a)\rangle$, 
then $x+y=\langle(a+b,a+b)\rangle=0$.  
The integer $y$ is then
called the \textit{additive inverse} of $x$  and is denoted
$-x$.  The additive inverse of $x$ is also called \textit{minus $x$}.
The basic property is that $x + (-x) = 0$, for every $x\in \ints$.

We can now define a new binary operation, called \textit{subtraction}
and denoted $-$, on $\ints$ by
setting $x-y= x+(-y)$.  In general, subtraction is neither
commutative nor associative.  However, we do have the following
basic properties.

\begin{theorem}
For all $x,y,z\in\ints$,
\begin{enumerate}
\item $x(-y)=-xy$;
\item $x(y-z)= xy-xz$; and
\item $-(x-y)=y-x$.
\end{enumerate}
\end{theorem}

Next, we define a total order on $\ints$ by
setting $x\le y$ in $\ints$ when $x=\langle(a,b)\rangle$, $y=\langle(c,d)
\rangle$ and
$a+d \le b+c$ in $\mathbb{N}_0$.

\begin{theorem}[Monotonic Law for Addition] 
Let $x,y,z\in \ints$.  If $x\le y$, then
$x+z\le y+z$.  Furthermore, if $x<y$, then $x+z<y+z$.
\end{theorem}

For multiplication, the situation is more complicated.

\begin{theorem}[Monotonic Law for Multiplication]  
Let $x,y,z\in \ints$.  If
$x<y$, then
\begin{enumerate}
\item $xz<yz$, if $z>0$, 
\item $xz=yz=0$, if $z=0$, and
\item $xz>yz$, if $z<0$.
\end{enumerate}
\end{theorem}

Now consider the function $f:\mathbb{N}_0\longrightarrow \ints$ defined
by $f(n) = \langle(n,0)\rangle$.  It is easy to show that
$f$ is an injection.  Furthermore, it respects addition
and multiplication, i.e., $f(n+m)=f(n)+f(m)$ and
$f(nm)=f(n)f(m)$.   Also, note that if $x\in \ints$, then
$x>0$ if and only if $x=f(n)$ for some $n\in \mathbb{N}_0$. So, it is 
customary to abuse notation slightly and say that $\mathbb{N}_0$ is a 
``subset'' of $\ints$.  Similarly, we can either consider the set
$\posints$ of positive integers as the set of natural numbers that are
successors, or as the set of integers
that are greater than~$0$.

When $n$ is a positive integer and $0$ is the zero in $\ints$, we
define $0^n=0$.  When $x\in\ints$, $x\neq 0$ and $n\in\mathbb{N}_0$,
we define $x^n$ inductively by~(i)~$x^0=1$ and $x^{k+1}=xx^k$.

\begin{theorem}
If $x\in\ints$, $x\neq0$, and $m,n\in\mathbb{N}_0$, then
$x^mx^n=x^{m+m}$ and $(x^m)^n=x^{mn}$.
\end{theorem}

\section{Obtaining the Rationals from the Integers}\label{s:rationals}

We consider the set $\rats$ of all ordered pairs in $\ints\times \ints$ of
the form $(x,y)$ with $y\neq 0$. Elements of $\rats$ are called
\textit{rational numbers}, or \textit{fractions}.  Define
an equivalence relation, denoted $=$, on $\ints$ by
setting $(x,y)=(z,w)$ if and only if $xw=yz$.  Here we
should point out that the symbol $=$ can be used (and often is)
to denote an equivalence relation. It is not constrained to mean ``identically
the same.''
  
When $q=(x,y)$ is a fraction, $x$ is called the \textit{numerator}
and $y$ is called the \textit{denominator} of $q$.  
Remember that the denominator of a fraction is never zero.

Addition of fractions is defined by 
\[
(a,b)+(c,d) = (ad+bc,bd)
\]
while multiplication is defined by
\[
(a,b)(c,d) = (ac,bd).
\]
As was the case with integers, it is important to pause
and prove that both operations are well defined.

\begin{theorem}
Let $x,y,z,w\in\rats$.
If $x=y$ and $z=w$, then $x+z=y+w$ and $xz=yw$.
\end{theorem} 

Addition and multiplication are both associative and commutative. 
Also, we have the distributive property.

\begin{theorem}
Let $x,y,z\in Q$.  Then
\begin{enumerate}
\item $x+y=y+x$ and $xy=yx$.
\item $x+(y+z) = (x+y)+z$ and $x(yz)=(xy)z$.
\item $x(y+z)=xy+xz$.
\end{enumerate}
\end{theorem}

The additive inverse of a fraction $(a,b)$ is just $(-a,b)$.
Using this, we define subtraction for fractions:
$(a,b)-(c,d)=(a,b)+(-c,d)$.

When $(a,b)$ is a fraction, and $a\neq 0$, the fraction $(b,a)$ is the 
\textit{reciprocal} of $(a,b)$.  The reciprocal is also
called the \textit{multiplicative inverse}, and the reciprocal 
of $x$ is denoted $x^{-1}$.  When $y\neq0$, we can then define
\textit{division} by setting $x/y=xy^{-1}$, i.e.,
$(a,b)/(c,d)=(ad, bc)$.  Of course, division by zero is not
defined, a fact that you probably already knew!

As was the case for  both $\mathbb{N}_0$ and $\ints$, when
$n$ is a positive integer, and $0$ is the zero in $\rats$,
we define $0^n=0$.  When $x=(a,b)$ is a fraction with $x\neq0$
and $n$ is a non-negative integer, we define $x^n$ inductively
by (i)~$x^0=1$ and (ii)~$x^{n+1}=xx^n$.

\begin{theorem}
If $x\in\rats$, $x\neq0$, and $m,n\in\ints$, then
$x^mx^n=x^{m+m}$ and $(x^m)^n=x^{mn}$.
\end{theorem}

Many folks prefer an alternate notation for fractions in
which the numerator is written directly over the denominator
with a horizontal line between them, so $(2,5)$ can
also be written as $\frac{2}{5}$.

Via the map $g(x) = (x,1)=\frac{x}{1}$, we again say that the integers
are a ``subset'' of the rationals.  As before, note that
$g(x+y) = g(x)+g(y)$, $g(x-y)=g(x)-g(y)$ and $g(xy)=g(x)g(y)$.

In the third grade,
you were probably told that $5 =\frac{5}{1}$, but by now
you are realizing that this is not exactly true.   
Similarly, if you had told your teacher that $\frac{3}{4}$ and
$\frac{6}{8}$ weren't really the same and were only ``equal'' in 
the broader sense of an equivalence relation defined on
a subset of the cartesian product of the integers, you
probably would have been sent to the Principal's office.

Try to imagine the trouble you would have gotten into had
you insisted that the real meaning
of $\frac{1}{2}$ was
\[
\frac{1}{2} =\langle(\langle(s(s(0)),s(0))\rangle,
  \langle(s(s(0)),0)\rangle)\rangle
\]

We can also define a total order on $\rats$. To do this, we assume
that $(a,b),(c,d)\in\rats$ have $b,d>0$. (If $b<0$, for example, we
would replace it by $(a',b')=(-a,-b)$, which is in the same
equivalence class as $(a,b)$ and has $b'>0$.) Then we set $(a,b)\leq
(c,d)$ in $\rats$ if $ad\leq bc$ in $\ints$.

\subsection{Integer Exponents}

When $n$ is a positive integer and $0$ is the zero in $\rats$, we
define $0^n=0$.  When $x\in\rats$, $x\neq 0$ and $n\in\mathbb{N}_0$,
we define $x^n$ inductively by~(i)~$x^0=1$ and $x^{k+1}=xx^k$.  When
$n\in\ints$ and $n<0$, we set $x^n=1/x^{-n}$.

\begin{theorem}
If $x\in\rats$, $x\neq0$, and $m,n\in\ints$, then
$x^mx^n=x^{m+m}$ and $(x^m)^n=x^{mn}$.
\end{theorem}

\section{Obtaining the Reals from the Rationals}\label{s:reals}

A full discussion of this would take us far away from 
a discrete math class, but let's at least provide the
basic definitions.  A subset $S\subset \rats$ of the rationals
is called a \textit{cut} (also, a \textit{Dedekind cut}), 
if it satisfies the following properties:

\begin{enumerate}
\item $\emptyset\neq S\neq \rats$, i.e, $S$ is a proper non-empty subset
of $\rats$.
\item $x\in S$ and $y<x$ in $\rats$ implies $y\in S$, for all $x,y\in \rats$.
\item For every $x\in S$, there exists $y\in S$ with $x<y$, i.e., 
$S$ has no greatest element.
\end{enumerate}

Cuts are also called \textit{real numbers}, so a real number is
a particular kind of set of rational numbers.
For every rational number $q$, the set $\bar{q}= \{p\in \rats: p< q\}$
is a cut.  Such cuts are called \textit{rational cuts}.  Inside
the reals, the rational cuts behave just like the rational
numbers and via the map $h(q)=\bar{q}$, we abuse notation again
(we are getting used to this) and say that
the rational numbers are a subset of the real numbers.  

But there are cuts which are not rational.  Here is one:
$\{p\in \rats: p\le 0\}\cup \{p\in \rats: p^2<2\}$.  The fact that
this cut is not rational depends on the familiar proof that
there is no rational $q$ for which $q^2=2$.

The operation of addition on cuts is defined in the natural
way.  If $S$ and $T$ are cuts, set $S+T=\{s+t:s\in S, t\in T\}$.
Order on cuts is defined in terms of inclusion, i.e., $S<T$ if and
only if $S\subsetneq T$.  A cut is \textit{positive} if it
is greater than $\bar{0}$.  When $S$ and $T$ are positive
cuts, the product $ST$ is defined by
\[
ST= \bar{0}\cup\{st:s\in S, t\in T, s\ge0, t\ge 0\}.
\]

One can easily show that there is a real number $r$ so that
$r^2=\bar{2}$.  You may be surprised, but perhaps not, to learn
that this real number is denoted $\sqrt2$.

There are many other wonders to this story, but enough for
one day.

\section{Obtaining the Complex Numbers from the Reals}

By now, the following discussion should be transparent.
The complex number system $\mathbb{C}$ is just the cartesian product
$\reals\times\reals$ with 
\begin{enumerate}
\item $(a,b) = (c,d)$ in $\mathbb{C}$ if and only if $a=c$ and $b=d$
in $\reals$.
\item $(a,b)+(c,d)=(a+c,b+d)$.
\item $(a,b)(c,d)=(ac-bd, ad+bc)$.
\end{enumerate}

Now the complex numbers of the form $(a,0)$ behave just like
real numbers, so is natural to say that the complex number system
\textit{contains} the real number system.  Also, note that
$(0,1)^2=(0,1)(0,1)=(-1,0)$, i.e., the complex number $(0,1)$ has the
property that its square is the complex number behaving like the
real number~$-1$.  So it is convenient to use a special symbol like
$i$ for this very special complex number and note that $i^2=-1$.

With this beginning, it is straightforward to develop all the
familiar properties of the complex number system.

\subsection{Decimal Representation of Real Numbers}

Every real number has a decimal expansion---although the
number of digits after the decimal point may be infinite.
A rational number $q=m/m$ from $\rats$ has an expansion
in which a certain block of digits repeats indefinitely.
For example, \[
\frac{2859}{35} = 81.6857142857142857142857142857142857142857142\dots
\]
In this case, the block $857142$ of size~$6$ is repeated forever.

Certain rational numbers have \textit{terminating} decimal expansions.
For example $385/8= 48.125$.  If we chose to do so, we could
write this instead as an infinite decimal by appending trailing $0$'s,
as a repeating block of size~$1$:
\[
\frac{385}{8} = 48.1250000000000000000000000000000000\dots
\]
On the other hand, we can also write the decimal expansion of
$385/8$ as:
\[
\frac{385}{8} = 48.12499999999999999999999999999999999\dots
\]
Here, we intend that the digit $9$, a block of size~$1$, be repeated forever.
Apart from this anomaly, the decimal expansion of real numbers is unique.

On the other hand, irrational numbers have non-repeating decimal 
expansions in which there is no block of repeating digits that
repeats forever.

You all know that $\sqrt{2}$ is irrational.  Here is the first part 
of its decimal expansion:
\[
\sqrt{2} =1.41421356237309504880168872420969807856967187537694807317667973\dots
\]
An irrational
number is said to be \textit{algebraic} if it is the root of
polynomial with integer coeffcients; else it is said to be
\textit{transcendental}.   
For example, $\sqrt{2}$ is \textit{algebraic} since it is the
root of the polynomial $x^2-2$.

Two other famous examples of irrational numbers are $\pi$ and $e$.
Here are their decimal expansions:
\[
\pi =3.14159265358979323846264338327950288419716939937510582097494459\dots
\]
and 
\[
e=2.7182818284590452353602874713526624977572470936999595749669676277\dots
\]
Both $\pi$ and $e$ are \textit{transcendental}.

\begin{example}
Amanda and Bilal, both students at a nearby university, have
been studying rational numbers that have
large blocks of repeating digits in their decimal expansions.
Amanda reports that she has found two positive integers
$m$ and $n$ with $n<500$ for which the decimal expansion
of the rational number $m/n$ has a block of 1961 digits which
repeats indefinitely.  Not to be outdone, Bilal brags that
he has found such a pair $s$ and $t$ of positive
integers with $t<300$ for which the
decimal expansion of $s/t$ has a block of $7643$ digits which
repeats indefinitely.  Bilal should be (politely) told to
do his arithmetic more carefully, as there is no such pair
of positive integers (\textit{Why}?).  On the other hand, Amanda may in fact
be correct---although, if she has done her work with more
attention to detail, she would have reported that the decimal
expansion of $m/n$ has a smaller block of repeating digits (\textit{Why}?).
\end{example}

\begin{proposition} 
There is no surjection from $\posints$ to the
set $X= \{x\in\reals:0<x<1\}$.
\end{proposition}

\begin{proof}
Let $f$ be a function from $\posints$ to $X$.
For each $n\in \posints$, consider the decimal
expanion(s) of the real number $f(n)$.  Then
choose a positive integer $a_n$ so that (1)~$a_n\le 8$, and
(2)~$a_n$ is not the $n^{th}$ digit after the decimal
point in any decimal expansion of $f(n)$.
Then the real number $x$ whose decimal expansion
is $x=.a_1a_2a_3a_4a_5\dots$ is an element of $X$
which is distinct from $f(n)$, for every $n\in\posints$.
This shows that $f$ is not a surjection.
\end{proof}

\section{The Zermelo-Fraenkel Axioms of Set Theory}

In the first part of this appendix, 
we put number systems on a firm foundation, but in the
process, we used an intuitive understanding of sets.
Not surprisingly, this approach is fraught with danger.  As was first discovered
more than 100 years ago, there are major conceptual hurdles
in formulating consistent systems of axioms for set theory.
And it is very easy to make statements that sound ``obvious''
but are not.

Here is one very famous example.  Let $X$ and $Y$ be
sets and consider the following two
statements:

\begin{enumerate}
\item There exists an injection $f:X\rightarrow Y$.
\item There exists a surjection $g:Y\rightarrow X$.
\end{enumerate}
If $X$ and $Y$ are finite sets, these statements are
equivalent, and it is perhaps natural to surmise that the
same is true when $X$ and $Y$ are infinite.  But that is not
the case.

A good source of additional (free) information on set theory is
the collection of Wikipedia articles.  Do a web search and
look up the following topics and people:

\begin{enumerate}
\item Zermelo Frankel set theory.
\item Axiom of Choice.
\item Peano postulates.
\item Georg Cantor, Augustus De Morgan, George Boole, Bertrand Russell
and Kurt G\"odel.
\end{enumerate}

Here is the system of axioms popularly known as ZFC, which
is an abbreviation for Zermelo-Frankel plus the Axiom of Choice.
In this system,
the notion of \textit{set} and the membership operator $\in$
are undefined.  However, if $A$ and $B$ are sets, then
exactly one of the following statements is true: 
(i)~$A\in B$ is \textit{true}; (ii)~$A\in B$ is \textit{false}.  
When $A\in B$ is false, we write $A\notin B$. Also, there 
is an equivalence relation $=$ defined on sets.

\medskip
\noindent
\textbf{Axiom of extensionality:}\quad
Two sets are equal if and only if they have the same elements. 
 
\medskip
\noindent
\textbf{Axiom of empty set:}\quad
There is a set $\emptyset$ with no elements. 
 
\medskip
\noindent
\textbf{Axiom of pairing:}\quad 
If $x$ and $y$ are sets, then there exists a set containing $x$ and $y$ 
as its only elements, which we denote by $\{x,y\}$.  Note: If
$x=y$, then we write only $\{x\}$.
 
\medskip
\noindent
\textbf{Axiom of union:}\quad 
For any set $x$, there is a set $y$ such that the elements of $y$ 
are precisely the elements of the elements of $x$. 
 
\medskip
\noindent
\textbf{Axiom of infinity:}\quad 
There exists a set $x$ such that $\emptyset\in x$ and whenever 
$y\in x$, so is $\{y ,\{y\}\}$. 
 
\medskip
\noindent
\textbf{Axiom of power set}\quad 
Every set has a power set. That is, for any set $x$, 
there exists a set $y$, such that the elements of $y$ are 
precisely the subsets of~$x$. 
 
\medskip
\noindent
\textbf{Axiom of regularity:}\quad 
Every non-empty set $x$ contains some element $y$ such that 
$x$ and $y$ are disjoint sets. 
 
\medskip
\noindent
\textbf{Axiom of separation (or subset axiom):}\quad 
Given any set and any proposition $P(x)$, there is a subset 
of the original set containing precisely those elements $x$ 
for which $P(x)$ holds. 
 
\medskip
\noindent
\textbf{Axiom of replacement:}\quad
Given any set and any mapping, formally defined as a 
proposition $P(x,y)$ where $P(x,y_1)$ and $P(x,y_2)$ implies 
$y_1 = y_2$, there is a set containing precisely the images 
of the original set's elements. 
 
\medskip
\noindent
\textbf{Axiom of choice:}\quad 
Given any set of mutually exclusive non-empty sets, there 
exists at least one set that contains exactly one element in 
common with each of the non-empty sets. 

%%% Local Variables:
%%% mode: latex
%%% TeX-master: "book"
%%% End:

