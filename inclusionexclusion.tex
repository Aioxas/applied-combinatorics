% inclusionexclusion.tex
% Updated January 11, 2012

\chapter{Inclusion-Exclusion}\label{ch:inclusion-exclusion}

In this chapter, we study a classic enumeration technique known as
Inclusion-Exclusion. In its simplest case, it is absolutely intuitive.
Its power rests in the fact that in many situations, we start with an
exponentially large calculation and see it reduce to a manageable
size. We focus on three applications that every student of
combinatorics should know: (1)~counting surjections, (2)~derangements,
and (3)~the Euler $\phi$-function.

\section{Introduction}\label{s:inclusion-exclusion:intro}

We start this chapter with an elementary example.
\begin{example} \label{exa:inclusion-exclusion:students}
  Let $X$ be the set of $63$ students in an applied combinatorics
  course at a large technological university. Suppose there are $47$
  computer science majors and $51$ male students. Also, we know there
  are $45$ male students majoring in computer science. How many
  students in the class are female students not majoring in computer
  science?

  Although the Venn diagrams that you've probably seen drawn many
  times over the years aren't always the best illustrations
  (especially if you try to think with some sort of scale), let's use
  one to get started. In \autoref{fig:venn-app-comb}, we see how the
  groups in the scenario might overlap.
 \begin{figure}[h]
    \centering
    \begin{overpic}{inclusionexclusion-figs/Venn-diagram}
      \put(15,28.5){Males}
%      \put(78,14){Females}
      \put(63,28.5){CS Majors}
      \put(22,8){Non-CS major females}
    \end{overpic}
    \caption{A Venn diagram for an applied combinatorics class}
    \label{fig:venn-app-comb}
  \end{figure}
  Now we can see that we're after the number of students in the white
  rectangle but outside the two shaded ovals, which is the female
  students not majoring in computer science. To compute this, we can
  start by subtracting the number of male students (the blue region)
  from the total number of students in the class and then subtracting
  the number of computer science majors (the yellow region). However,
  we've now subtracted the overlapping region (the male computer
  science majors) \textit{twice}, so we must add that number
  back. Thus, the number of female students in the class who are not
  majoring in computer science is
  \[63 - 51 - 47 + 45 = 10.\]
\end{example}

\begin{example}\label{exa:inclusion-exclusion:int-solns}
  Another type of problem where we can readily see how such a
  technique is applicable is a generalization of the problem of
  enumerating integer solutions of equations. In \autoref{ch:strings},
  we discussed how to count the number of solutions to an equation
  such as
  \[x_1 + x_2 + x_3 + x_4 = 100,\] where $x_1>0$, $x_2, x_3 \geq 0$
  and $2\leq x_4\leq 10$. However, we steered clear of the situation
  where we add the further restriction that $x_3\leq 7$. The previous
  example suggests a way of approaching this modified problem.

  First, let's set up the problem so that the lower bound on each
  variable is of the form $x_i\geq 0$. This leads us
  to the revised problem of enumerating the integer solutions to
  \[x_1' + x_2 + x_3 + x_4' = 97\] with $x_1',x_2,x_3,x_4'\geq 0$,
  $x_3\leq 7$, and $x_4'\leq 8$. (We'll then have $x_1 = x_1'+1$ and
  $x_4 = x_4' + 2$ to get our desired solution.) To count the number
  of integer solutions to this equation with $x_3\leq 7$ and $x_4'\leq
  8$, we must exclude any solution in which $x_3 > 7$ \emph{or} $x_4'
  > 8$. There are $C(92,3)$ solutions with $x_3 > 7$, and the number
  of solutions in which $x_4'>8$ is $C(91,3)$. At this point, it might
  be tempting to just subtract $C(92,3)$ and $C(91,3)$ from
  $C(100,3)$, the total number of solutions with all variables
  nonnegative. However, care is required. If we did that, we would
  eliminate the solutions with both $x_3>7$ \emph{and} $x_4'>8$
  \emph{twice}. To account for this, we notice that there are
  $C(83,3)$ solutions with both $x_3>7$ and $x_4'>8$. If we add this
  number back in after subtracting, we've ensured that the solutions
  with both $x_3>7$ and $x_4'>8$ are not included in the total count
  and are not excluded more than once. Thus, the total number of
  solutions is 
  \[\binom{100}{3} - \binom{92}{3} - \binom{91}{3} + \binom{83}{3} = 6516.\]
\end{example}

From these examples, you should start to see a pattern emerging that
leads to a more general setting. In full generality, we will consider
a set $X$ and a family $\mathcal{P}=\{P_1,P_2,\dots,P_m\}$ of
\textit{properties}.  We intend that for every $x\in X$ and each
$i=1,2,\dots,m$, either $x$ satisfies $P_i$ or it does not.  There is
no ambiguity.  Ultimately, we are interested in determining the number
of elements of $X$ which satisfy \textit{none} of the properties in
$\mathcal{P}$. In
\hyperref[exa:inclusion-exclusion:students]{Example~\ref*{exa:inclusion-exclusion:students}},
we could have made property $P_1$ ``is a computer science major'' and
property $P_2$ ``is male''. Then the number of students satisfying
\emph{neither} $P_1$ nor $P_2$ would be the number of female students
majoring in something other than computer science, exactly the number
we were asked to determine. What would the properties $P_1$ and $P_2$
be for
\hyperref[exa:inclusion-exclusion:int-solns]{Example~\ref*{exa:inclusion-exclusion:int-solns}}?

Let's consider three examples of larger sets of properties. These
properties will come back up during the remainder of the chapter as we
apply inclusion-exclusion to some more involved situations. Recall
that throughout this book, we use the notation $[n]$ for the set
$\{1,2,\dots,n\}$ when $n$ is a positive integer.

\begin{example}\label{exa:inclusion-exclusion:prop-inject}
  Let $m$ and $n$ be fixed positive integers and let $X$ consist of
  all functions from $[n]$ to $[m]$.  Then for each $i=1,2,\dots,m$,
  and each function $f\in X$, we say that $f$ satisfies $P_i$ if there
  is no $j$ so that $f(j)=i$. In other words, $i$ is not in the image
  or output of the function $f$.

  As a specific example, suppose that $n=5$ and $m=3$. Then the
  function given by the table
  \begin{center}
    \begin{tabular}{c|c|c|c|c|c}
      $i$ & 1 & 2 & 3 & 4 & 5\\
      \hline
      $f(i)$ & 2 & 3 & 2 & 2 & 3
    \end{tabular}
  \end{center}
  satisfies $P_1$ but not $P_2$ or $P_3$.
\end{example} 

\begin{example}\label{exa:inclusion-exclusion:prop-derange} 
  Let $m$ be a fixed positive integer and let $X$ consist of all
  bijections from $[m]$ to $[m]$.  Elements of $X$ are called
  \textit{permutations}.  Then for each $i=1,2,\dots,m$, and each
  permutation $\sigma\in X$, we say that $\sigma$ satisfies $P_i$ if
  $\sigma(i)=i$.

  For example, the permutation $\sigma$ of $[5]$ given in by the table
  \begin{center}
    \begin{tabular}{c|c|c|c|c|c}
      $i$ & 1 & 2 & 3 & 4 & 5\\
      \hline
      $\sigma(i)$ & 2 & 4 & 3 & 1 & 5
    \end{tabular}
  \end{center}
  satisfies $P_3$ and $P_5$ and no other $P_i$.
\end{example} 
Note that in the previous example, we could have said that $\sigma$
satisfies property $P_i$ if $\sigma(i)\neq i$.  But remembering that
our goal is to count the number of elements satisfying none of the
properties, we would then be counting the number of permutations
satisfying $\sigma(i)=i$ for each $i=1,2,\dots,n$, and perhaps we
don't need a lot of theory to accomplish this task---the number is
one, of course.

\begin{example}\label{exa:inclusion-exclusion:prop-divis}
  Let $m$ and $n$ be fixed positive integers and let $X=[n]$.  Then
  for each $i=1,2,\dots,m$, and each $j\in X$, we say that $j$
  satisfies $P_i$ if $i$ is a divisor of $j$. Put another way, the
  positive integers that satisfy property $P_i$ are precisely those
  that are multiples of $i$.

  At first this may appear to be the most complicated of the sets of
  properties we've discussed thus far. However, being concrete should
  help clear up any confusion. Suppose that $n=m=15$. Which properties
  does $12$ satisfy? The divisors of $12$ are $1$, $2$, $3$, $4$, $6$,
  and $12$, so $12$ satisfies $P_1$, $P_2$, $P_3$, $P_4$, $P_6$, and
  $P_{12}$. On the other end of the spectrum, notice that $7$
  satisfies only properties $P_1$ and $P_7$, since those are its only
  divisors.
\end{example} 

\section{The Inclusion-Exclusion Formula}\label{s:inclusion-exclusion:statement}

Now that we have an understanding of what we mean by a property, let's
see how we can use this concept to generalize the process we used in
the first two examples of the previous section.

Let $X$ be a set and let $\mathcal{P}=\{P_1,P_2,\dots,P_m\}$ be a
family of properties. Then for each subset $S\subseteq [m]$, let
$N(S)$ denote the number of elements of $X$ which satisfy property
$P_i$ for all $i\in S$.  Note that if $S=\emptyset$, then $N(S)=|X|$,
as every element of $X$ satisfies every property in $S$ (which
contains no actual properties).

Returning for a moment to
\hyperref[exa:inclusion-exclusion:students]{Example~\ref*{exa:inclusion-exclusion:students}}
with $P_1$ being ``is a computer science major'' and $P_2$ being ``is
male,'' we note that $N(\{1\})=47$, since there are $47$ computer
science majors in the class. Also, $N(\{2\})=51$ since $51$ of the
students are male. Finally, $N(\{1,2\})=45$ since there are $45$ male
computer science majors in the class.

In the examples of the previous section, we subtracted off $N(S)$ for
the sets $S$ of size $1$ and then added back $N(S)$ for the set of
properties of size $2$, since we'd subtracted the number of things
with both properties (male computer science majors or solutions with
both $x_3>7$ and $x_4'>8$) twice. Symbolically, we determined that the
number of objects satisfying none of the properties was
\[N(\emptyset) - N(\{1\}) - N(\{2\}) + N(\{1,2\}).\] Suppose that we
had three properties $P_1,P_2$, and $P_3$. How would we count the
number of objects satisfying none of the properties?  As before, we
start by subtracting for each of $P_1$, $P_2$, and $P_3$. Now we have
removed the objects satisfying both $P_1$ and $P_2$ twice, so we must
add back $N(\{1,2\})$. similarly, we must do this for the objects
satisfying both $P_2$ and $P_3$ and both $P_1$ and $P_3$. Now let's
think about the objects satisfying all three properties. They're
counted in $N(\emptyset)$, eliminated \emph{three times} by the
$N(\{i\})$ terms, and added back three times by the $N(\{i,j\})$
terms. Thus, they're still being counted! Thus, we must yet
subtract $N(\{1,2,3\})$ to get the desired number:
\[N(\emptyset) - N(\{1\}) - N(\{2\}) - N(\{3\}) + N(\{1,2\}) +
N(\{2,3\}) + N(\{1,3\}) - N(\{1,2,3\}).\]
We can generalize this as the following theorem:

\begin{theorem}[Principle of Inclusion-Exclusion]\label{thm:inclusion-exclusion}
The number of elements of $X$ which satisfy none of the properties
in $\mathcal{P}$ is given by
\begin{equation}
\sum_{S\subseteq [m]} (-1)^{|S|}N(S).\label{eq:inclusion-exclusion}
\end{equation}
\end{theorem}
\begin{proof}
  We proceed by induction on the number $m$ of properties. If $m=1$,
  then the formula reduces to $N(\emptyset)-N(\{1\})$.  This is
  correct since it says just that the number of elements which do not
  satisfy property $P_1$ is the total number of elements minus the
  number which do satisfy property $P_1$.

  Now assume validity when $m\le k$ for some $k\ge1$ and consider the
  case where $m=k+1$.  Let $X'=\{x\in X: x$ satisfies $P_{k+1}\}$ and
  $X''=X-X'$ (i.e., $X''$ is the set of elements that do not satisfy
  $P_{k+1}$). Also, let $\mathcal{Q}=\{P_1,P_2,\dots,P_k\}$.  Then for
  each subset $S\subseteq [k]$, let $N'(S)$ count the number of
  elements of $X'$ satisfying property $P_i$ for all $i\in S$.  Also,
  let $N''(S)$ count the number of elements of $X''$ satisfying
  property $P_i$ for each $i\in S$.  Note that $N(S)=N'(S)+N''(S)$ for
  every $S\subseteq [k]$.

  Let $X'_0$ denote the set of elements in $X'$ which satisfy none of
  the properties in $\mathcal{Q}$ (in other words, those that satisfy
  only $P_{k+1}$ from $\mathcal{P}$), and let $X''_0$ denote the set
  of elements of $X''$ which satisfy none of the properties in
  $\mathcal{Q}$, and therefore none of the properties in $\mathcal{P}$.

  Now by the inductive hypothesis, we know
  \[
  |X'_0| = \sum_{S\subseteq [k]} (-1)^{|S|}N'(S)\qquad \text{and}\qquad
 |X''_0| = \sum_{S\subseteq [k]} (-1)^{|S|}N''(S).
\]
It follows that
\begin{align*}
|X''_0| &= \sum_{S\subseteq [k]} (-1)^{|S|}N''(S) = \sum_{S\subseteq [k]} (-1)^{|S|}\left(N(S)-N'(S)\right)\\
        &= \sum_{S\subseteq [k]} (-1)^{|S|}N(S)+
           \sum_{S\subseteq [k]} (-1)^{|S|+1}N(S\cup\{k+1\})\\
        &= \sum_{S\subseteq [k+1]} (-1)^{|S|}N(S).
\end{align*}
\end{proof}

\section{Enumerating Surjections}\label{s:inclusion-exclusion:surjections}

As our first example of the power of inclusion-exclusion, consider the
following situation: A grandfather has $15$ distinct lottery tickets
and wants to distribute them to his four grandchildren so that each
child gets at least one ticket. In how many ways can he make such a
distribution? At first, this looks a lot like the problem of
enumerating integers solutions of equations, except here the lottery
tickets are not identical! A ticket bearing the numbers $1$, $3$,
$10$, $23$, $47$, and $50$ will almost surely not pay out the same
amount as one with the numbers $2$, $7$, $10$, $30$, $31$, and $48$,
so who gets which ticket really makes a difference. Hopefully, you
have already recognized that the fact that we're dealing with lottery
tickets and grandchildren isn't so important here. Rather, the
important fact is that we want to distribute distinguishable objects
to distinct entities, which calls for counting functions from one set
(lottery tickets) to another (grandchildren). In our example, we don't
simply want the total number of functions, but instead we want the
number of surjections, so that we can ensure that every grandchild
gets a ticket.

For positive integers $n$ and $m$, let $S(n,m)$ denote the number of
surjections from $[n]$ to $[m]$.  Note that $S(n,m)=0$ when $n<m$.  In
this section, we apply the Inclusion-Exclusion formula to determine a
formula for $S(n,m)$.  We start by setting $X$ to be the set of all
functions from $[n]$ to $[m]$.  Then for each $f\in X$ and each
$i=1,2,\dots,m$, we say that $f$ satisfies property $P_i$ if $i$ is
not in the range of $f$.

\begin{lemma}
  For each subset $S\subseteq [m]$, $N(S)$ depends only on $|S|$.  In
  fact, if $|S|=k$, then
  \[
  N(S)=(m-k)^n.
  \]
\end{lemma}
\begin{proof}
Let $|S|=k$.  Then a
function $f$ satisfying property $P_i$ for each $i\in S$ is
a string of length $n$ from an alphabet consisting of $m-k$ letters.
This shows that \[N(S)=(m-k)^n.\]
\end{proof}

Now the following result follows immediately from this lemma by
applying the Principle of Inclusion-Exclusion, as there are $C(m,k)$
$k$-element subsets of $[m]$.
\begin{theorem}
  The number $S(n,m)$ of surjections from $[n]$ to $[m]$ is given by:
\[
S(n,m) = \sum_{k=0}^m (-1)^k\binom{m}{k}(m-k)^n.
\]
\end{theorem}

For example, 
\begin{align*}
S(5,3) &= \binom{3}{0}(3-0)^5-\binom{3}{1}(3-1)^5+\binom{3}{2}(3-2)^5-\binom{3}{3}(3-3)^5\\
       &= 243 -96+3-0\\
       &= 150.
\end{align*}

Returning to our lottery ticket distribution problem at the start of
the section, we see that there are $S(15,4)=1016542800$ ways for the
grandfather to distribute his $15$ lottery tickets so that each of the
$4$ grandchildren receives at least one ticket.

\section{Derangements}\label{s:inclusion-exclusion:derangements}

Now let's consider a situation where we can make use of the properties
defined in
\hyperref[exa:inclusion-exclusion:prop-derange]{Example~\ref*{exa:inclusion-exclusion:prop-derange}}. Fix a
positive integer $n$ and let $X$ denote the set of all permutations on
$[n]$.  A permutation $\sigma\in X$ is called a \textit{derangement}
if $\sigma(i)\neq i$ for all $i=1,2,\dots,n$. For example, the first
permutation given below is a derangement, while the second is not.
\begin{center}
  \begin{tabular}{c|cccc}
    $i$ & 1 & 2 & 3 & 4\\\hline
    $\sigma(i)$ & 2 & 4 & 1 & 3
  \end{tabular}\hspace{0.3in}
  \begin{tabular}{c|cccc}
    $i$ & 1 & 2 & 3 & 4\\\hline
    $\sigma(i)$ & 2 & 4 & 3 & 1
  \end{tabular}
\end{center}
If we again let $P_i$ be the property that $\sigma(i)=i$, then the
derangements are precisely those permutations which do not satisfy
$P_i$ for any $i=1,2,\dots,n$.

\begin{lemma}
  For each subset $S\subseteq [n]$, $N(S)$ depends only on $|S|$.  In
  fact, if $|S|=k$, then
  \[
  N(S) = (n-k)!
  \]
\end{lemma}

\begin{proof}
  For each $i\in S$, the value $\sigma(i)=i$ is fixed.  The other
  values of $\sigma$ are a permutation among the remaining $n-k$
  positions, and there are $(n-k)!$ of these.
\end{proof}

As before, the principal result of this section follows immediately
from the lemma and the Principle of Inclusion-Exclusion.

\begin{theorem}
For each positive integer $n$, the number $d_n$ of derangements of $[n]$
satisfies
\[
d_n=\sum_{k=0}^n (-1)^k\binom{n}{k}(n-k)!.
\]

\end{theorem}

For example,
\begin{align*}
d_5 & =\binom{5}{0}5!-\binom{5}{1}4!+\binom{5}{2}3!-\binom{5}{3}2!+
   \binom{5}{4}1!-\binom{5}{5}0!\\
    &=120-120+60-20+5-1\\
    &=44.
\end{align*}

It has been traditional to cast the subject of derangements as a
story, called the Hat Check problem. The story belongs to the period
of time when men wore top hats. For a fancy ball, $100$ men check
their top hats with the Hat Check person before entering the ballroom
floor. Later in the evening, the mischeivous hat check person decides
to return hats at random. What is the probability that all $100$ men
receive a hat other than their own? It turns out that the answer is
very close to $1/e$, as the following result shows.

\begin{theorem}
  For a positive integer $n$, let $d_n$ denote the number of
  derangements of $[n]$. Then
  \[
  \lim_{n\rightarrow\infty}\frac{d_n}{n!} = \frac{1}{e}.
  \] 
  Equivalently, the fraction of all permutations of $[n]$ that are
  derangments approaches $1/e$ as $n$ increases.
\end{theorem}

\begin{proof}
It is easy to see that
\begin{align*}
\frac{d_n}{n!} &= \frac{\sum_{k=0}^n (-1)^k\binom{n}{k}(n-k)!}{n!}\\
               &= \sum_{k=0}^n (-1)^k \frac{n!}{k!(n-k)!}\frac{(n-k)!}{n!}\\
               &= \sum_{k=0}^n (-1)^k \frac{1}{k!}.
\end{align*}
Recall from Calculus that the Taylor series expansion of $e^x$ is given
by 
\[
e^x = \sum_{k=0}^{\infty} \frac{x^k}{k!},
\]
and thus the result then follows by substituting $x=-1$.
\end{proof}

Usually we're not as interested in $d_n$ itself as we are in
enumerating permutations with certain restrictions, as the following
example illustrates.

\begin{example}
  Consider the Hat Check problem, but suppose instead of wanting no
  man to leave with his own hat, we are interested in the number of
  ways to distribute the $100$ hats so that precisely $40$ of the men
  leave with their own hats.

  If $40$ men leave with their own hats, then there are $60$ men who
  do not receive their own hats. There are $C(100,60)$ ways to choose
  the $60$ men who will not receive their own hats and $d_{60}$ ways
  to distribute those hats so that no man receives his own. There's
  only one way to distribute the $40$ hats to the men who must receive
  their own hats, meaning that there are
  \begin{align*}
    \binom{100}{60}d_{60} =
    &420788734922281721283274628333913452107738151595140722182899444\\
    &67852500232068048628965153767728913178940196920
    \end{align*}
  such ways to return the hats.
\end{example}

\section{The Euler $\phi$ Function}\label{s:inclusion-exclusion:euler-phi}

After reading the two previous sections, you're probably wondering why
we stated the Principle of Inclusion-Exclusion in such an abstract
way, as in those examples $N(S)$ depended only on the size of $S$ and
not its contents. In this section, we produce an important example
where the value of $N(S)$ \textit{does} depend on $S$.
Nevertheless, we are able to make a reduction to obtain a useful end
result.  In what follows, let $\posints$ denote the set of positive
integers.

For a positive integer $n\ge2$, let 
\[
\phi(n)=|\{m\in \posints: m\le n, \gcd(m,n)=1\}|.
\]
This function is usually called the Euler $\phi$ function or the Euler
totient function and has many connections to number theory. We won't
focus on the number-theoretic aspects here, only being able to compute
$\phi(n)$ efficiently for any $n$.

For example, $\phi(12)=4$ since the only numbers from $\{1,2,\dots,12\}$
that are relatively prime to $12$ are $1$, $5$, $7$ and $11$. As a
second example, $\phi(9)=6$ since $1$, $2$, $4$, $5$, $7$ and $8$ are
relatively prime to $9$.  On the other hand, $\phi(p)=p-1$ when $p$ is 
a prime.  Suppose you were asked to compute $\phi(321974)$.  How
would you proceed?

In \autoref{ch:induction} we discussed a recursive procedure for
determining the greatest common divisor of two integers, and we wrote
a code for accomplishing this task.  Let's assume that we have a
function declared as follows:

\medskip
\noindent
\begin{tt}
int gcd(int m, int n);
\end{tt}

\medskip
\noindent
that returns the greatest common divisor of $m$ and $n$.

Then we can calculate $\phi(n)$ with this code snippet:

\begingroup
\parindent0pt
\begin{tt}
answer = 1;

for  (m = 2; m < n; m++) \{\\
\mbox{}\hspace{.25in} if  (gcd(m,n) == 1) \{\\
\mbox{}\hspace{.5in} answer++;\\
\mbox{}\hspace{.25in} \}\\
\}

return(answer);
\end{tt}
\endgroup

\medskip 

A program called \texttt{phi.c} using the code snippet above answers
almost immediately that $\phi(321974) = 147744$.

On the other hand, in just under two minutes the
program reported that 
\[
\phi(319572943) = 319524480.
\]

So how could we find $\phi(1369122257328767073)$?

Clearly, the program is useless to tackle this beast! It not only
iterates $n-2$ times but also invokes a recursion during each
iteration. Fortunately, Inclusion-Exclusion comes to the rescue.

\begin{theorem}\label{thm:eulerphi}
Let $n\ge2$ be a positive integer and suppose that 
$n$ has $m$ distinct prime factors: $p_1$, $p_2,\dots,p_m$.
Then
\begin{equation}\label{eq:eulerphi}
\phi(n) = n\prod_{i = 1}^{m}\frac{p_i-1}{p_i}.
\end{equation}
\end{theorem}
\begin{proof}  We present the argument when $m=3$. The full result
is an easy extension.

Our argument requires the following elementary proposition
whose proof we leave as an exercise.

\begin{proposition}\label{prop:num-divisible}
Let $n\ge2$, $k\ge1$, and let $p_1,p_2,\dots,p_k$ be distinct primes each
of which divide $n$ evenly (without remainder). 
Then the number of integers from $\{1,2,\dots,n\}$ which are
divisble by each of these $k$ primes is
\[
\frac{n}{p_1p_2\dots p_k}.
\]
\end{proposition}

Then Inclusion-Exclusion yields:

\begin{align*}
\phi(n) &= n -\left(\frac{n}{p_1}+\frac{n}{p_2}+\frac{n}{p_3}\right) +
  \left(\frac{n}{p_1p_2}+\frac{n}{p_1p_3}+\frac{n}{p_2p_3}\right)-\frac{n}{p_1p_2p_3}\\
   &= n \frac{p_1p_2p_3 -(p_2p_3+p_1p_3+p_1p_2)+
      (p_3+p_2+p_1) - 1}{p_1p_2p_3}\\
   &=n \frac{p_1-1}{p_1}\frac{p_2-1}{p_2}\frac{p_3-1}{p_3}.
\end{align*}

\end{proof}

\begin{example}

Maple reports that

\[
1369122257328767073 = (3)^3(11)(19)^4(31)^2(6067)^2
\]
is the factorization of $1369122257328767073$ into primes.
It follows that
\[
\phi(1369122257328767073) = 1369122257328767073
\,\,\frac{2}{3}\,\frac{10}{11}\,\frac{18}{19}\,\frac{30}{31}\,\frac{6066}{6067}.
\]

Thus Maple quickly reports that
\[
\phi(1369122257328767073) =760615484618973600.
\]

\end{example}

\begin{example}
Amanda and Bruce receive the same challenge from their professor, namely
to find $\phi(n)$ when 
\begin{align*}
n  = &31484972786199768889479107860964368171543984609017931\\
   &  39001922159851668531040708539722329324902813359241016\\
   &  93211209710523.
\end{align*}

However the Professor also tells Amanda that $n=p_1p_2$ is the product
of two large primes where
\[
p_1 = 470287785858076441566723507866751092927015824834881906763507
\]
and
\[
p_2 = 669483106578092405936560831017556154622901950048903016651289.
\]
Is this information of any special value to Amanda?  Does it really make
her job any easier than Bruce's?  Would it level the playing field
if the professor told Bruce that $n$ was the product of two primes?
\end{example}

\section{Discussion}

Yolanda said ``This seemed like a very short chapter, at least it did to
me.''  Bob agreed ``Yes, but the professor indicated that the goal
was just provide some key examples.  I think he was hinting at more
general notions of inversion---although I haven't a clue as to what
they might be.''

Clearly aggravated, Zori said ``I've had all I can stand of
this big integer stuff.  This won't help me to earn a living.''
Xing now was uncharacteristically firm in his reply ``Zori.
You're off base on this issue.  Large integers, and specifically
integers which are the product of large primes, are central to
public key cryptography.  If you, or any other citizen, were
highly skilled in large integer arithmetic and could quickly
factor integers with, say $150$ digits, then you would be able
to unravel many important secrets.  No doubt your life would
be in danger.''

At first, the group thought that Xing was way out of bounds---but
they quickly realized that Xing felt absolutely certain of what he
was saying.  Zori was quiet for the moment, just reflecting
that maybe, just maybe, her skepticism over the relevance
of the material in applied combinatorics was unjustified. 

\section{Exercises}

\begin{enumerate}
\item A school has $147$ third graders. The third grade teachers have
  planned a special treat for the last day of school and brought ice
  cream for their students. There are three flavors: mint chip,
  chocolate, and strawberry. Suppose that $60$ students like (at
  least) mint chip, $103$ like chocolate, $50$ like strawberry, $30$
  like mint chip and strawberry, $40$ like mint chip and chocolate,
  $25$ like chocolate and strawberry, and $18$ like all three
  flavors. How many students don't like any of the flavors available?
\item There are $1189$ students majoring in computer science at a
  particular university. They are surveyed about their knowledge of
  three programming languages: C++, Java, and Python. The survey
  results reflect that $856$ students know C++, $748$ know Java, and
  $692$ know Python. Additionally, $639$ students know both C++ and
  Java, $519$ know both C++ and Python, and $632$ know both Java and
  Python. There are $488$ students who report knowing all three
  languages. How many students reported that they did not know any of
  the three programming languages?
\item How many positive integers less than or equal to $100$ are
  divisible by $2$? How many positive integers less than or equal to
  $100$ are divisible by $5$? Use this information to determine how
  many positive integers less than or equal to $100$ are divisible by
  \emph{neither} $2$ nor $5$.
\item How many positive integers less than or equal to $100$ are
  divisible by none of $2$, $3$, and $5$?
\item How many positive integers less than or equal to $1000$ are
  divisible by none of $3$, $8$, and $25$?
\item The State of Georgia is distributing \$$173$ million in funding
  to Fulton, Gwinnett, DeKalb, Cobb, and Clayton counties (in millions
  of dollars). In how many ways can this distribution be made,
  assuming that each county receives at least \$$1$ million, Clayton
  county receives at most \$$10$ million, and Cobb county receives at
  most \$$30$ million? What if we add the restriction that Fulton
  county is to receive at least \$$5$ million (instead of at least
  \$$1$ million)?
\item How many integer solutions are there to the equation $x_1 + x_2
  + x_3 + x_4 = 32$
  with $0\leq x_i\leq 10$ for $i=1,2,3,4$?
\item How many integer solutions are there to the inequality
  \[y_1 + y_2 + y_3 + y_4 < 184\]
  with $y_1>0$, $0<y_2\leq 10$, $0\leq y_3\leq 17$, and $0\leq y_4 <
  19$?
\item A graduate student eats lunch in the campus food court every
  Tuesday over the course of a $15$-week semester. He is joined each
  week by some subset of a group of six friends from across
  campus. Over the course of a semester, he ate lunch with each friend
  $11$ times, each pair $9$ times, and each triple $6$ times. He ate
  lunch with each group of four friends $4$ times and each group of
  five friends $4$ times. All seven of them ate lunch together only
  once that semester. Did the graduate student ever eat lunch alone?
  If so, how many times?
\item A group of $268$ students are surveyed about their ability to
  speak Mandarin, Japanese, and Korean. There are $37$ students who do
  not speak any of the three languages surveyed. Mandarin is spoken by
  $174$ of the students, Japenese is spoken by $139$ of the students,
  and Korean is spoken by $112$ of the students. The survey results
  also reflect that $102$ students speak both Mandarin and Japanese,
  $81$ students speak both Mandarin and Korean, and $71$ students
  speak both Japanese and Korean. How many students speak all three
  languages?
\item As in
  \hyperref[exa:inclusion-exclusion:prop-inject]{Example~\ref*{exa:inclusion-exclusion:prop-inject}},
  let $X$ be the set of functions from $[n]$ to $[m]$ and let a
  function $f\in X$ satisfy property $P_i$ if there is no $j$ such
  that $f(j)=i$.
  \begin{enumerate}
  \item Let the function $f\colon [8]\to [7]$ be defined by the table
    below.
    \begin{center}
      \begin{tabular}{c|c|c|c|c|c|c|c|c}
        $i$ & 1 & 2 & 3 & 4 & 5 & 6 & 7 & 8\\\hline
        $f(i)$ & 4 & 2 & 6 & 1 & 6 & 2 & 4 & 2
      \end{tabular}
    \end{center}
    Does $f$ satisfy property $P_2$? Why or why not? What about
    property $P_3$? List all the properties $P_i$ (with $i\leq 7$)
    satisfied by $f$.
  \item Is it possible to define a function $g\colon [8]\to [7]$ that
    satisfies no property $P_i$, $i\leq 7$? If so, give an example. If
    not, explain why not.
  \item Is it possible to define a function $h\colon [8]\to [9]$ that
    satisfies no property $P_i$, $i\leq 9$? If so, give an example. If
    not, explain why not.
  \end{enumerate}

\item As in
  \hyperref[exa:inclusion-exclusion:prop-derange]{Example~\ref*{exa:inclusion-exclusion:prop-derange}},
  let $X$ be the set of permutations of $[n]$ and say that $\sigma\in
  X$ satisfies property $P_i$ if $\sigma(i) = i$.
  \begin{enumerate}
  \item Let the permutation $\sigma\colon [8]\to [8]$ be defined by the table
    below.
    \begin{center}
      \begin{tabular}{c|c|c|c|c|c|c|c|c}
        $i$ & 1 & 2 & 3 & 4 & 5 & 6 & 7 & 8\\\hline
        $\sigma(i)$ & 3 & 1 & 8 & 4 & 7 & 6 & 5 & 2
      \end{tabular}
    \end{center}
    Does $\sigma$ satisfy property $P_2$? Why or why not? What about
    property $P_6$? List all the properties $P_i$ (with $i\leq 8$)
    satisfied by $\sigma$.
  \item Give an example of a permutation $\tau\colon[8]\to[8]$ that
    satisfies properties $P_1$, $P_4$, and $P_8$ and no other
    properties $P_i$ with $1\leq i\leq 8$.
  \item Give an example of a permutation $\pi\colon [8]\to[8]$ that
    does not satisfy any property $P_i$ with $1\leq i\leq 8$.
  \end{enumerate}

\item As in
  \hyperref[exa:inclusion-exclusion:prop-divis]{Example~\ref*{exa:inclusion-exclusion:prop-divis}},
  let $m$ and $n$ be positive integers and $X=[n]$. Say that $j\in X$
  satisfies property $P_i$ for an $i$ with $1\leq i\leq m$ if $i$ is a
  divisor of $j$.
  \begin{enumerate}
  \item Let $m=n=15$. Does $12$ satisfy property $P_3$? Why or why
    not? What about property $P_5$? List the properties $P_i$ with
    $1\leq i\leq 15$ that $12$ satisfies.
  \item Give an example of an integer $j$ with $1\leq j\leq 15$ that
    satisfies exactly two properties $P_i$ with $1\leq i\leq 15$.
  \item Give an example of an integer $j$ with $1\leq j\leq 15$ that
    satisfies exactly four properties $P_i$ with $1\leq i\leq 15$ or
    explain why such an integer does not exist.
  \item Give an example of an integer $j$ with $1\leq j\leq 15$ that
    satisfies exactly three properties $P_i$ with $1\leq i\leq 15$ or
    explain why such an integer does not exist.
  \end{enumerate}
\item How many surjections are there from an eight-element set to a
  six-element set?
\item A teacher has $10$ books (all different) that she wants to
 distribute to John, Paul, Ringo, and George, ensuring that each of
 them gets at least one book. In how many ways can she do this?
\item A supervisor has nine tasks that must be completed and five
  employees to whom she may assign them. If she wishes to ensure that
  each employee is assigned at least one task to perform, how many
  ways are there to assign the tasks to the employees?
\item A professor is working with six undergraduate research
  students. He has $12$ topics that he would like these students to
  begin investigating. Since he has been working with Katie for
  several terms, he wants to ensure that she is given the most
  challenging topic (and possibly others). Subject to this, in how
  many ways can he assign the topics to his students if each student
  must be assigned at least one topic?
\item List all the derangements of $[4]$. (For brevity, you may write
  a permutation $\sigma$ as a string
  $\sigma(1)\sigma(2)\sigma(3)\sigma(4)$.)
\item How many derangements of a nine-element set are there?
\item A soccer team's equipment manager is in a hurry to distribute
  uniforms to the last six players to show up before a match. Instead
  of ensuring that each player receives his own uniform, he simply
  hands a uniform to each of the six players. In how many ways could
  he hand out the uniforms so that no player receives his own uniform?
  (Assume that the six remaining uniforms belong to the last six
  players to arrive.)
\item A careless payroll clerk is placing employees' paychecks into
  pre-labeled envelopes. The envelopes are sealed before the clerk
  realizes he didn't match the names on the paychecks with the names
  on the envelopes. If there are seven employees, in how many ways
  could he have placed the paychecks into the envelopes so that
  exactly three employees receive the correct paycheck?
\item The principle of inclusion-exclusion is not the only approach
  available for counting derangements. We know that $d_1=0$ and
  $d_2=1$. Using this initial information, it is possible to give a
  recursive form for $d_n$. In this exercise, we consider two
  recursions for $d_n$.
  \begin{enumerate}
  \item Give a combinatorial argument to prove that the number of
    derangements satisfies the recursive formula $d_n =
    (n-1)(d_{n-1}+d_{n-2})$ for $n\geq 2$. (\emph{Hint}: For a
    derangement $\sigma$, consider the integer $k$ with
    $\sigma(k)=1$. Argue based on the number of choices for $k$ and
    then whether $\sigma(1)=k$ or not.)
  \item Prove that the number of derangements also satisfies the
    recursive formula $d_n = nd_{n-1} + (-1)^n$ for $n\geq
    2$. (\emph{Hint}: You may find it easiest to prove this using the
    other recursive formula and mathematical induction.)
  \end{enumerate}
\item Determine $\phi(18)$ by listing the integers it counts as well
  as by using the formula of \autoref{thm:eulerphi}.
\item Compute $\phi(756)$.
\item Given that $1625190883965792 = (2)^5(3)^4(11)^2(13)(23)^3(181)^2$,
  compute \[\phi(1625190883965792).\] 
\item Prove
  \hyperref[prop:num-divisible]{Proposition~\ref*{prop:num-divisible}}.
\item At a very small school, there is a class with nine students in
  it. The students, whom we will denote as $A$, $B$, $C$, $D$, $E$,
  $F$, $G$, $H$, and $I$, walk from their classroom to the lunchroom
  in the order $ABCDEFGHI$. (Let's say that $A$ is at the front of the
  line.) On the way back to to their classroom after lunch, they would
  like to walk in an order so that no student walks immediately behind
  the same classmate he or she was behind on the way to lunch. (For
  instance, $ACBDIHGFE$ and $IHGFEDCBA$ would meet their
  criteria. However, they would not be happy with $CEFGBADHI$ since it
  contains $FG$ and $HI$, so $G$ is following $F$ again and $I$ is
  following $H$ again.) 
  \begin{enumerate}
  \item One student ponders how many possible ways there would be for
    them to line up meeting this criterion. Help him out by
    determining the exact value of this number.
  \item Is this number bigger than, smaller than, or equal to the
    number of ways they could return so that no student walks in the
    same position as before (i.e., $A$ is not first, $B$ is not
    second, \dots, and $I$ is not last)?
  \item What fraction (give it as a decimal) of the total
    number of ways they could line up meet their criterion of no
    student following immediately behind the same student on the
    return trip?
 \end{enumerate}
\end{enumerate}

%%% Local Variables: 
%%% mode: latex
%%% TeX-master: "book"
%%% End: 
